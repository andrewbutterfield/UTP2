\chapter{Issues}

The entries here are being linked to file TODO.txt, most recent
in each category on top.
Each entry is identified by a letter
(\textbf{S},\textbf{C},\textbf{U},\textbf{D}) and number,
most recent being highest.

\section{Soundness}

\section{Completeness}

\subsection*{C--1 Need to support alphabets properly (0.85$\alpha$5)}

Let's present a proof that illustrates the issues that need to be tackled
to handle alphabets properly.
We shall prove the associativity of sequential composition
in a general settings
(i.e. suitable for many theories with different homogenous alphabets).

Definitions and Laws:
\begin{eqnarray*}
  \elabel{DEF-;} &&
   P;Q
   \defs
   \exists \qsep Obs_m @ P[Obs_m/\qsep Obs'] \land Q[Obs_m/\qsep Obs], \mbox{ fresh }m
\\ \elabel{and-E-distr} &&
   N \land (\exists  \qsep xs @ P)
   =
   (\exists \qsep xs @ N \land P), ~xs \notin N
\end{eqnarray*}
The proof (ignoring ACI rearrangements)
\begin{eqnarray*}
  && A ; ( B ; C )
\EQ{\ecite{DEF-;}, $m$ is fresh}
\\&& A ; (\exists \qsep Obs_m @ B[Obs_m/\qsep Obs'] \land C[Obs_m/\qsep Obs])
\EQ{\ecite{DEF-;}, $n$ is fresh}
\\&& \exists \qsep Obs_n @ A[Obs_n/\qsep Obs']
     \land (\exists \qsep Obs_m @ B[Obs_m/\qsep Obs'] \land C[Obs_m/\qsep Obs])[Obs_n/\qsep Obs]
\EQ{perform substitution, $Obs_n \neq Obs_m$}
\\&& \exists \qsep Obs_n @ A[Obs_n/\qsep Obs']
     \land \exists \qsep Obs_m @ B[Obs_m/\qsep Obs'][Obs_n/\qsep Obs] \land C[Obs_m/\qsep Obs][Obs_n/\qsep Obs]
\EQ{compose substitutions, $Obs_n \neq Obs_m$}
\\&& \exists \qsep Obs_n @ A[Obs_n/\qsep Obs']
     \land \exists \qsep Obs_m @ B[Obs_m,Obs_n/\qsep Obs',Obs] \land C[Obs_m/\qsep Obs]
\EQ{\ecite{and-E-distr}, $Obs_m \notin A[Obs_n/\qsep Obs']$}
\\&& \exists \qsep Obs_n,Obs_m @ A[Obs_n/\qsep Obs']
     \land B[Obs_m,Obs_n/\qsep Obs',Obs] \land C[Obs_m/\qsep Obs]
\EQ{\ecite{and-E-distr}, $Obs_n \notin C[Obs_m/\qsep Obs]$}
\\&& \exists \qsep Obs_m @
       (\exists \qsep Obs_n @ A[Obs_n/\qsep Obs'] \land B[Obs_m,Obs_n/\qsep Obs',Obs])
        \land C[Obs_m/\qsep Obs]
\EQ{de-compose substitutions, $Obs_n \neq Obs_m$}
\\&& \exists \qsep Obs_m @
       (\exists \qsep Obs_n @ A[Obs_n/\qsep Obs'][Obs_m/\qsep Obs']
           \land B[Obs_n/\qsep Obs][Obs_m/\qsep Obs'])
        \land C[Obs_m/\qsep Obs]
\EQ{undo substitution, $Obs_n \neq Obs_m$}
\\&& \exists \qsep Obs_m @
       (\exists \qsep Obs_n @ A[Obs_n/\qsep Obs']
           \land B[Obs_n/\qsep Obs])[Obs_m/\qsep Obs']
        \land C[Obs_m/\qsep Obs]
\EQ{\ecite{;-def}}
\\&& \exists \qsep Obs_m @ (A ; B )[Obs_m/\qsep Obs'] \land C[Obs_m/\qsep Obs]
\EQ{\ecite{;-def}}
\\&& (A ; B ) ; C
\end{eqnarray*}
In practise we expect this to be done using the strategy of reducing
both sides to the same thing, with the addition of
some $\alpha$-equivalence at the end.

Now we look at some proof-steps in detail
and comment on how this might be implemented.

The first step
\begin{eqnarray*}
  && A ; ( B ; C )
\EQ{\ecite{DEF-;}, $m$ is fresh}
\\&& A ; (~\exists \qsep  Obs_m @ B[Obs_m/\qsep Obs'] \land C[Obs_m/\qsep Obs]~)
\end{eqnarray*}
matches goal $B;C$ against definition law
\begin{equation*}
  P;Q
   ~~\defs~~
   \exists \qsep Obs_m @ P[Obs_m/\qsep Obs'] \land Q[Obs_m/\qsep Obs], \mbox{ fresh }m
\end{equation*}
The pattern matching of $(B;C)$ against $(P;Q)$ succeeds with binding
$\setof{P \mapsto B, Q \mapsto C}$.
During instantiation the requirement to instantiate $Obs_m$
is met by choosing a fresh $m$ as required and then building the
rhs of the definition using the bindings. The root $\qsep Obs$
is left unchanged.

The second step is similar,
except that as $m$ is no longer fresh, we find a different value
instead, here denoted as $n$.

The third step
\begin{eqnarray*}
  && \exists \qsep Obs_n @ A[Obs_n/\qsep Obs']
     \land (\exists \qsep Obs_m @ B[Obs_m/\qsep Obs'] \land C[Obs_m/\qsep Obs])[Obs_n/\qsep Obs]
\EQ{perform substitution, $Obs_n \neq Obs_m$}
\\&& \exists \qsep Obs_n @ A[Obs_n/\qsep Obs']
     \land \exists \qsep Obs_m @ B[Obs_m/\qsep Obs'][Obs_n/\qsep Obs] \land C[Obs_m/\qsep Obs][Obs_n/\qsep Obs]
\end{eqnarray*}
is a (built-in) application of the following law of substitutions:
\begin{equation*}
 (\exists \qsep xs @ P)[es/\qsep ys]
 =
 (\exists \qsep xs @ P[es/\qsep ys]), ~ ys \neq xs
\end{equation*}
which is the usual one generalised to work with $\qsep xs$ variables.

Step four
\begin{eqnarray*}
  && \exists \qsep Obs_n @ A[Obs_n/\qsep Obs']
     \land \exists \qsep Obs_m @ B[Obs_m/\qsep Obs'][Obs_n/\qsep Obs] \land C[Obs_m/\qsep Obs][Obs_n/\qsep Obs]
\EQ{compose substitutions, $Obs_n \neq Obs_m$}
\\&& \exists \qsep Obs_n @ A[Obs_n/\qsep Obs']
     \land \exists \qsep Obs_m @ B[Obs_m,Obs_n/\qsep Obs',Obs] \land C[Obs_m/\qsep Obs]
\end{eqnarray*}
is also part of the builtin substitution mechanism
implementing the following laws of substitution:
\begin{eqnarray*}
  E[es/\qsep xs][fs/\qsep ys] &=& E[es,fs/\qsep xs,ys], \quad xs \DISJ ys \land ys \DISJ es
\\ E[es/\qsep xs][fs/\qsep xs] &=& E[es/\qsep xs], \quad xs \notin es
\end{eqnarray*}
Given our intended interpretation of $\qsep Obs_x$ we have the laws
\RLEQNS{
   Obs &\DISJ& Obs'
\\ Obs &\DISJ& Obs_x, & \textrm{any}~x
\\ Obs' &\DISJ& Obs_x, & \textrm{any}~y
\\ Obs_x &\DISJ& Obs_y, & x \neq y
}

In the fifth step
\begin{eqnarray*}
  && \exists \qsep Obs_n @ A[Obs_n/\qsep Obs']
     \land \exists \qsep Obs_m @ B[Obs_m,Obs_n/\qsep Obs',Obs] \land C[Obs_m/\qsep Obs]
\EQ{\ecite{and-E-distr}, $Obs_m \notin A[Obs_n/\qsep Obs']$}
\\&& \exists \qsep Obs_n,Obs_m @ A[Obs_n/\qsep Obs']
     \land B[Obs_m,Obs_n/\qsep Obs',Obs] \land C[Obs_m/\qsep Obs]
\end{eqnarray*}
we match goal interior
\begin{equation*}
   A[Obs_n/\qsep Obs']
   \land \exists \qsep Obs_m @ B[Obs_m,Obs_n/\qsep Obs',Obs] \land C[Obs_m/\qsep Obs]
\end{equation*}
 against the lefthand side of the law
\begin{equation*}
N \land (\exists  \qsep xs @ P)
   ~~=~~
   (\exists \qsep xs @ N \land P), \quad xs \notin N
\end{equation*}
with initial bindings
\begin{eqnarray*}
  N &\mapsto& A[Obs_n/\qsep Obs']
\\ \qsep xs &\mapsto& \qsep Obs_m
\\ P &\mapsto& B[Obs_m,Obs_n/\qsep Obs',Obs] \land C[Obs_m/\qsep Obs]
\end{eqnarray*}
with an evaluation of the instantiated side condition
$(\qsep Obs_m \notin A[Obs_n/\qsep Obs'])$.
The law for free variables of a substitution is
\begin{equation*}
 \fv.(A[es/\qsep xs])
 =
 (\fv.A \setminus xs)
 \union
 \bigcup_{i \in \#xs}\left\{\fv.e_i\cond{x_i \in fv.A}\emptyset\right\}
\end{equation*}
We need to be able to conclude that:
\begin{itemize}
  \item $Obs_m \DISJ Obs_n$, which is straightforward.
  \item $Obs_m \DISJ \fv.A$, but we seem to be stymied, because
  we know nothing about the free variables of $A$.
  We could have had this information specified as a side-condition
  of the original goal, i.e. something like $\fv.A \subseteq \setof{\qsep Obs,Obs'}$.
  Instead, however,  we exploit the fact that $n$ is \emph{fresh},
  and use a key assumption, that no fresh variable is free in any
  original meta-variable like $A$.
  This means that the freshness requirement for $\qsep Obs_x$
  has to be handled carefully and consistently.
\end{itemize}


\section{Useability}

\section{Documentation}

\section{old issues}

\textbf{All of this material will be reorganised and indexed
from TODO.txt}

\subsection{Live --- Autumn 2009}


\begin{description}
  \item[Butterfield, 2009-10-15]
    Induction strategy may need to check user supplied induction variable
    for freshness.
  \item[Butterfield, 2009-10-15]
    Automate full LaTeX generation of a Theory
    --- with proofs in individual files, as appendices,
    and cross-referencing.
  \item[Butterfield, 2009-10-09]
    We need to be very careful with partial function definitions.
    The following set of laws (Theory \texttt{Lists}) are inconsistent:
    \begin{eqnarray*}
       (1) && (x:xs) \neq \nil
    \\ (2) && \seqof x = x : \nil
    \\ (3) && front \seqof x = \nil
    \\ (4) && front (x:xs) = x : front (xs)
    \end{eqnarray*}
    Demonstration:
    \begin{eqnarray*}
      && \nil
    \EQ{ (3) backwards}
    \\&& front\seqof x
    \EQ{ (2) }
    \\&& front(x : \nil)
    \EQ{ (4) }
    \\&& x: front\nil
    \\&& \coz{we now contradict (1), with $xs$ instantiated by $front\nil$}
    \end{eqnarray*}
    The solution is to rewrite (4) as either (4a) or (4b):
    \begin{eqnarray*}
       (4a) && xs \neq \nil \implies front(x:xs) = x:front(xs)
    \\ (4b) && front(x:y:ys) = x:front(y:ys)
    \end{eqnarray*}
    We need to ensure that functions defined pattern matching style as
    above have either mutually exclusive cases, or agree on overlaps
  \item[Butterfield, 2009-09-23]
    \texttt{tlequiv} treats all \texttt{Tvar}s as equivalent
    \\ should check \texttt{Tvar} bindings for consistency on each side
    \\ \textbf{We need tMatch for types}
  \item[Bhandal 2009-07-27]
     Law reduce to allow substitutions
     \textit{
       Looking into it \ldots
     }
\end{description}



\subsection{Open}

Refactoring the \LaTeX\ parsing and pretty-printing code:
\begin{enumerate}
  \item
    First, identify all the \LaTeX-ASCII translation tables:
    \begin{itemize}
       \item
         \texttt{LaTeXSetup.laTeXRefToMathMap},
         used in \texttt{LaTeXSetup.stdLaTeXLayout};
         itself used in \texttt{LaTeXPretty} and \texttt{SAOITHIN}.
       \item
         \texttt{LaTeXSetup.latexASciiMapping},
         used in \texttt{LaTeXSetup.latex2asciiTrie};
         not used anywhere else !
       \item
         \texttt{LaTeXSetup.laTeXTokens}
         used in {LaTeXLexer}.
    \end{itemize}
\end{enumerate}


\subsubsection{Fundamental}

\begin{description}
  \item[Invoking definedness on \texttt{Obs} predicate]
   For conjecture Builtin\$0\$offset-pfx, namely that $ s \leq s\cat(tr'-tr) = tr \leq tr'$,
   we cannot get definedness to apply.
   The issue seems to be that we need a way to check laws that don't mention
   definedness, in order to add it in if needed.
   We don't want to have to have a law:
   \begin{eqnarray*}
     \ecite{pfx-cat} && s \le s \cat t \defs \Defd(s) \land \Defd(t)
   \end{eqnarray*}
   Instead we prefer the terser:
   \begin{eqnarray*}
     \ecite{pfx-cat} && s \le s \cat t \defs True
   \end{eqnarray*}
   And rely on something else to catch, in the following case, that
   $s \le s \cat (tr'-tr)$ needs to have $tr \le tr'$ as a side-condition.
   This would also obviate the need for Builtin laws
   saying, for example that $\Defd(a+b) = \Defd(a) \land \Defd(b)$.
   All of this could be inferred (what about non-strict funtions/operators ?).
  \item[Genericity]
    we need to carefully explore generic theory issues,
    such as reactive algebra, parameterised over types and predicates,
    or generic laws across all theories such as healthy idempotence.
    (See \texttt{Design} module).
    \\
    We will need a formal notation to describe how theories interrelate,
    so RD (reactive designs) can inherit from
    \texttt{R}, \texttt{RAlg} and \texttt{Design}, for instance (Zeyda work?).
  \item[H-overloading]
    We may need to break \H{}-overloading, distinguishing \mkH{} from \isH{}.
\end{description}


\subsubsection{I want it now !}

\begin{description}
  \item[prime matching]
    It would be nice to have variable patterns that specify matching
    against pre-, post- or both types of observation variable,
    to allow us to encode laws like:
    $$
      (P \land b');Q \equiv P;(b \land Q)
    $$
    or
    $$
     (b \land P);Q \equiv b \land (P;Q)
    $$
    \\
    \textit{
      We need a convention for qvars that matches any,
      or just pre- or post-variables.
      We also need to ensure that replacement/instantiation is correct.
      If we match $tr'$ against $x'$ (say), and the replacement q-var
      in the law is $x$, then we need to instantiate it as $tr$.
      For compatibility with existing laws, we retain $x$ as matching
      anything and use $`x$ and $x`$ to match pre- and post
      variables respectively.
      \\
      Will be covered by new meta-matching scheme.
    }
  \item[Substitution Composition]~
     $$(P[s,s\cat(tr'-tr)/tr,tr'])[tr/s] \equiv P[tr,tr\cat(tr'-tr)/s,tr']$$,
     but given that $s$ is not free in $P$ we should be able to get
     $P[tr\cat(tr'-tr)/tr']$.
  \item[Error Wrapper]
    We want a predicate function that signals a "should never happen error",
    but which is defined as identity, so application makes it disappear.
    \\\textit{
      partially in place, but not sure about it now.
    }
  \item[Focus within substitutions]
    For proofs (particularly involving \RR2) we need to be able
    to focus on individual expressions within substitutions.
  \item[Law Variants]
    Laws like $A \land (A \lor B) \equiv A$,
    should occur in commutative variants as well.
  \item[Law Combinations]
    Merging splitting with other laws would be useful.
    \\
    \textit{Idea, manually applying a law, but where the law
    uses context-based transformations like splitting to prepare the predicate
    for matching against the law --- perhaps we define a "normal-form"
    for expressing these rules}
  \item[Simultaneous one-point law]
    We would like the one-point law to work with a number of simultaneous
    equalities (think $STET$).
  \item[ACI handling]
    We need an effective way to handle assoc./comm./idem. manipulation,
    particularly the ability to single out sub-terms according to some criteria.
  \item[Pvar conflict, matching and building]
    Matching $RH$ gets a hit on law $\CSP1(RH)\equiv RH$,
    but the substitution when the law is applied binds $CSP1$
    to its expansion --- perhaps should suppress this.
    Interestingly, the ``Replacement'' shown in the Match window
    does not have the expansion applied~!
    \\
    Also, this law is ranked very poorly as it is a TREqv instance
    --- perhaps it should only be ranked this way if the trivial rhs
    is a Pvar not in the predicate table (introduce a better context-sensitive
    ranking scheme for matches).
  \item[Undefinedness for Expressions]
    We want laws for handling the issue of undefinedness.
\end{description}

\subsubsection{Ma\~nana}

\begin{description}
  \item[Alphabetised Predicates]
    What UTP is really about~!
    \\\textit{Developing the use of meta-variables to make this implicit}
  \item[Display Options and Enhancements]
    Nice display, pretty printed, of predicates.
    \\\textit{
      Karen Forde did this for on-screen in a separate branch,
      yet to be integrated. We have a \LaTeX\ pretty-printer (Simon Dardis)
    }
\end{description}

