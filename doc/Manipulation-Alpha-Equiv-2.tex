\subsubsection{Determining $\alpha$-Equivalence}

Variables are equivalent if identical and neither is bound,
or they are both in their respective bound sets,
in which case they can be mapped:
\begin{eqnarray*}
   \alfBndVarL  &\defs& \alfBndVarR  \qquad \alfBndVarN
\\ \alfBndLstVL  &\defs& \alfBndLstVR  \qquad \alfBndLstVN
\\ \alfFreeVarL &\defs& \alfFreeVarR \qquad \alfFreeVarN
\\ \alfFreeLstVL &\defs& \alfFreeLstVR \qquad \alfFreeLstVN
\end{eqnarray*}
For composites, we recurse down,
and then merge the resulting bindings,
which may fail:
\begin{eqnarray*}
   \alfCompL &\defs& \alfCompR, \qquad \alfCompN
\end{eqnarray*}
When we descend through a binder
we extend the bound variable sets,
(which should always have the same cardinality),
and we then take the returned mappings,
merge with the possible bijections
and finally remove any entries not mentioned in either of the original bound sets:
\begin{eqnarray*}
   \lefteqn{\alfQuantL, \qquad \alfQuantN}
\\ &\defs& \alfQuantR
\\ && \WHERE
        \alfQuantS, \mbox{ also respecting Std/Lst distinction}
\end{eqnarray*}
We define a map restriction  ($|$) that removes any maplet
if neither of its components are in the correspond set in the specified pair,
as a lifting of the following pointwise definition:

\ALFMAPRESTRICT

Any (predicate/expression) meta-variables must match themselves,
with an exception for regular variables
which can match expression meta variables of the same name
provided they are not known or bound
\begin{eqnarray*}
   \alfFreeMVarL &\defs& \alfFreeMVarR, \qquad \alfFreeMVarN
\end{eqnarray*}
Substitutions can be viewed as complex composites,
where we expect ordering of substitutions to matter (for now):
\begin{eqnarray*}
   \lefteqn{\alfSubL, \qquad \alfSubN}
\\ &\defs&
   \alfSubR
\\ && \WHERE
        \alfSubS, \mbox{ also respecting Std/Lst distinction}
\end{eqnarray*}
