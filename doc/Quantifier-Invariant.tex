A (quantifier) variable-list is well-formed if
\begin{itemize}
  \item all variables occur at most once
  \item all reserved list-variables have possible denotations that are disjoint
  \item no standard observation variable is unavoidably covered by a reserved
    denotation.
\end{itemize}
The terms ``possible'' and ``unavoidably'' cover the cases where reserved
list-variables have root subtractions that are not known observables,
and so can only be determined at matching time, when a binding of such
roots are available.
In effect, we define two binary relations:
\begin{enumerate}
  \item
    one ($\disjRSV$), between two reserved list-variables,
    asserts that there exists a binding
    that makes their denotations disjoint (``possible disjointness'').
  \item the other ($\notObsInRSV$), between a known standard observation variable
  and a reserved-variable
   assert the existence of a binding that ensures the former is not contained in the denotation
   of the latter (``can escape'').
\end{enumerate}
We express these relationships in terms of the pair-denotations:
\begin{eqnarray*}
   (V_1\ominus X_1) \disjRSV (V_2\ominus X_2)
   &\defs&
   \exists \mu @ (V_1\setminus \mu X_1) \cap (V_2\setminus \mu X_2) = \emptyset
\\ x \notObsInRSV (V \ominus X)
   &\defs&
   \exists \mu @ x \notin (V \setminus \mu X)
\end{eqnarray*}
These relations are statically decidable (without knowing bindings)
only when the all subtracted roots are themselves known observations,
or they are absent, or equal:
\begin{eqnarray*}
   (V_1\ominus \emptyset) \disjRSV (V_2\ominus \emptyset)
   &\equiv&
   V_1 \cap V_2 = \emptyset
\\ V_1\ominus \emptyset) \disjRSV (V_2\ominus X_2)
   &\equiv&
   \exists \mu @ V_1 \cap (V_2\setminus \mu X_2) = \emptyset
\\ (V \ominus X) \disjRSV (V\ominus X_)
   &\equiv&
   \False
\\ x \notObsInRSV (V \ominus \emptyset)
   &\equiv&
   x \notin V
\end{eqnarray*}
If the $X_i$ contain list-variables only, then such bindings always exist.
In some situations, with std-variables in $X_i$, no such binding may exist.
For now we don't try to tease out these cases further, but just note that
the bindings that matter will be those forced by the pattern matching
of an entire goal predicate, which will force matters.

In the event that no observation variables have been defined,
i.e., in a ``generic'' UTP theory that defines common notions such
a sequential composition and/or refinement,
then the disjoint tests above will always succeed as all reserved variables
will have the empty-set as their denotation.
In this case we need to check the variables directly:
\begin{eqnarray*}
   Obs^d \disjRSV Mdl^e &\equiv& d \neq e
\\ Obs^d \disjRSV Scr^e &\equiv& d \neq e
\\ Mdl^d \disjRSV Scr^e &\equiv& \True
\end{eqnarray*}
Note that here there are no subtractions.
