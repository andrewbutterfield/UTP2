\section{Overview}

The logic of UTP is a variant of the equational logic
espoused by David Gries\cite{gries.93}, with extensions as described below.
In formal terms we take the work of Tourlakis \cite{journals/logcom/Tourlakis01},
and adapt this appropriately.

\section{Syntax}

Particularly in the implementation, we use the terminology
``Expression'' and ``Predicate''
where many logicians use ``Term'' and ``Formula''.
We do not have a clean separation of the two,
as we have expressions that define values with the aid of predicates
(e.g. set comprehensions, and unique values).
Other features to note are the presence of explicit substitutions
in the object language---they are not simply part of the meta-language
defining inference rules---and the use of explicit pattern-matching
meta-variables in quantifier variable lists.

We use the following syntax notation, where
$::=~|~{}^*~{}^+~[~]~(~)~\litm~$ have special meaning:

\begin{tabular}{|c|l|}
  \hline
  symbol & meaning
\\\hline
  $::=$ & is defined to be
\\
  $|$ & separates alternatives
\\
  $x^*$ & zero or more $x$
\\
  $x^+$ & one or more $x$
\\
  $x^{(*|+)}_s$ & (zero$|$one) or more $x$ separated by $s$
\\
  $[x]$ & optional $x$
\\
  $(\ldots)$ & grouping
\\
  $\litm[$ & the symbol itself
\\\hline
\end{tabular}

Any other character/symbol denotes itself,
and writing $x \in NonT ::= \ldots$ lets $x$ stand for anything that
satisfies the the definition of non-terminal $NonT$
that occurs on the righthand side of the $::=$.

\subsection{Names}


We have a number of variable namespaces (Fig. \ref{fig:UTP2:variables}):
\begin{description}
  \item[Observation Variables]
    correspond to term (or ``ordinary'' variables) in predicate logic,
    and capture observations of the system behaviour being modelled.
  \item[Constants]
    We take all constants (Numeric etc) as ``names'' for their values.
  \item[Function Names]
   are used in applications, as we do not yet support full higher order
   function notation.
  \item[Type Variables]
   are used to support a simple polymorphic type system,
   whose main purpose is to prevent spurious matches.
  \item[Expression Names]
   are names that denote arbitrary expressions
   (known as term ``schematic variables'' in some circles).
  \item[Predicate Names]
   are names that denote arbitrary predicates
   (also known as formula ``schematic variables'').
  \item[Variable-List Names]
   are names that denote a list of variables,
   in contexts where such lists make sense 
   (e.g., quantifier and substitution variable lists).
   These are distinguished from Observation Variables, Expression and Predicate
   Names, is that they have a dollar postfix, e.g, $\lst x, \lst E, \lst P$.
   The only exceptions are the names $OBS$, $MDL$ and $SCR$ which are taken as
   Variable-list variables whose meaning is always a (sorted) list of
   known observation variables (all, model and script respectively).
\end{description}

\begin{figure}
  \framebox{\VARMATHSYNTAX}
  \caption{\UTP2 Variables}
  \label{fig:UTP2:variables}
\end{figure}

The key distinction between observation variables
and metavariables is what they represent:
\begin{description}
  \item[Observations]
     are variables that denote values in some
    kind of semantic domain, with some form of typing used
    to distinguish different parts of that domain.
    The meaning of an expression with free observation variables
    has to be defined w.r.t an environment mapping names to values.
    \begin{eqnarray*}
     \rho \in Env &=& Name \pfun Val
    \end{eqnarray*}
  \item[Metavariables]
     denote expression or formula syntax.
     The meaning of an expression or predicate with free metavariables
     has to be defined w.r.t. an meta-environment (interpretation) mapping names
     to expressions or predicates.
     \begin{eqnarray*}
        \Phi \in EInt &=& Name \pfun Expr \\
     \\ \Psi \in PInt &=& Name \pfun Pred
     \end{eqnarray*}
  \item[List Variables]
    are variables that match lists of variables, where such lists
    make sense.
\end{description}
A key difference between observations $x,y$ and metavariables $X,Y$
is the  outcome of asking for free variables:
\begin{eqnarray*}
  \fv (x \oplus y)&=& \setof{x,y}
\\ \fv (X \bigoplus Y) &=& \fv.X \cup \fv.Y
\end{eqnarray*}
The free variables of a metavariable are dependent on the
the meta-environment (interpretation) in place.
The requirement to handle free-variables in this way leads
to the need to have an explicit notation to describe free-variable sets.

\newpage
\subsection{Types (Sorts)}

We have the type syntax shown in Figure \ref{fig:UTP2:types}.
\begin{figure}
\boxedm{\TYPEMATHSYNTAX}
  \caption{\UTP2 Type Syntax}
  \label{fig:UTP2:types}
\end{figure}
We do not have naturals or reals as basic types,
although the former will probably be introduced shortly,
to support probabilistic reasoning.
The $Env$ type is a placeholder for name environments,
where the range type is often not expressible using the
above type language, because it is some form of universal type.


\subsection{Expressions (Terms)}

The core expression syntax (Fig. \ref{fig:UTP2:expressions})
has constants, variables
(for both values and expressions),
application of (named) functions%
\footnote{%
We use a definition table to connect function names to abstractions
---we should really do a proper higher-order expression model here,
but it's not a high priority right now.}
,
and abstractions and substitutions over observation, expression and predicate variables.
Also provided are equality, and definite descriptions, which has predicate components (to be defined).
Not shown are the explicit support for binary operators,
or booleans, integers, sets, lists and maps.
\begin{figure}
\framebox{\EXPRMATHSYNTAX}
  \caption{\UTP2 Expression Syntax}
  \label{fig:UTP2:expressions}
\end{figure}
We note here, that unlike terms in \cite{journals/logcom/Tourlakis01},
our expressions here have quantifiers, and may contain predicates
(well-founded formulas) as sub-components.


\subsection{Predicates (Formulas)}

The core logic is higher order,
with quantification, abstraction and substitution for variables, expressions and predicates.
In particular we have predicates that use predicate-sets
(useful for recursion theory, in particular).
\begin{figure}
\framebox{\PREDMATHSYNTAX}
  \caption{\UTP2 Predicates (and Sets)}
  \label{fig:UTP2:predicates}
\end{figure}
We also have a simple polymorphic type-system, and a predicate asserting that
an expression has a specified type---details to be added.
This means that our logic is in fact many-sorted.


\section{Substitution}

The meaning of $A[x:=t]$ in Tourlakis is not given
explicitly, but is discussed, particularly with regard
to the  side-condition designed to avoid variable capture ($t$ substitutable in $x$ (in $A$)).
We shall define $P[x_1,\ldots,x_n:=e_1,\ldots,e_n]$
as meta-notation describing the simultaneous syntactical substitution
of each $e_i$ for all free occurrences of the corresponding $x_i$ in $P$, with $\alpha$-renaming being
used to avoid name-capture. We will write this in shorthand as $[\vec x:=\vec e]$.

So we have explicit substitutions in the object language (written $[e/x]$),
and syntactical-substitutions used to define axiom schemas and inference rules
(written $[x:=e]$).
We use the notation $[\vec x := \vec e]\hide\vec y$
to denote a substitution with all entries $(x_i,e_i)$ removed
where $x_i$ is a member of $\vec y$.
We also have a notion of substitution composition,
denoted by juxtaposition.

There is also another complication,
given that we can abstract over different classes of variables.
Consider the following example (with $X$ used as a predicate meta-variable,
to make it stand out):
$$
  ((\Lambda X @ Q \lor X)R)[e/x]
$$
If we evaluation this by reducing the predicate application first we obtain:
\begin{eqnarray*}
  && ((\Lambda X @ Q \lor X)R)[e/x]
\EQV{Predicate $\beta$-reduction}
\\&& (Q \lor R)[e/x]
\EQV{Defn. of observation substitution}
\\&& Q[e/x] \lor R[e/x]
\end{eqnarray*}
Now let's do the substitution first:
\begin{eqnarray*}
  && ((\Lambda X @ Q \lor X)R)[e/x]
\EQV{Defn.of observation substitution}
\\&& ((\Lambda X @ Q \lor X)[e/x])(R[e/x])
\end{eqnarray*}
The question now concerns how we evaluate
$$
(\Lambda X @ Q \lor X)[e/x].
$$
If we say that the quantifier applies to predicate variables,
but not to observation variables, so it can be ignored,
then we obtain
$$
  \Lambda X @ Q[e/x] \lor X[e/x]
$$
If we do this and then $\beta$-reduce, we obtain
$$
  Q[e/x] \lor (R[e/x])[e/x]
$$
This is not the correct outcome, as substitutions are not idempotent
(consider $e=x+1$ in this example).

So we see that the rule for substitution here is
context dependent --- we cannot apply an observation substitution
to any $\Lambda$-bound predicate (or expression) meta-variable.
This generalises to whenever we substitute for one type
of variable (obs/expr/pred) under a lambda binding a different kind.
If the variables are $\forall$ or $\exists$-bound,
then the substitution is safe.

In Figure \ref{fig:UTP2:obs-expr-Expr-subst}
we show the basic definition of syntactical substitution of expressions
for observations, in expressions,
without details of how variable capture is avoided.
Following figures deal with predicates, and substitution
for predicate and expression meta-variables
\begin{figure}
  \boxedm{
    \DEFOEESUBST
  }
  \caption{%
    Observation substitution for Expressions
     (modulo $\alpha$-renaming to avoid variable capture)
  }
  \label{fig:UTP2:obs-expr-Expr-subst}
\end{figure}

\begin{figure}
  \boxedm{$$
  \DEFOEPSUBST{fig:UTP2:obs-expr-Expr-subst}
$$}
  \caption{%
    Observation substitution for Predicates
     (modulo $\alpha$-renaming to avoid variable capture)
  }
  \label{fig:UTP2:obs-expr-Pred-subst}
\end{figure}

\begin{figure}
  \boxedm{
  \DEFEEESUBST
  }
  \caption{%
    Expression substitution for Expressions
     (modulo $\alpha$-renaming to avoid variable capture)
  }
  \label{fig:UTP2:expr-expr-Expr-subst}
\end{figure}

\begin{figure}
  \boxedm{
  \DEFPPESUBST
  }
  \caption{%
    Predicate substitution for Expressions
     (modulo $\alpha$-renaming to avoid variable capture)
  }
  \label{fig:UTP2:pred-pred-Expr-subst}
\end{figure}


\begin{figure}
  \boxedm{$$
  \DEFEEPSUBST{fig:UTP2:expr-expr-Expr-subst}
$$}
  \caption{%
    Expression substitution for Predicates
     (modulo $\alpha$-renaming to avoid variable capture)
  }
  \label{fig:UTP2:expr-expr-Pred-subst}
\end{figure}

\begin{figure}
  \boxedm{$$
  \DEFPPPSUBST{fig:UTP2:pred-pred-Expr-subst}
$$}
  \caption{%
    Predicate substitution for Predicates
     (modulo $\alpha$-renaming to avoid variable capture)
  }
  \label{fig:UTP2:pred-pred-Pred-subst}
\end{figure}

\newpage
\section{Free Variable Set Notation}
In determining free observation variables we have to deal with the
presence of meta-variables denoting arbitrary predicates and expressions,
as well as the presence of explicit substitutions, and quantifier list-variables
that can denote lists of variables or expressions.
This means that the free variables of a predicate or expression
are contingent on not just the meta/list-variables,
but whether or not certain observation variables are free/present in (any
instantiation of) those meta/list-variables.

We shall use the lambda calculus to illustrate the consequences
of having explicit substitution and quantifier-list matching in our logic.

We start with the untyped lamdba-calculus ($L_0$) where we have an unbounded set of variables ($v \in V$):
\begin{eqnarray*}
   v,w,x \in V && \mbox{Given.}
\\ e,f \in L_0 & ::=& v | e~e| \lambda v @ e
\end{eqnarray*}
with $\lambda x_1,x_2,\ldots,x_n @ e$
as syntactic sugar for
$\lambda x_1 @ \lambda x_2 @ \ldots \lambda x_n @ e$,
itself often shortened to $\lambda \vec x @ e$.

Notation aside: We shall assume that $\vec a$ is shorthand for $a_1,\ldots,a_n$,
for $n \geq 0$, and $a_i$ will refer to the $i$th component, or indicate
an iteration $i \in 1\ldots n$, depending on context.

The free variables for $L_0$ are defined in the usual way:
\begin{eqnarray*}
   S_0 &=& \power V
\\ \fv,\fv_0 &:& L_0 \fun S_0
\\ \fv(v) &\defs& \setof v
\\ \fv(f~e) &\defs& \fv(f) \union \fv(e)
\\ \fv(\lambda v @ e) &\defs& \fv(e) \setminus \setof v
\end{eqnarray*}

So is substitution:
\begin{eqnarray*}
   \_[\_:=\_] &:& L_0 \fun V \times L_0 \fun L_0
\\ v[x:=e] &\defs&  e \cond{x=v} v
\\ (f~e)[x:=e] &\defs& (f[x:=e])~(e[x:=e])
\\ (\lambda v @ f)[x:=e]
   &\defs&
   \left\{
     \begin{array}{ll}
       \lambda v @ f, & x=v \\
       \lambda v @ f[x:=e], & x\neq v \land v \notin \fv(e)  \\
       \lambda w @ f[v:=w][x:=e], & x\neq v \land v \in \fv(e) \land \mbox{new}~w
     \end{array}
   \right.
\end{eqnarray*}
This extends to simultaneous substitutions as follows
(here $\vec x$ has no duplicates):
\begin{eqnarray*}
   \_[\_:=\_] &:& L_0 \fun (V^n \times (L_0)^n) \fun L_0
\\ e[\nil:=\nil] &\defs& e, \mbox{ the empty case}
\\ v[\vec x:=\vec e] &\defs&  e_i \cond{v=x_i} v, \mbox{ for some }i
\\ (f~e)[\vec x:=\vec e] &\defs& (f[\vec x:=\vec e])~(e[\vec x:=\vec e])
\\ (\lambda v @ f)[\vec x:=\vec e]
   &\defs&
   \left\{
     \begin{array}{ll}
       \lambda w @ f[v:=w][\vec x\setminus x_i:=\vec e\setminus e_i], & v=x_i\\
       \lambda w @ f[v:=w][\vec x:=\vec e], & v\notin\vec x\\
     \end{array}
   \right. \mbox{new}~w
\end{eqnarray*}
This then leads to key theorems regarding free variables and substitution:
\begin{eqnarray}
   \fv(e[x:=f])
   &=&
   \fv(e)\setminus\setof x \cup (\fv(f) \cond{x \in \fv(e)} \emptyset)
\\ \fv(e[\vec x:=\vec e])
   &=&
   \fv(e)\setminus\vec x
   \cup
   \bigcup_i\setof{\fv(e_i) \cond{x_i \in \fv(e)} \emptyset}
\end{eqnarray}
The first is provable, with care, by induction on $e$
(even the base-case is non-trivial !).

We then extend our language to include explicit expression meta-variables ($E \in M$)
and explicit simultaneous substitutions:
\begin{eqnarray*}
   E \in M && \mbox{Given.}
\\ e,f \in L_1 &::\!\!+&  E | e[\vec f/\vec x]
\end{eqnarray*}
Here it is understood that vectors $\vec f$ and $\vec x$ are of the same length
and the latter has no duplicates.

We now extend the notion of free variables, but note that we cannot ``expand'' the
application of $\fv$ to $E$.
The effect of this is that we no longer can return a set of variables, but must instead
return an expression ($s \in S_1$) denoting such a set as a function of its explicit meta-variables:
\begin{eqnarray*}
  s \in S_1 &::=& \setof{\vec v} | E | s \setminus s | \bigcup \setof{\vec s} | v \in s \sthen s
\end{eqnarray*}
The last construction is a conditional: for $v \in s_0 \sthen s_1$,
if the condition holds, then it denotes $s_1$, otherwise it denotes the empty set.

There are obvious injections $L^{01}$ and $S^{01}$ that embed $e_0 : L_0$  and $s_0 : S_0$
into $L_1$ and $S_1$ respectively.

Given an environment ($\rho_M$) mapping $E$ to values from $L_0$,
we can then map both $L_1$ to $L_0$ and $S_1$ to $S_0$ as follows
(where we drop ${}_M$ in most cases as it is obvious from context):
\begin{eqnarray*}
   \rho_M &:& M \fun L_0
\\ L^{10} &:& (M \fun L_0) \fun  L_1 \fun L_0
\\ L^{10}_\rho(v) &\defs& v
\\ L^{10}_\rho(f~e) &\defs& (L^{10}_\rho(f))~(L^{10}_\rho(e))
\\ L^{10}_\rho(\lambda v @ e) &\defs& \lambda v @ (L^{10}_\rho(e))
\\ L^{10}_\rho(E) &\defs& \rho(E)
\\ L^{10}_\rho(e[\vec f/\vec x]) &\defs& (L^{10}_\rho(e))[\vec x:=\vec f]
\\
\\ S^{10} &:& (M \fun L_0) \fun S_1 \fun S_0
\\ S^{10}_\rho(\setof{\vec v}) &\defs& \setof{\vec v}
\\ S^{10}_\rho(E) &\defs& \fv_0(\rho(E))
\\ S^{10}_\rho(s_1 \setminus s_2) &\defs& S^{10}_\rho(s_1) \setminus S^{10}_\rho(s_2)
\\ S^{10}_\rho(\bigcup\setof{\vec s}) &\defs&  \bigcup\setof{\vec{S^{10}_\rho(s)}}
\\ S^{10}_\rho(v \in s_0 \sthen s_1)
    &\defs&
     S^{10}_\rho(s_1)
    \cond{v \in S^{10}_\rho(s_0)}
     \emptyset
\end{eqnarray*}

We can extend $\fv$ to $L_1$ as follows:
\begin{eqnarray*}
   \fv,\fv_1 &:& L_1 \fun S_1
\\ \fv(x) &\defs& \setof x
\\ \fv(e_1~e_2) &\defs& \bigcup( \fv e_1,\fv e_2 )
\\ \fv(\lambda x @ e) &\defs& (\fv e) \setminus \setof x
\\ \fv(E) &\defs& E
\\ \fv (e[\vec f/\vec x])
   &\defs&
   \fv(e)\setminus\setof{\vec x}
   \cup
   \bigcup_i \setof{ x_i \in \fv(e) \sthen \fv(f_i)}
\end{eqnarray*}
For a substitution, we see that the presence of the free variables
of a replacement expression ($e_i$) is contingent
on the presence of the corresponding \emph{target} variable ($x_i$)
in the free variables of the \emph{base} expression ($e$).
If the base expression is a meta-variable, then we get the following
instantiation of the last law:
\begin{eqnarray*}
   \fv(E)[e_1,\ldots,e_n/x_1,\ldots,x_n]
   &=& E\setminus\setof{x_1,\ldots,x_n}
       \union
       \bigcup \setof{  x_i \in  E @ \sthen e_i }
\end{eqnarray*}
We cannot either perform the set difference operation,
nor evaluate any of the conditionals.
In effect, in order to give an accurate description of the free variables
of this language we need to return a (variable-set valued) expression
that describes how the resulting set of variables is contingent
upon the (yet to be determined) free variables of the meta-variables.
This is the motivation for the $\sthen$ construct in $S_1$.



We now find that we can construct the following diagram relating the $L_i$ and $S_i$:
\begin{center}
 \begin{tikzcd}[row sep=4em,column sep=4em]
     L_1 \rar{\fv_1} \dar[bend left]{L^{10}_\rho}
 &   S_1 \dar[bend left]{S^{10}_\rho}
 \\
     L_0 \rar[swap]{\fv_0} \uar[hook,bend left]{L^{01}}
  &  S_0 \uar[hook,bend left]{S^{01}}
 \\
 \end{tikzcd}
\end{center}
From this we can immediately suggest a few lemmas/theorems:
\begin{eqnarray}
   L^{10}_\rho(L^{01}(e_0)) &=& e_0, \mbox{ for all }\rho
\\ S^{10}_\rho(S^{01}(s_0)) &=& s_0, \mbox{ for all }\rho
\\ \fv_1(L^{01}(e_0)) &=& S^{01}(\fv_0(e_0))
\\ S^{10}_\rho(\fv_1(e_1)) &=& \fv_0(L^{10}_\rho(e_1)), \mbox{ for all }\rho
\end{eqnarray}
Proofs of these are by induction over the leftmost $e_i$ in each case,
and are best done in the order given above.
The following is an easy consequence of the above:
\begin{eqnarray*}
   && S^{10}_\rho(\fv_1(L^{01}(e_0)))
\\ &=& S^{10}_\rho(S^{01}(\fv_0(e_0)))
\\ &=& \fv_0(e_0)
\end{eqnarray*}

At this point we need to extend the language further to have explicit
quantifier meta-variables that stand for lists of ordinary variables
or corresponding lists of expressions:
\begin{eqnarray*}
  \lst q,\lst r \in Q && \mbox{Given.}
\end{eqnarray*}
We then extend our language again (where $\lstvec q$, like $\vec x$, has no duplicates):
\begin{eqnarray*}
   e,f \in L_2 &::\!\!+& \lambda \vec v,\lstvec q @ e | e[\vec f,\lstvec r/\vec x,\lstvec q]
\end{eqnarray*}
As before, there is an obvious injection $L^{12} : L_1 \fun L_2$,
as well as $L^{02} : L_0 \fun L_2$,
and the syntactic sugar $\lambda \vec x @ e$  is now a proper part of $L_2$.
In effect the $L_2$ extensions subsume the lambda and substitution
constructs of $L_1$.

The $\lst q$ and $\lst r$ are intended to denote lists (possibly empty)
of ordinary variables ($\vec v$) and expressions ($\vec f$) respectively.
To this end we introduce two new environments:
\begin{eqnarray*}
   \rho_V &:& Q \fun V^*
\\ \rho_E &:& Q \fun L_1^*
\end{eqnarray*}
We can use these to define the conversion
$$L^{21}_{(\rho_V,\rho_E)} : L_2 \fun L_1,$$
and coupled with $\rho_M$, we can then get
$$L^{20}_{(\rho_V,\rho_E,\rho_M)} : L_2 \fun L_0.$$
In the sequel we shall often use $\rho$ to denote one of the above
when it is  obvious from context,
or to denote the entire tuple-parameter of a conversion: i.e. $L^{20}_\rho$.

We can now define the conversion:
\begin{eqnarray*}
   L^{21} &:& (Q \fun V^*) \times(Q \fun L_1^*) \fun  L_2 \fun L_1
\\ L^{21}_\rho(v) &\defs& v
\\ L^{21}_\rho(f~e) &\defs& (L^{21}_\rho(f))~(L^{21}_\rho(e))
\\ L^{21}_\rho(E) &\defs& E
\\ L^{21}_\rho(\lambda \vec v,\lstvec q @ e)
   &\defs&
   \lambda \vec v\cat\vec{\rho_V(\lst q)} @ (L^{21}_\rho(e))
\\ L^{21}_\rho(e[\vec f,\lstvec r/\vec x,\lstvec q])
   &\defs&
   (L^{21}_\rho(e))
     [ \vec{L^{21}(f)}\cat\vec{\rho_E(\lst r)}
       /
       \vec x \cat \vec{\rho_V(\lst q)}
     ]
\end{eqnarray*}
Inbuilt here, are assumptions about the lengths of various lists matching
up in $[\vec f,\lstvec r/\vec x,\lstvec q]$:
\begin{eqnarray*}
   \len\vec f &=& \len\vec x
\\ \len\lstvec r &=& \len\lstvec q
\\ \len(\rho_E(\lst r_i)) &=& \len(\rho_V(\lst q_i))
\end{eqnarray*}

We now need to extend the variable set language to cope with the $Q$ extensions
(again we have $S_2$ subsuming one $S_1$ component, and adding a new one:
\begin{eqnarray*}
   s \in S_2 &::\!\!+& \setof{\vec v,\lstvec q} | \lst q \in s \ssthen \lst r
\end{eqnarray*}
Here we introduce the $\ssthen$ symbol,
similar to $\sthen$,
but note that the contents
and meaning are different: $\lst q$ will denote a list of variables ($\vec v$),
and $\lst r$ will denote a list of sets ($\vec s$), of the same length.
The overall value will be merging all $s_i$ where $v_i$ is in $s$.
Again,the embedding $S^{12} : S_1 \fun S_2$ should be obvious.

Now, the opposite conversion:
\begin{eqnarray*}
   S^{21} &:& (Q \fun V^*) \times(Q \fun L_1^*) \fun S_2 \fun S_1
\\ S^{21}_\rho(\setof{\vec v,\lstvec q}) &\defs& \setof{\vec v\cat\vec{\rho_V(\lst q)}}
\\ S^{21}_\rho(E) &\defs& E
\\ S^{21}_\rho(s_1 \setminus s_2) &\defs& S^{21}_\rho(s_1) \setminus S^{21}_\rho(s_2)
\\ S^{21}_\rho(\bigcup\setof{\vec s}) &\defs&  \bigcup\setof{\vec{S^{21}_\rho(s)}}
\\ S^{21}_\rho(v \in s_0 \sthen s_1)
   &\defs&
   v \in S^{21}_\rho(s_0) \sthen S^{21}_\rho(s_1)
\\ S^{21}_\rho(\lst q \in s \ssthen \lst r)
    &\defs&
    \bigcup_i
    \setof{  \rho_V(\lst q)_i \in S^{21}_\rho(s) \sthen \fv_1(\rho_E(\lst r)_i) }
\end{eqnarray*}

We are now in a position to define free variables over $L_2$:
\begin{eqnarray*}
   \fv,\fv_2 &:& L_2 \fun S_2
\\ \fv(x) &\defs& \setof x
\\ \fv(e_1~e_2) &\defs& \bigcup( \fv e_1,\fv e_2 )
\\ \fv(\lambda \vec x,\lstvec q @ e) &\defs& (\fv e) \setminus \setof{\vec x,\lstvec q}
\\ \fv(E) &\defs& E
\\ \fv (e[\vec f,\lstvec r/\vec x,\lstvec q])
   &\defs&
   \fv(e)\setminus \setof{\vec x,\lstvec q}
\\ && {} \cup
   \bigcup_i
   \setof{
     x_i \in \fv(e) \sthen \fv(f_i)
   }
\\ && {} \cup
   \bigcup_j
   \setof{
     \lst q_j \in \fv(e) \ssthen \lst r_j
   }
\end{eqnarray*}

Once more, we have a commuting square:
\begin{center}
 \begin{tikzcd}[row sep=4em,column sep=4em]
     L_2 \rar{\fv_2} \dar[bend left]{L^{21}_\rho}
 &   S_2 \dar[bend left]{S^{21}_\rho}
 \\
     L_1 \rar[swap]{\fv_1} \uar[hook,bend left]{L^{12}}
  &  S_1 \uar[hook,bend left]{S^{12}}
 \\
 \end{tikzcd}
\end{center}
and corresponding theorems:
\begin{eqnarray}
   L^{21}_\rho(L^{12}(e_1)) &=& e_1, \mbox{ for all }\rho
\\ S^{21}_\rho(S^{12}(s_1)) &=& s_1, \mbox{ for all }\rho
\\ \fv_2(L^{12}(e_1)) &=& S^{12}(\fv_1(e_1))
\\ S^{21}_\rho(\fv_2(e_2)) &=& \fv_1(L^{21}_\rho(e_2)), \mbox{ for all }\rho
\end{eqnarray}

At this point we consider normal forms for free-variable
set expressions, which require three further
language extensions:
we extend membership to include conjunction,
and introduce a shorthand for the empty conjunction (true).
We also allow explicit subtractions on the rhs of $\ssthen$:
$$\begin{array}{rcll}
  m \in Member  & ::= & v \in s | \lst q \in s | \lst q \in \lst r & \mbox{Element Membership}
\\              &  |  & \bigwedge (m_1,\ldots,m_n) & \mbox{Conjunction}
\\ \top &\defs& \bigwedge()
\\ s \in S_3 &::\!\!+& m \ssthen \lst r \setminus \setof{\vec v,\lstvec q}
\end{array}$$
Note that we also allow a membership predicate of the form $\lst q \in s$
to appear in $\sthen$, with a different semantics to the same thing in $\ssthen$:
\begin{eqnarray*}
   S^{21}_\rho(\lst q \in s_1 \sthen s_2)
    &\defs&
    \bigcup_i
    \setof{  \rho_V(\lst q)_i \in S^{21}_\rho(s_1) \sthen s_2 }
\end{eqnarray*}
In addition we allow $\lst q \in \lst r$ to appear in $\ssthen$,
where we break the need for corresponding indices of $\lst q$ and $\lst r_1$ to match,
but retain it between $\lst q_i$ and $(\lst r_2)_i$:
\begin{eqnarray*}
   S^{21}_\rho(\lst q \in \lst r_1 \ssthen \lst r_2)
    &\defs&
    \bigcup_i
    \setof{  \rho_V(\lst q)_i \in\fv_1(\rho_E(\lst r_1)) \sthen \fv_1(\rho_E(\lst r_2)_i) }
\end{eqnarray*}

We also introduce
a form of conjunction for membership predicates in $\sthen$
and the notion of  a naked $Q$ variable $\lst r$ denoting a list of expressions:
\begin{eqnarray*}
   s \in S_3 &::\!\!+& m \sthen s |  \lst r
\end{eqnarray*}
Embedding $S^{23} : S_2 \fun S_3$ is obvious,
whilst the conversion in the opposite direction is less so:
\begin{eqnarray*}
   S^{32} &:& S_3 \fun S_2
\\ S^{32}(\bigwedge () \sthen s)
   &\defs&
   s
\\ S^{32}(\bigwedge (m_1,\ldots,m_n) \sthen s)
   &\defs&
   m_1 \sthen (m_2 \sthen \ldots (m_n \sthen s)\ldots)
\\ S^{32}(\lst r) &\defs& \lst q \in \setof \lst q \ssthen \lst r
\end{eqnarray*}

So, we can use $\fv_2$ to get the desired free-variable set for any instance of $L_2$,
as a member of $S_2$,
and then embed into $S_3$ for normalisation and reasoning.

\newpage
At this point it is worth writing out $S_3$ explicitly (as $S$),
and fine-tuning the syntax of membership:
\begin{eqnarray*}
   s \in S &::=& \setof{\vec v,\lstvec q}
\\ &|& E
\\ &|& s \setminus s
\\ &|&  \bigcup \setof{\vec s}
\\ &|&  m \sthen s
\\ &|&  \lst q \in s \ssthen \lst r \setminus \setof{\vec v,\lstvec q}
\\ &|&  \lst r
\\ m \in Mmbr  & ::= & v \in s | \lst q \in s  | \bigwedge (m_1,\ldots,m_n)
\end{eqnarray*}
It is worth stressing how the constructs $\lst r$, $\sthen$ and $\ssthen$ should be interpreted:
\begin{itemize}
  \item As a standalone,
        $\lst r$ denotes the union of the free-variables of the list of expressions given by $\rho_E$.
        $$\sem{\lst r} = \bigcup_i\setof{\fv(\rho_E(\lst r)_i)}$$
  \item
     The construct $(v|\lst q) \in s \sthen t$ equals $t$ if all of $v$ or $\lst q$ is contained in $s$.
     If either $s$ or $t$ is an instance of $\lst r$ then it is viewed as per the previous bullet-point.
     \begin{eqnarray*}
        \sem{v \in s \sthen t} &=& \sem{t} \cond{v \in \sem{s}} \emptyset
     \\\sem{\lst q \in s \sthen t} &=& \sem{t} \cond{\lst q \subseteq \sem{s}} \emptyset
     \end{eqnarray*}
  \item The construct $\lst q \in s \ssthen \lst r \setminus \setof{\vec v,\lstvec q}$ is well-formed only if $\len\rho_V(\lst q)=\len\rho_E(\lst r)$,
     and basically determines its result on a component-wise examination of the elements
     of $\lst q$.
     \begin{eqnarray*}
        \sem{\lst q \in s \ssthen \lst r\setminus \setof{\vec v,\lstvec q}}
        &=&
        \bigcup_{i=1}^{\len~\lst q}\setof{ \fv(\rho_E(\lst r)_i) \cond{\rho_V(\lst q)_i \in \sem{s}} \emptyset} \setminus \setof{\vec v,\lstvec q}
     \\ &=&
        \bigcup_{i=1}^{\len~\lst q}\setof{ \rho_V(\lst q)_i \in s \sthen \fv(\rho_E(\lst r)_i)} \setminus \setof{\vec v,\lstvec q}
     \end{eqnarray*}
     If $s$ is an instance of $\lst r$ then we use the interpretation in the first bullet point.
     The case $v \in s \ssthen \lst r$ is only valid if $\len\rho_E(\lst r)=1$.
\end{itemize}
In effect $\sthen$ is a global conditional, while $\ssthen$ works pointwise
on corresponding members of $\lst q$ and $\lst r$.

Note that if $\lst q$ and $\lst r$ have length one, then $\lst q \in s \sthen \lst r$ and $\lst q \in s \ssthen \lst r$
are the same.

\newpage
A conditional expression $m \sthen s$ is \emph{upfront}
if $s$ does not itself contain any conditionals,
noting that $m \ssthen \lst r$ are always up-front by construction.
A set-expression is \emph{atomic} if it has one of the
following forms ($a$ is $v$ or $\lst q$):
$$
  \setof{ a_1, \ldots, a_n}
  \qquad\qquad
  E \setminus \setof{ a_1 , \ldots, a_n }, ~  n \geq 0
  \qquad\qquad \lst r\setminus \setof{ a_1 , \ldots, a_n }, ~  n \geq 0
$$
We define our normal form to be a union of upfront conditional set-expressions
whose set components are atomic.
We convert to normal-form by repeatedly applying the following equivalences
left-to-right,
designed to bring union and conditional to the top
\begin{eqnarray*}
   m_1 \sthen (m_2 \sthen s) &=& \bigwedge(m_1,m_2) \sthen s
\\ (m \sthen s) \setminus \setof{a_i} &=& m \sthen (s \setminus \setof{a_i})
\\ m \sthen \bigcup( s_i ) &=& \bigcup( m \sthen s_i )
\\ m \sthen \setof{} &=& \setof{}
\\ \lst q \in (v \in s_0 \sthen s_1) \ssthen \lst r
   &=&
   v \in s_0 \sthen ( \lst q \in s_1 \ssthen \lst r)
\\ \lst q_1 \in (\lst q_2 \in s_2 \ssthen \lst r_2) \ssthen \lst r_1
   &=&
   \bigwedge(\lst q_1 \in \lst r_2,\lst q_2 \in s_2) \ssthen \lst r_1
\\ \bigwedge( m_i, a \in (m \sthen s_1) )\sthen s_2
   &=& \bigwedge( m_i, a \in s_1, m ) \sthen s_2
\\ \bigwedge( m_i, a \in \setof{a_i}, a \in \setof{a_j} )\sthen s_3
   &=& \bigwedge( m_i, a \in (\setof{a_i}\cap\setof{a_j}), m ) \sthen s_3
\\ \bigwedge( m_i, a \in \setof{} ) \sthen s &=& \setof{}
\\ a \in N \setminus \setof{a_i}
   &=& a \in N, \qquad a \notin \setof{a_i}
\\ &=& a \in \setof {}, \qquad a \in \setof{a_i}
\\ \bigwedge( m_i, a \in \bigcup( s_i ) )\sthen s
   &=& \bigcup( \bigwedge( m_i, a \in s_i ) \sthen s)
\\ \bigcup( s_i ) \setminus \setof{ a_i }
   &=&
   \bigcup( s_i \setminus \setof{ a_i} )
\\ (s \setminus \setof{ a_i} ) \setminus \setof{ a_j }
   &=&
   s \setminus \setof{ a_i, a_j }
\\ s &=& \bigcup( \top \sthen s )
\end{eqnarray*}
We assume that nested unions and conjunctions are flattened on-the-fly.




\newpage
\section{Axioms}

We take our inspiration from \cite{journals/logcom/Tourlakis01}:
\begin{quote}
\begin{description}
  \item[Ax1.] All propositional axioms from \cite{gries.93}.
  \item[Ax2.] $A \lor (\forall x @ B) \equiv (\forall x @ A \lor B), x \notin A$
  \item[Ax3.] $(\forall x @ A) \implies A[x:=t]$ ($t$ substitutable in $x$ (in $A$))
  \item[Ax4.] $x = x$
  \item[Ax5. (Liebniz)] $x = t \implies (A \equiv A[x:=t])$
     \\($t$ substitutable in $x$ (in $A$))
\end{description}
\end{quote}
The meaning of $A[x:=t]$ in Tourlakis is not given
explicitly, but is discussed, particularly with regard
to the  side-condition designed to avoid variable capture ($t$ substitutable in $x$ (in $A$)).
We shall define $A[x:=t]$
as meta-notation describing the substitution
of $t$ for all free $x$ in $A$, with $\alpha$-renaming being
used to avoid name-capture (so the side-conditions on Ax3 and 5 above can be dropped).
Our take on the propositional axioms is shown in Figure \ref{fig:UTP2:prop-axioms}.
\begin{figure}
\begin{center}
  \boxedm{$$
  \AXPROP
  $$}
\end{center}
  \caption{\UTP2 Propositional Axioms}
  \label{fig:UTP2:prop-axioms}
\end{figure}
The remaining axioms are shown in Figure \ref{fig:UTP2:non-prop-axioms}.
\begin{figure}
\begin{center}
\boxedm{$$
  \AXNONPROP
$$}
\end{center}
  \caption{\UTP2 Non-propositional Axioms}
  \label{fig:UTP2:non-prop-axioms}
\end{figure}
The substitution axioms (\LNAME{Ax-XXX-Subst}) are experimental.
The reflexivity axiom, and that for $\theta$ are in fact schemas, indexed by all possible
types (sorts), as we have a many-sorted logic.

Also worth noting are the $\beta$-reduction axioms, e.g.
$$\AXObetaRed~.$$
The lhs can match a lambda-expression with only one variable ($(\lambda x @ e)f$)
which results in the rhs being $(\lambda \qsep @ e)[f/x]$,
which is just $e[f/x]$.
So, to maintain consistency, we should be able to match a rhs
of the form $e[f/x]$ and then view it as a zero-argument lambda abstraction,
and succeed, returning the lhs as $(\lambda x @ e)f$.




\section{Inference}

From \cite{journals/logcom/Tourlakis01}:
\begin{description}
  \item[Inf1. (Substitution)]  (no capture)
  \item[Inf2. (Leibniz)] $\INFER{A\equiv B}{C[p:=A] \equiv C[p:=B]}$
  \item[Inf3. (Equanimity)] $\INFER{A, A \equiv B}{B}$
  \item[Inf4. (Transitivity)] $\INFER{A\equiv B, B \equiv C}{A \equiv C}$\\
\end{description}
We note that Inf4 is derivable from  Inf2 and Inf3, so we treat it
as derivable.
Our inference rules are shown in Figure \ref{fig:UTP2:inference-rules}.
\begin{figure}
\begin{center}
\boxedm{$$
  \INFERENCES
$$}
\end{center}
  \caption{\UTP2 Inference Rules}
  \label{fig:UTP2:inference-rules}
\end{figure}

\section{Proof/Theorems}

We adopt the definitions from \cite{journals/logcom/Tourlakis01}
with minor changes in notation:

The set of $\Gamma$-theorems, $\mathbf{Thm}_\Gamma$,
is the $\subseteq$-smallest subset of $Pred$ that satisfies:
\begin{description}
  \item[Th1]
     $\mathbf{Thm}_\Gamma$ contains as subset the closure under (Substitution)
     of all the propositional axioms (Fig. \ref{fig:UTP2:prop-axioms})
     and all the instances of the axiom schemata for the non-propositional part
     (Fig. \ref{fig:UTP2:non-prop-axioms})---the so-called logical axioms.
  \item[Th2] $\Gamma \subseteq \mathbf{Thm}_\Gamma$---the non-logical axioms.
  \item[Th3] $\mathbf{Thm}_\Gamma$ is closed under (Leibniz) and (Equanimity).
\end{description}
We write $p \in \mathbf{Thm}_\Gamma$ as $\Gamma \vdash p$,
and use $\vdash p$ to denote $\emptyset \vdash p$.

A finite sequence $p_1,\ldots,p_n$ of $Pred$
is a $\Gamma$-proof iff every $p_i$, for $i=1,\ldots,n$
is one of
\begin{description}
  \item[Pr1] A logical axiom
  \item[Pr2] A member of $\Gamma$
  \item[Pr3] The result of either (Leibniz) or (Equanimity)
   applied to some $p_j$, with $j < i$.
\end{description}


\section{Meta-Theorems}

We shall need to identify the derived rules that are embodied,
either in the matcher or in the provided proof strategies,
as these form part of the prover ``core''
(e.g. the deduction theorem and the \texttt{Assume} strategy,
or the all the derived variants of \LNAME{AX-$\lor$-$\forall$-scope},
which are implemented by \texttt{completeMatch}, in module \texttt{Manipulation}).
Some of these rules are presented in Figure \ref{fig:UTP2:derived-rules}.
\begin{figure}
\begin{center}
\boxedm{$$
  \DERIVED
$$}
\end{center}
  \caption{\UTP2 Derived Rules}
  \label{fig:UTP2:derived-rules}
\end{figure}
Post's Tautology Theorem
requires more background.
Of considerable importance is the Herbrand Deduction Theorem
and its variants, shown in Figure \ref{fig:UTP2:deduction-theorem}.
\begin{figure}
\begin{center}
\boxedm{$$
  \DEDUCTION
$$}
\end{center}
  \caption{\UTP2 Deduction Theorem}
  \label{fig:UTP2:deduction-theorem}
\end{figure}
The Flexible Deduction rule is implemented,
as suggested in \cite{journals/logcom/Tourlakis01},
not by doing the substitution, but by making $\vec x$
behave like constants, by marking them as ``known'',
to use the parlance in module \texttt{Matching}.


\section{Undefinedness}

We have yet to make a firm decision how to handle undefinedness,
apart from noting that the technical details of how this is
done have implications for the validity of the reflexivity of equals,
and of the Deduction Theorem, among others.

Type matching can only return type-bindings,
so it is simpler than the others.

For the formal presentation we assume the following mathematical
type syntax:
\begin{eqnarray*}
  \rho,\tau \in Type
    & ::= &  B \mbox{ --- Boolean}
  \\ &|& Z \mbox{ --- Integer}
  \\ &|& t \mbox{ --- Type variable}
  \\ &|& ? \mbox{ --- Arbitrary Type}
  \\ &|& P~\tau \mbox{ --- Set Type Constructor}
  \\ &|& \tau^* \mbox{ --- List Type Constructor}
  \\ &|& \tau \times \tau \mbox{ --- Product Type Constructor}
  \\ &|& \tau \fun \tau \mbox{ --- Function Type Constructor}
  \\ &|& !s \mbox{ --- Error Type}
\end{eqnarray*}

We can define inference rules for matching:
\input{doc/formal/Matching-Type-Rules}
The following rules are controversial
(a form of reverse matching):
$$\begin{array}{r@{\qquad}l}
   \MRTVarRN & \MRTVarR
\\ \MRTArbRN& \MRTArbR
\end{array}$$
The rules are intended to check law matches
for type-compatibility.
