When checking side-conditions we need to handle meta-variables properly, by returning a side-condition on those variables rather than just a true/false result. So, the result of the check
$$
  x \not\in  ( P \land y=z \lor Q )
$$
should be
$$
  x \not\in P \land x \notin Q
$$
and not False, or ``don't-know''.

We shall associate the following mathematical notation
with the \texttt{SideCond} datatype,
where $P$ and $E$ are meta-variables denoting arbitrary predicates and
expressions respectively,
and $V$ are explicitly enumerated sets of variables:

\begin{tabular}{|l|c|}
  \hline
  \texttt{SideCond} & Notation
\\\hline
  \texttt{SCtrue}  & $\TRUE$
\\\hline
  \texttt{SCisCond P}  & $\isCond P$
\\\hline
  \texttt{SCnotFreeIn PredM V P}  & $P \DISJ V$
\\\hline
  \texttt{SCnotFreeIn ExprM V E}  & $E \DISJ V$
\\\hline
  \texttt{SCareTheFreeOf PredM V P}  & $P = V$
\\\hline
  \texttt{SCcoverTheFreeOf PredM V P}  & $ P \subseteq V $
\\\hline
  \texttt{SCfresh PredM V}  & $\isFresh V$
\\\hline
  \texttt{SCAnd S}  & $\bigwedge S$
\\\hline & $ \FALSE  $
\\\hline
\end{tabular}

Notes:
\begin{enumerate}
  \item Here $P \DISJ V$ is shorthand for $P \cap V = \emptyset$.
  \item $\FALSE$ does not have an explicit \texttt{SideCond} representation
   but is useful in what follows.
   Given \texttt{Maybe SideCond} we can interpret \texttt{Nothing} as $\FALSE$.
  \item
    In the mathematical notation we may allow $P$ and $E$ to be instantiated
    as general predicates as well as just denoting meta-variables,
    in which case we use notation $p$ and $e$.
    \\However, in general we recursively descend through $e$ or $p$
    until we get conjunction of conditions on meta-variables $E$ and $P$.
  \item In the sequel we use $M$ to stand for either $P$ or $E$ when the distinction
    is immaterial.
  \item
    The variable-sets $V$ are always explicitly enumerated,
    so set-theoretic properties involving these can always be calculated.
\end{enumerate}

\paragraph{Freshness}
Freshness is a property of both predicate and expression meta-variables and
means they must be match variables.
The general rule is that the only occurrences of fresh variables should be those that
are explicitly mentioned in the law.

Consider a law fragment
$$L(x,P,Q,R), \isFresh x$$
matched against a goal
$$G(v,A,B,C)$$
with bindings
$$
 \beta = \mapof{x \mapsto v, P \mapsto A, Q \mapsto B, R \mapsto C}
$$
The side condition $\isFresh x$, translated to $\isFresh v$, is satisfied if the only occurrence of $v$ is in the binding $\mapof{x \mapsto v}$, and it does not occur anywhere in  $A$, $B$ or $C$. Putting it formally:
\begin{eqnarray*}
 \beta~\textbf{sat}~\isFresh x
 &\defs&
 \exists v @ v=\beta(x) \land v \notin \bigcup vars(rng(\beta\hide\setof x))
\end{eqnarray*}
A consequence of $\isFresh x$ for any expression variable $x$
is that it immediately implies $P \DISJ \setof x$
for all other predicate meta-variables $P$ in the law.
