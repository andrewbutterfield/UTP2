\subsection{Classical 2nd-Order Logic (Prawitz 1956)}

This is a summary of key results from \cite{Prawitz56}.
The notation has been re-worked, and key definitions
formalised in a more modern functional style.

\subsubsection{Syntax (I\S1)}

Symbol Categories:
\begin{eqnarray*}
   x,y,z &:& Var \mbox{ --- variables: only occur bound}
\\ a,b,c &:& Par \mbox{ --- parameters: only occur free}
\\ p^n,q^n,p,q &:& Pred.\nat \mbox{ --- Predicates with Arity}
\\ \setof{\F,\land,\lor,\implies,\forall,\exists} &=& LCon \mbox{ --- Logical Constants}
\\ J,K &:& Const \mbox{ --- Individual Constants}
\\ F,G,H &:& Opn.\nat \mbox{ --- Operation Constants with arity}
\\ P,Q,P^n,Q^n &:& PCon.\nat \mbox{ --- Predicate Constants with Arity}
\end{eqnarray*}

Terms:
\begin{eqnarray*}
    t,u \in Term &::=& a ~|~ K ~|~ F.n(t_1, \ldots, t_n)
\end{eqnarray*}

Formulae:
\begin{eqnarray*}
   \alpha \in AtomicF &::=& \F ~|~ p.n(t_1, \ldots t_n) ~|~ P.n(t_1, \ldots t_n)
\\ A,B,C \in Formula &::=& \alpha ~|~ A \land B ~|~ A \lor B ~|~ A \implies B
\\ && {} ~|~ \forall x @ A ~|~ \forall x @ A^x_a ~|~ \exists x @ A ~|~ \exists x @ A^x_a
\\ A^x_a \in Formula^x_a &\defs& \setof{ A[x/a] | A \in Formula }
\\ \Delta,\Gamma &\in& \power Formula
\end{eqnarray*}

Pseudo-Terms and Pseudo-Formulas:
\begin{eqnarray*}
    \pseudo t,\pseudo u \in pTerm &::=& x ~|~ t
\\ Term &\subset& pTerm
\\ \pseudo \alpha \in pAtomicF &::=& \F ~|~ p.n(\pseudo t_1, \ldots \pseudo t_n)
                                ~|~ P.n(\pseudo t_1, \ldots \pseudo t_n)
\\ \pseudo A \in pFormula &::=& \pseudo \alpha
                            ~|~ \pseudo A \land \pseudo B
                            ~|~ \pseudo A \lor \pseudo B
                            ~|~ \pseudo A \implies \pseudo B
\\                    && {} ~|~ \forall x @ \pseudo A
                            ~|~ \exists x @ \pseudo A
\\ \pseudo A^x_a \in pFormula^x_a &\defs& \setof{ \pseudo A[x/a] | \pseudo A \in pFormula }
\\ Formula &\subset& pFormula
\end{eqnarray*}

\newpage

Definitions (incomplete, but enough to get the drift \ldots):
\begin{eqnarray*}
   degree &:& pFormula \fun \nat
\\ degree(\pseudo A) &\defs& \mbox{no. of occurrences of logical constants (except \F)}
\\ principalSign &:& (pFormula \hide pAtomicF) \fun LCon
\\ principalSign(\pseudo A \land \pseudo B) &\defs& \land
\\ \ldots
\\ principalSign(\exists x @ \pseudo A) &\defs& \exists
\\ scope &:& LCon \times pFormula \pfun \power pFormula
\\ scope(\land,\pseudo A \land \pseudo B) &\defs& \setof{\pseudo A,\pseudo B}
\\ \ldots
\\ scope(\exists,\exists x @ \pseudo A) &\defs& \setof{\pseudo A}
\\ FV,BV  &:& pFormula \fun \power Var
\\ FV(Formula) &=& \setof\emptyset
\\ univClosure,existClosure &:& Formula \times \power Par \fun Formula
\\ univClosure(A,\setof{a_1,\ldots,a_n})
   &\defs&
   \forall x_1,\ldots,x_n @ A[a_1,\ldots,a_n/x_1,\ldots,x_n]
\\ isSentence &:& Formula \fun Bool
\\ isSentence(A) &\defs& A \mbox{ has no parameters}
\\ subFormula &:& Formula \rel Formula
\\ &\infer& subFormula(A,A)
\\ subFormula(B \land C,A) &\infer& subFormula(B,A), subFormula(C,A)
\\ subFormula(\forall x @ B,A) &\infer& subFormula(B[t/x],A)
\end{eqnarray*}
Measure $degree$ is clearly intended to provide a well-founded relation
on formula for use in inductive proofs.

Substitutions
\begin{eqnarray*}
   A[x/t] && \mbox{substitute $x$ for term $t$ in $A$}
\\ A[u/t] && \mbox{substitute term $u$ for term $t$ in $A$}
\\ \pre A[x/t]
  &\defs&
  \mbox{$t$ not in scope of $\forall x$ or $\exists x$ in A (capture avoidance)}
\\ \pseudo A[t/x] && \mbox{substitute term $t$ for free $x$ in $\pseudo A$}
\end{eqnarray*}

Defined Signs:
\begin{eqnarray*}
   \lnot A &\defs& A \implies \F
\\ A \equiv B &\defs& (A \implies B) \land (B \implies A)
\end{eqnarray*}

\newpage
\subsubsection{Deduction (I\S2)}

If we have deduction:
$$
\INFER{A_1, \ldots ,A_n}{B}
$$
we can write this as $\setof{A_1,\ldots,A_n} \infer B$,
or more simply as $A_1, \ldots , A_n \infer B$.

Four ways to discharge assumptions:
\begin{enumerate}
  \item Given $\Gamma,A \infer B$, deduce $A \implies B$, discharge  $A$
  \item Given $\Gamma,\lnot A \infer \F$, deduce $A$, discharge $\lnot A$
  \item Given $A \lor B$; $\Gamma_1,A \infer C$; and $\Gamma_2,B \infer C$,
        deduce $C$, discharge $A$ in 2nd deduction, and $B$ in 3rd.
  \item Given $\exists x @ A$ and $\Gamma,A[a/x] \infer B$,
        deduce $B$, discharge $A[a/x]$,
        provided $a$ not in $\exists x @ A$, $B$ or any assumption
        other than $A[a/x]$ on which $B$ depends in this deduction.
\end{enumerate}
In more modern notation:
\begin{enumerate}
  \item $$\INFER{\Gamma,A \infer B}{\Gamma \infer A \implies B}$$
  \item $$\INFER{\Gamma,\lnot A \infer \F}{\Gamma \infer A}$$
  \item $$\INFER{\Gamma_0 \infer A \lor B \qquad \Gamma_1,A \infer C \qquad \Gamma_2,B \infer C}{\Gamma_0,\Gamma_1,\Gamma_2 \infer C}$$
  \item $$\INFER{\Gamma_0 \infer \exists x @ A \qquad \Gamma,A[a/x] \infer B}{\Gamma_0,\Gamma \infer B},~a \mbox{ fresh}$$
\end{enumerate}

\newpage
Deduction laws, modern style (almost, let $\Gamma_i = \Gamma$ for all $i$ to get
the more typical presentation):
$$\begin{array}{rcl}
   \INFERN%
     {\Gamma_1 \infer A \qquad \Gamma_2 \infer B}%
     {\Gamma_1,\Gamma_2 \infer A \land B}%
     {\INTRO{$\land$}}
&& \INFERN%
     {\Gamma_0 \infer A \lor B \qquad \Gamma_1,A \infer C \qquad \Gamma_2,B \infer C}%
     {\Gamma_0,\Gamma_1,\Gamma_2 \infer C}%
     {\ELIM{$\lor$}}
\\
\\ \INFERN%
     {\Gamma \infer A \land B}%
     {\Gamma \infer A}%
     {\ELIM{l$\land$}}
&& \INFERN%
     {\Gamma \infer B}%
     {\Gamma \infer A \lor B}%
     {\INTRO{l$\lor$}}
\\
\\ \INFERN%
     {\Gamma \infer A \land B}%
     {\Gamma \infer B}%
     {\ELIM{r$\land$}}
&& \INFERN%
     {\Gamma \infer A}%
     {\Gamma \infer A \lor B}%
     {\INTRO{r$\lor$}}
\\
\\ \INFERN%
     {\Gamma,A \infer B}%
     {\Gamma \infer A \implies B}%
     {\INTRO{$\implies$}}
&& \INFERN%
     {\Gamma_1 \infer A \qquad \Gamma_2 \infer A \implies B}%
     {\Gamma_1,\Gamma_2 \infer B}%
     {\ELIM{$\implies$}}
\\
\\ \INFERN%
     {\Gamma \infer A}%
     {\Gamma \infer \forall x @ A[x/a]}%
     {\INTRO{$\forall$}},~a \not\in \Gamma
&& \INFERN%
     {\Gamma \infer \forall x @ A}%
     {\Gamma \infer A[t/x]}%
     {\ELIM{$\forall$}}
\\
\\ \INFERN%
     {\Gamma \infer A[t/x]}%
     {\Gamma \infer \exists x @ A}%
     {\INTRO{$\exists$}}
&& \INFERN%
     {\Gamma_0 \infer \exists x @ A \qquad \Gamma,A[a/x] \infer B}%
     {\Gamma_0,\Gamma \infer B}%
     {\ELIM{$\exists$}},
\\ && ~a \not\in \exists x@A,\Gamma
\\
\\ \INFERN%
     {\Gamma \infer \F}%
     {\Gamma \infer A}%
     {\LNAME{\F--i}}
&& \INFERN%
     {\Gamma,\lnot A \infer \F}%
     {\Gamma \infer A}%
     {\LNAME{\F--c}}
\\
\\ \multicolumn{3}{c}{\mbox{Gentzen's rules for negation}}
\\ \INFERN%
     {\Gamma,A \infer \F}%
     {\Gamma \infer \lnot A}%
     {\INTRO{$\lnot$}}
&& \INFERN%
     {\Gamma_1 \infer A \qquad \Gamma_2 \infer \lnot A}%
     {\Gamma_1,\Gamma_2 \infer \F}%
     {\ELIM{$\lnot$}}
\\ (\mbox{instance of }\INTRO{$\implies$})
&& (\mbox{instance of }\ELIM{$\implies$})
\end{array}$$
\begin{itemize}
  \item Minimal logic: drop \LNAME{\F--i},\LNAME{\F--c}.
  \item Intuitionistic logic: drop \LNAME{\F--c}.
  \item Classical logic: keep all.
\end{itemize}



\subsubsection{Propositional Connectives}


\subsubsection{Axioms}


\subsubsection{Inference Rules}


\subsubsection{Derived Rules}


\subsubsection{Valuations}


\subsubsection{Meta-theorems}

