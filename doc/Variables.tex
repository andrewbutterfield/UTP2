\section{Variables in \UTP2}

\subsection{Motivation}

We want a machine-readable notation that is lightweight,
with a minimal amount of declaration on the part of the user,
and a maximal degree of inference on the part of the parser,
and law matching facilities.


\subsection{Variable Classes}

We recognise four classes of variables in \UTP2:
\begin{description}
  \item[Standard] ($x,y,z$)
     denote values in the relevant semantic space,
     and include the \emph{observation} variables that form part of a UTP predicate's alphabet.
  \item[Expr-Schematic] ($E,F,G$)
    represent arbitrary expressions, particularly in general laws and definitions.
  \item[Pred-Schematic ($P,Q,R$)]
    represent arbitrary predicates,  particularly in general laws and definitions.
  \item[List] ($q,r$)
    denote lists of zero or more Standard and List variables,
    and are only valid were lists of variables appear,
    i.e., quantifier variable lists, explicit substitution target lists,
    and the special case of 2-place predicates.
    This category of variable differs considerably from the other three:
    \begin{itemize}
      \item
        There are no binding constructs for these---they generally occur in binding positions
        for other (currently just Standard) variable classes.
    \end{itemize}
\end{description}
The first three classes all have possible binders, and so the notion of
whether or not any given occurrence is free or bound makes sense,
as is the notion of sets of free or bound variables for each of the classes.
In general, the phrase ``free (bound) variables'' refers to ``free (bound) Standard variables",
as this is were most of the focus and complexity lies.
The key difference between Standard and Expr-Schematic variables
is what is returned as the free variable set of a single such variable:
\begin{eqnarray*}
   fv(x) &\defs& \setof x
\\ fv(E) &\defs& fv.E
\end{eqnarray*}
Here $fv.E$ is a token that denotes
the free variables of whatever expression syntax $E$ represents,
once that is determined.
In general, if a law has a side-condition of the form $a \in fv(b)$,
with $a$ and $b$ both variables,
we consider $b$ to be a schematic variable, and if $a$ occurs in a context where
it could either be a Standard or Expr-Schematic variable, we take it to be Standard.

\newpage
\subsection{Variable Matching}

The general principle behind matching is that a variable matches
any object of the right ``kind'', except:
\begin{enumerate}
  \item
    if the variable is bound, in which case it can only match another bound variable of the same ``kind''.
  \item
    if the variable is ``known'',
    in which case it only matches itself, or its ``definition'', if one exists.
\end{enumerate}
\begin{description}
  \item[``kind'']
    describes the sort of entity we are interested in: these are (Standard) variables (\textsf{V}),
    expressions (\textsf{E}), predicates (\textsf{P}) and  variable-lists (\textsf{V*}),
    noting that Standard and Expr-Schematic variables are considered instances of expressions,
    and that Pred-Schematic variables are instances of predicates.
  \item[``known'']
    variables are those that have been declared as having a special meaning.
    Examples include the observation variables of a UTP theory such as Reactive Processes (e.g. $ok$ and $ok'$),
    or the introduction of a named predicate from that theory, e.g. $J$.
    (e.g. $J \defs (ok\implies ok') \land wait'=wait \land tr'=tr \land ref'=ref$)
  \item[``definition'']
    Some known variables have definitions, which is an object of the same kind
    for which they may be substituted.
    An example is $J$ in the Reactive theory
    (e.g. $J \defs (ok\implies ok') \land wait'=wait \land tr'=tr \land ref'=ref$),
    and so will match only themselves or their definition.
    Other variables are known but have no such definition (e.g., $ok$) and so only match themselves.
\end{description}
If a binder binds an otherwise known variable,
then all instances in scope of that binding are bound, and not known.
So, in a context where $ok$ is known, then in the
pattern:
$$
ok \lor \exists ok @ ok \implies ok'
$$
the first occurrence of $ok$ ($ ok'$) can only match $ok$ ($ok'$), whilst the
second and third occurrences of $ok$ can match any bound variables, so the
above can match:
$$
ok \lor \exists a @ a \implies ok'
$$
In practise, we $\alpha$-rename laws/patterns so that bound variables
do not clash with known ones, or each other. making bindings much simpler.
To see why, consider matching the following, where $ok$ is known:
$$
ok \land (\exists ok @ \lnot ok) \land (\exists ok @ ok \lor ok')
$$
against:
$$
ok \land (\exists a @ \lnot a) \land (\exists b @ b \lor ok')
$$
We need three different bindings for $ok$ in different parts of the pattern
($ok \mapsto ok$, $ok \mapsto a$ and $ok \mapsto b$, respectively).

\newpage
\subsection{Variable Syntax}

All of the variables mentioned above fit the same general lexical pattern.

 Well-formed variables have the following syntax:
 \begin{eqnarray*}
    s \in Std && \mbox{``standard'' variables}
 \\ \ell \in Lst && \mbox{list variables}
 \\ g \in Gen &::=&  s | \ell
 \\ R \in Rsv &::=& O | M | S
 \\ d \in Dcr &::=& {}^\cdot | {}' | {}_x | {}^{``}
 \\ v \in Var &::=& \left( g | R [g,\ldots,g] \right)^d
 \end{eqnarray*}
THIS STRUCTURE NEEDS REWORKING! O,M and S are List Variables.
AND STANDARD VARIABLES HAVE OBJECT-LOGIC VARIABLES AND SCHEMATIC VARIABLES.


At present the various classes of variables belong to the following
syntactical groups:

\begin{tabular}{|l|c|}
  \hline
  Variable Class & Syntax Group
\\\hline
  \textbf{Standard} & $Var$
\\\hline
  \textbf{Expr-Schematic} & $Gen$
\\\hline
  \textbf{Pred-Schematic} & $Gen$
\\\hline
  \textbf{List} & $Lst$, $Rsv$
\\\hline
\end{tabular}

We note that the four classes described here have to a large extent been
superseded by the five syntactical groups ($Std$, $Lst$, $Gen$, $Rsv$, $Var$) as described above.
