\subsubsection{Preliminaries}

We want to assess $\alpha$-equivalence,
as the best way to complete a proof.
Given that we have three kinds of variables (Std, General-List, Reserved-List)
we require $\alpha$-equivalence to respect the kind distinction,
and in particular we expect the reserved list names and decorations to match,
except that subscripts may differ.

If the two predicates are $\alpha$-equivalent we should be able to determine
a bijective mapping from the \emph{bound} variables of the first predicate to those of the second.
We define $\alpha$-equivalence recursively,
yielding either failure, or a mapping between bound variables
on each side that induces equivalence.
This paramterised at every level by the current set of variable bindings
(empty at the top level).
\begin{eqnarray*}
   \_\alfeqv\_ &:& Pred^2 \fun \Bool
\\ P_1 \alfeqv P_2
   &\defs&
   (P_1 \balfeqv\emptyset\emptyset P_2) \neq  inl()
\\ \_ \balfeqv\_\_ \_ &:& (Pred\cross \power Var)^2 \fun 1 + BIJ
\end{eqnarray*}
Given that we have a number of contexts in which variable ordering is unimportant,
such as quantifier lists, our bijective mapping has two components:
an explicit bijection coupled with a pair of sets of variables of the same size,
representing the need for some bijection to exist between them.
\begin{eqnarray*}
  Bij,(\beta,L,R) \in  BIJ
  &=&  (Var \pbij Var) \cross \power Var \cross \power Var
\\ \Inv{BIJ}(\beta,L,R)
  &\defs&
  L \DISJ \dom~\beta
  \land
  R \DISJ \rng~\beta
\\ && {} \land
  \#(L|_{Std}) = \#(R|_{Std})
  \land
  \#(L|_{Lst}) = \#(R|_{Lst})
\end{eqnarray*}
Note that we need to track bound variables for each side
as we descend:
We define a bijection merge that fails if the two bijections are in conflict,
otherwise glues them together.
Here, and in the sequel, we only show cases that succeed---all other situations
result in failure.
\begin{eqnarray*}
 \_\uplus\_ &:& BIJ^2 \fun 1 + BIJ
\\ (\beta_1,L_1,R_1) \uplus (\beta_2,L_2,R_2)
   &\defs&
   (\beta_1\sqcup\beta_2,L,R),
\\ &&\qquad
     L \DISJ \dom~(\beta_1\sqcup\beta_2)
     \land
     R \DISJ \rng~(\beta_1\sqcup\beta_2)
\\ && \WHERE
\\ && \quad L = L_1\setminus(\dom~\beta_2)
                \cup
                L_2\setminus(\dom~\beta_1)
\\ && \quad R = R_1\setminus(\rng~\beta_2)
                \cup
                R_2\setminus(\rng~\beta_1)
\\ && \textbf{provided}
\\ && \quad  \forall \maplet u v \in \beta_1
          @ (u \in L_2 \land v \in R_2) \lor (u \notin L_2 \land v \notin R_2))
\\ && \quad \forall \maplet u v \in \beta_2
          @ (u \in L_1 \land v \in R_1) \lor (u \notin L_1 \land v \notin R_1))
\end{eqnarray*}

We define $\sqcup$,
as a finite-map lifting of the following pointwise definition.
\begin{eqnarray*}
   \_\sqcup\_ &:& (A \pfun B)^2 \fun 1 + (A \pfun B)
\\ \mapof{u \to v_1} \sqcup \mapof{u \to v_2}
   &\defs& \mapof{u \to v_1}, \qquad v_1 = v_2
\end{eqnarray*}
