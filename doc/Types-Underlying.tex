Given that our expression language does not contain boolean connectives
(represented solely at the predicate level),
then the only way an expression $e$ can have type $Bool$ are enumerated as follows:
\begin{enumerate}
  \item a boolean constant or variable ($b:\Bool$);
  \item the equality of two expressions of the same type ($e_1,e_2:T$)
  \item an application of a function of type $T \fun \Bool$,
        where $T$ is a non-boolean type.
        \\(One exception: the unary logical negation operator
           during parsing will occur as a top-level function name
           of type $\Bool\fun\Bool$).
\end{enumerate}
In the first case the underlying type is $\Bool$,
in the second case it is $T \times T$, and in the third it is $T$.
If $\Gamma \infer e \utype T$ means that $e$ has underlying type $T$
in context $\Gamma$, then the following rules allow this to be inferred:
\begin{eqnarray*}
   \INFER%
     {\Gamma \infer v :: \Bool}
     {\Gamma \infer v \utype \Bool} && [UVar]
\\ \INFER%
     {\Gamma \infer k :: \Bool}
     {\Gamma \infer k \utype \Bool} && [UConst]
\\ \INFER%
     {\Gamma \infer f~a :: \Bool \qquad \Gamma \infer a :: T }
     {\Gamma \infer f~a \utype T} && [UApp_1]
\\ \INFER%
     {\Gamma \infer f~a :: \Bool \qquad \Gamma \infer f :: T \fun \Bool}
     {\Gamma \infer f~a \utype T} && [UApp_2]
\end{eqnarray*}
Top-level conditional expressions should be lifted to
a conditional predicate.
