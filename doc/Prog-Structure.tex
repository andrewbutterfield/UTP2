\section{Overview}

The sources  for \UTP2 comprise a very large collection of \LaTeX\ sources, and Haskell Literate Scripts,
which are themselves also valid \LaTeX.

The directory/folder structure has been designed to support ease of
development for the main development environment, namely WinEdt/MikTeX/ghci
on Windows (currently Windows 8).

\section{Directory Structure}

The top-level contains all the \LaTeX\ master documents,
currently including:
\begin{description}
  \item[UTP2-MAIN.tex] \UTP2\ sources and documentation
  \item[UTP2-Hacking-MAIN.tex] This document
  \item[UTP2-Reference-MAIN.tex] Reference manual
  \item[UTP2-User-Manual-MAIN.tex] User Guide.
\end{description}
Almost all other files present at this level should be considered
as junk, even if tracked by Mercurial. This will be tidied up at some
future date.

\newpage
Subdirectories are organised as follows:
\begin{description}
  \item[src]~\\
     Haskell source files, as well as MS-DOS batch files (.bat)
     for building under Windows.
  \item[doc]~\\
     Mainly \LaTeX\ files giving documentation of various forms,
     as well as text files to do with installation.
     It has a couple of sub-directories to manage certain types of documentation:
     \begin{description}
       \item[formal] mainly formal definitions of aspects of the logic
       \item[images] images (obviously!)
       \item[papers] sources for conference/journal papers about \UTP2
       \item[styles] \LaTeX\ style files (currently ignored).
       \item[screenshots] screen shots of the tool in action,
           arranged in topic sub-directories
     \end{description}
  \item[batch]~\\
     Created by someone from a unix background, to hold .bat files.
     \\(Deprecated, unused, will probably disappear).
  \item[licence]~Licensing files.
  \item[orphans]~unwanted and unloved --- also likely to vanish.
  \item[resource]~mainly sound and help files.
    \\The help file (with the long unpronounceable name)
     has been subsumed into the relevant code,
     and is no longer required.
  \item[test]~\\
     test stuff, currently unused,
     but we will probably flesh this out at some stage.
  \item[thlib]~\\
     This is where we build \UTP2\ theories to drive and test the development,
     and most of which will become part of a standard theory library release.
  \item[www]~\\
     Stuff for the (release) website.
\end{description}
\section{Literate File Structure}

All the Haskell source files are literate scripts
(.lhs extension) that are themselves valid \LaTeX\ files,
in which the Haskell source is enclosed in
\verb"\begin{code}" \ldots \verb"\end{code}" environments.
The \texttt{code} environment is defined in the style file \texttt{doc/saoithin.sty}.

We do not use ``bird-tracks'' or \texttt{lhs2tex},
nor do we use Hackage/Haddock in any way.

\newpage
\begin{haskell}
An example of some .lhs source is below:
\subsection{\UTP2 Source Example}

\begin{code}
module Example where
import Utilities
\end{code}

\subsubsection{Intro}

We can have a suitably mathematical comment:
$\sigma \circ \sigma = \mathsf{id}$
and then some code:
\begin{code}
sigma = negate
\end{code}
\end{haskell}
When typeset, this results in:

\subsection{\UTP2 Source Example}

\begin{code}
module Example where
import Utilities
\end{code}

\subsubsection{Intro}

We can have a suitably mathematical comment:
$\sigma \circ \sigma = \mathsf{id}$
and then some code:
\begin{code}
sigma = negate
\end{code}

\section{\UTP2\ Distribution Structure}

At present Unix and Mac OS X users have to build from source,
and at present we do not have proper makefiles or install scripts.

For windows users we package up a binary release.

Below are listing of all the relevant installation text files.

\newpage
\subsection{README.txt}
\verbatiminput{doc/README.txt}

\newpage
\subsection{COPYING.txt}
\verbatiminput{licence/COPYING.txt}

\newpage
\subsection{INSTALL.txt}
\verbatiminput{doc/INSTALL.txt}

\newpage
\subsection{MANIFEST.txt}
\verbatiminput{doc/MANIFEST.txt}
