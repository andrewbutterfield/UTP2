\textbf{Resolving via Side-Conditions} \\
We have tables of the form:
\begin{eqnarray*}
  x_i &\mapsto& \setof{a_1,\ldots,a_n}
\\ \lst x_j &\mapsto& \setof{a_1,\ldots,a_n,\lst a_1,\ldots,\lst a_m}
\\ \lst (\lst x_j,\lst e_j) &\mapsto&
        \setof{(a_{k+1},f_{k+1}),\ldots,(a_n,f_n),
               (\lst a_1,\lst f_1),\ldots,(\lst a_m,\lst f_m)}
\end{eqnarray*}
We then look at each pattern side-condition: $V \mathcal R M$,
where $\mathcal R \in \setof{=,\supseteq,\DISJ}$,
and replace standard $M$ by the free variables of its translation into goal-space,
giving a mapping from pattern variable sets
to a relation with a free-variable expression:
\begin{eqnarray*}
  V \mapsto (\mathcal R,F)
\end{eqnarray*}
We do not handle list meta-variables ($\lst M$) at present.

In general $F$ is a union of an enumeration, a list-variable less an enumeration
and either of the above conditional on some set membership.
Given such a mapping,
and from our tables a mapping of the form
\begin{eqnarray*}
 x  &\mapsto& \setof{a_1,\ldots,a_n,\lst a_1,\ldots,\lst a_m} = A, \quad x \in V
\end{eqnarray*}
then we adjust the range set $A$ as follows:
\begin{eqnarray*}
   (\DISJ,F)     && A \setminus F
\\ (=,F)         && A \cap F
\\ (\subseteq,F) && A
\end{eqnarray*}
We only consider enumerations when doing this adjustment,
as everything else is too complicated!
