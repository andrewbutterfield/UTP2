We want to assess $\alpha$-equivalence,
as the best way to complete a proof.
If the two predicates are $\alpha$-equivalent it returns
a bijective mapping from the first predicate to the second.
We define $\alpha$-equivalence recursively,
yielding either failure, or a mapping between bound variables
on each side that induces equivalence.
first noting that we need to track bound variables for each side
as we descend:
\begin{eqnarray*}
   \_\alfeqv\_ &:& Pred^2 \fun 1 + (Var \pfun Var)
\\ P_1 \alfeqv P_2
   &\defs&
   P_1 \balfeqv\emptyset\emptyset P_2
\end{eqnarray*}
We define a mapping merge that fails if the two maps are in conflict,
otherwise glues them together,
and a map restriction that removes any maplet
if neither of its components are in the specified set
(both can be specified pointwise):
\begin{eqnarray*}
 \_\uplus\_ &:& (Var \pfun Var)^2 \fun 1 + (Var \pfun Var)
\\ \mapof{u \to v_1} \uplus \mapof{u \to v_2}
   &\defs& \mapof{u \to v_1}, \qquad v_1 = v_2
\\ \_|\_ &:& (Var \pfun Var) \times \power Var \fun (Var \pfun Var)
\\ \mapof{u \to v} | B
   &\defs&
   \mapof{} \cond{\setof{u,v} \DISJ B} \mapof{u \to v}
\end{eqnarray*}
In the sequel we only show cases that succeed - all other situations
result in failure.
For composites, we recurse down,
and then merge the resulting bindings,
which may fail:
\begin{eqnarray*}
   (P_1 \land Q_1) \balfeqv{B_1}{B_2} (P_2 \land Q_2)
   &\defs&
   (P_1 \balfeqv{B_1}{B_2} P_2)
   \uplus
   (Q_1 \balfeqv{B_1}{B_2} Q_2)
\end{eqnarray*}
Any (predicate/expression) meta-variables must match themselves,
with an exception for regular variables
which can match expression meta variables of the same name
provided they are not known or bound
\begin{eqnarray*}
   M \balfeqv{B_1}{B_2} M &\defs& \mapof{}
\\ M \balfeqv{B_1 \setminus x}{B_2\setminus \setof x} x &\defs& \mapof{}, \quad \IF~x=M
\end{eqnarray*}
When we descend through a binder
we extend the bound variable sets.
We then take the returned mappings,
remove any entries not mentioned in either of the original bound sets,
and merge those.
\begin{eqnarray*}
   (\exists x_1 @ P_1) \balfeqv{B_1}{B_2} (\exists x_2 @ P_2)
   &\defs&
   (P_1 \balfeqv{B_1 \cup \setof{x_1}}{B_2\cup\setof{x_2}} P_2)
   |_{(B_1 \cup B_2)}
\end{eqnarray*}
Variables are equivalent if identical and neither is bound,
or they are both in their respective bound sets,
in which case they can be mapped:
\begin{eqnarray*}
   x_1 \balfeqv{B_1 \cup \setof{x_1}}{B_2\cup\setof{x_2}} x_2
   &\defs& \mapof{x_1 \to x_2}
\\ x \balfeqv{B_1 \setminus \setof{x}}{B_2 \setminus\setof{x}} x
   &\defs& \mapof{}
\end{eqnarray*}
For substitutions (which can be viewed as abstractions applied
to arguments) we simply view the target variables
as bound in the substitution body, obtaining:
\begin{eqnarray*}
   P_1[e_1/x_1] \balfeqv{B_1}{B_2} P_2[e_2/x_2]
   &\defs&
   (P_1 \balfeqv{B_1 \cup \setof{x_1}}{B_2\cup\setof{x_2}} P_2)
   |_{(B_1 \cup B_2)}
\\ && \uplus (e_1 \balfeqv{B_1}{B_2}  e_2)
\end{eqnarray*}
A complication arises when we have simultaneous substitutions,
as the ordering of the replacement expressions should be immaterial.
We use the bindings returned from the substitution body
to help find an appropriate permutation.
A consequence of this is that our algorithm is currently incomplete
in that it may classify substitutions as incomplete when in fact they
are.
Note that we expect a one-to-one mapping for all QVars,
even multiple q-variables.