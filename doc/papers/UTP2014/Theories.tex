\section{\UTP2 Theories}\label{sec:theories}


A \UTP2 theory,
can be considered as a slice through the more conventional notion
of a ``classical'' UTP theory, it being
a coherent collection of the following items:
an alphabet defining the observations that can be made;
a set of healthiness conditions that characterise predicates that describe
realistic/feasible systems;
a signature that defines the abstract syntax of the language being defined;
definitions of the language constructs as healthy predicates;
and
laws that relate the behaviours of the various language components.

In \UTP2 we use the term ``\texttt{Theory}'' to refer to
such collections and coherent subsets,
along with various other pieces of ancillary information. For our purposes we can consider
a theory as being \emph{represented} by the following structures:

\begin{eqnarray*}
   Theory &=& \textbf{record}
%\\ && laws : Name \ffun Law
\\ && type : Name \ffun Type
%\\ && lang : Name \ffun LangSpec
\\ && preds : Name \ffun Pred
\\ && exprs : Name \ffun Expr
\\ && obs : Name \ffun Type \times OClass
\\ && \ldots \mbox{plus other stuff not relevant here}
\\ && \textbf{end}
\end{eqnarray*}
As notation $A \ffun B$ denotes a partial finite function from $A$
to $B$, it is effectively a finite table using a key of type $A$ to lookup
a value of type $B$, if present.

We now describe the purpose of each of the above tables:
\begin{description}
  \item[$type$]
    maps names (standard variables) to types, names in effect being those that
    denote constant values, such as arithmetic operators (+),
    or defined functions.
    This table only gives typing information:
    any name here will also need some other definition,
    either through the $exprs$ table (typically a named constant or shorthand),
    or via a definitional axiom for that name (typically a function/operator definition).
    The main role is when matching expressions,
    in order to prevent lots of spurious matches, particularly against
    the righthand sides of laws like $a+0=a$, $s \cat \nil = s$.
    This table provides information for
    an on-the-fly Hindley-Damas-Milner type inference engine\cite{DAMAS82},
    which annotates sub-expressions in both target and pattern predicates
    with their types, ``under the hood'',
    for use by the matching engine.
  \item[$pred$]
    maps names (predicate metavariables) to predicates, in effect allowing us to use these names
    as shorthands for often complex predicates.
    This is a common pattern of working in UTP, with such as examples as $J$
    (theory of designs), or $\Skip$ and $B$, (reactive theory).
  \item[$exprs$]
     is like $pred$ but for expressions.
  \item[$obs$]
    maps names (standard variables) that denote the observations/alphabet of the theory
    to their type, and indication of their observation class: model or script.
    To allow for the full generality of UTP,
    which does support non-homogeneous relations,
    it is necessary to explicitly list out both dashed and un-dashed variants
    were both occur. In effect this is an extension of the $type$ table,
    specialised for supporting the alphabet role of UTP.
\end{description}
In matching,
in general a pattern variable of a given kind (standard, meta-variable)
matches any target object if the same kind (value, expression or predicate),
so acting like a schematic variable.
But some variables are ``known'', like $ok$, $B$ and $J$,
and effectively either denote themselves, or an equivalent expansion.
The matcher needs to know what names are known in order to ensure
that they only match themselves, or their expansions.
In terms of the matcher algorithm, this turns out to be a minor
complication, requiring table information to be available,
and just needing a membership check in the above tables
when trying to match individual variables.

The type system is very simple, with basic types, composites,
and type-variables, to support polymorphism.
\begin{eqnarray*}
   t \in Type &::=& \Bool | \Int | t \fun t | \power t |  \tau | \ldots
\end{eqnarray*}

\section{Proof Contexts}\label{sec:contexts}

As an example, consider Figure \ref{fig:hier-of-theory}.
\begin{figure}[h]
  \centering
    \begin{tikzpicture}[>=stealth,->,shorten >=2pt,looseness=.5,auto]
     \matrix[matrix of nodes,
             nodes={rectangle,draw,minimum size=4mm},
             column sep={1cm,between origins},
             row sep={1cm,between origins}]
      {              & |(X)|\texttt{Circus}  \\
        |(W)|\texttt{While} &              & |(C)|\texttt{CSP}                 \\
                     & |(D)|\texttt{Design} &              & |(R)|\texttt{Reactive}  \\
                     &              & |(u)|\texttt{UTP}              \\
                     &              & |(b)|\texttt{Base}                   \\
                     &              & |(r)|\texttt{\_ROOT}                 \\
      };
      \draw(X) -- (W) ; \draw (X) -- (C) ;
      \draw(W) -- (D) ; \draw (C) -- (D) ; \draw (C) -- (R) ;
      \draw (D) -- (u) ; \draw (R) -- (u) ;
      \draw (u) -- (b) ;
      \draw (b) -- (r) ;
    \end{tikzpicture}
  \caption{A Hierarchy of \texttt{Theory}s}
  \label{fig:hier-of-theory}
\end{figure}
Here we see theory slices organised as an acyclic directed graph,
where each slice inherits material from those below it.
At the bottom we have the \texttt{\_ROOT Theory} (slice),
which is hardwired in%
\footnote{
\texttt{\_ROOT} is the only thing hardwired---all other slices and their hierarchy
can be custom-built to suit the user.
}%
, and simply contains just the axioms of the underlying logic.

When we start a proof of a conjecture,
we do so from a given theory slice.
Imagine as an example, we state the following conjecture as part of theory \texttt{While}
(a simple imperative language):
\[
  \LNAME{$:=$-self-seq}
  \quad
  ( x:= e \comp x := f) =  x := f[e/x]
\]
We then start operating in a \emph{Proof Context} that consists
of the sequence of theory slices from \texttt{While} down to\texttt{ \_ROOT}.
This defines all the information about laws, definitions, know observation variables
that are in scope at the point were the conjecture is defined.
A successful proof of such a conjecture will result in it being added
to the collection of laws associated with its theory,
namely \texttt{While} in this example.

Another important principle is that when matching part of the proof of
{\LNAME{$:=$-self-seq}}
against a law in the \texttt{UTP} theory (say), the proof context is
shrunk to that from \texttt{UTP} down.
This is to ensure that specific uses of a name in a higher theory
(e.g. $J$ in Design, as per {\LNAME{$J$-def}})
does not mask a possible general use of $J$ as a schematic variable
in a lower theory%
\footnote{Variable $J$ may not be a plausible general predicate variable,
but $B$ is, and yet it is given a specific definition in \texttt{Reactive}.}%
.
