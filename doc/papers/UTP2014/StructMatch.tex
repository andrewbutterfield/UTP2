\section{Structural Matching}\label{sec:struct:matching}

A representative collection of match inference rules are:

\begin{eqnarray*}
   \MRKnownVN    && \MRKnownV
\\ \MRUnknownVN  && \MRUnknownV
\\ \MROrN        && \MROr
\\ \MRAllN       && \MRAll
\\ \MRSubstN     && \MRSubst
\end{eqnarray*}
The notation $\beta_1 \uplus \beta_2$ denotes the merging of
two bindings, which can fail if they contradict one another.
The notation $\beta_1 \cong \beta_2$ means that $\beta_1 \uplus \beta_2$
will not fail.

The first rule covers the situation, in expression matching, where we are matching
a known standard  variable against itself.
The second rule shows that an unknown standard variable matches any expression.
In fact there are a number of detailed rules to do with matching of decorations
(dashes and subscripts) but we don't consider those here.
The third rule is a typical composite structural rule:
match the corresponding subparts, and then merge the bindings.
The fourth rule deals with quantifier matching, and introduces the notion
of matching two lists of variables
($u_1,\ldots,u_p \matches v_1,\ldots,v_t \withbind \beta_2$).
This is where the difficulties arise.
The last rule is for substitution, and again we find ourselves matching lists,
in this case containing expression/variable pairs:
($(e_1,u_1),\ldots,(e_p,u_p) \matches (f_1,v_1),\ldots,(f_t,v_t) \withbind \beta_1$).

We also note the following rule,
that covers the empty quantifier list case:
\[
\MRNullAllN \qquad \MRNullAll
\]

The rules \MRAllN\ and \MRSubstN\ require us to match lists of things
against lists of things,
were some of those things may themselves be list-variables
that themselves denote lists of things
(``thing'' here is either a variable or an expression).
The ordering in any of these lists is not important,
so the consequence is that our matching has become non-deterministic.
The challenge is to search all the possible matches to find at least
one that is compatible with other bindings,
and constraints introduced by the side-conditions.


The structural matching phase handles simple instances of this problem itself,
such as matching a singleton pattern variable list $\lst x$ against any target
variable-list $v_1,\ldots,v_t$.
Here the only possible outcome is to succeed with binding $\lst x \mapsto v_1,\ldots,v_t$.
Another common simple case is pattern $x,\lst y$ against target $u,\lst v$.
Standard pattern variables can only match standard target variables,
so the only possible match and binding results in
$ x \mapsto u, \lst y \mapsto \lst v$.

If the list match problem is not one of a few easy simple cases as just outlined,
then we have to \emph{defer} the match: in effect we return
it as a pattern-list,target-list pair $(\seqof{u_1,\ldots,u_p},\seqof{v_1,\ldots,v_t})$.
The type-signature of structural matching differs from $match$
in that is returns bindings, and sets of deferred match pairs:
\begin{eqnarray*}
   structM &:& Context \fun Pred \fun Pred \fun (1 + (Bind \times Deferred))
\\ Deferred &=& QDeferred \times SDeferred
\\ QDeferred &=& (QList \times QList)
\\ QList &=& Var^*
\\ SDeferred &=& (SList \times SList)
\\ SList &=& (Expr \times Var)^*
\end{eqnarray*}
A structural match is complete if the deferred lists are all empty,
otherwise the deferred match completion phase has to be invoked.
