We want to provide for the substitution
for free variables,
whilst ensuring that free-variable capture
does not occur.
We will ensure that we are only using well-formed substitutions
that have been normalised, so the only occurrence of list-variables
are in substitution pairs of the form
$$
L\!\!\dagger\!\setminus S\dagger \quad/\quad L\!\!\ddagger\!\setminus S\ddagger
$$
where all variables in $S$, if any, are known observation variables
(\emph{I'm not sure that this is a consequence of normalisation as described earlier}).

The scheme we follow is based on the following definition
of free-variable substitution for the $\lambda$-calculus,
(extended to (partially) support list-variables):
\begin{eqnarray*}
   v[r/x] &\defs& r \cond{x=v} v
\\ v\!\ddagger[L\!\dagger\! / L\ddagger]
   &\defs& v\!\dagger \cond{v \in \sem{L}} v\!\ddagger
\\ M\!\ddagger[L\!\dagger\! / L\ddagger]
   &\defs& L\!\dagger \cond{M=L} M\!\ddagger
\\ (e_1~e_2)[r/x] &\defs& (e_1[r/x])~(e_2[r/x])
\\ (\lambda b \cdot e)[r/x] &\defs& (\lambda b \cdot e)
\\&& \cond{b=x}
\\&& \left(
       \begin{array}{c}
         \lambda n \cdot ((e[n/b])[r/x]) \\
         \cond{b \mbox{ free in } r} \\
         \lambda b \cdot (e[r/x]) \\
       \end{array}
     \right)
     , \qquad n \mbox{ fresh}
\end{eqnarray*}
We check for $b$'s freeness in $r$ to avoid generating fresh
variables when not necessary.
So, for substitution in the presence of a binder, we can identify
three cases, which we list out explicitly, with labels:
$$\begin{array}{lll}
  (\lambda b \cdot e)[r/x]
\\ (\textsc{bnd}) & b=x & \lambda b \cdot e
\\ (\textsc{fre}) & b \neq x \land b \mbox{ not free in }r & \lambda b \cdot e[r/x]
\\ (\textsc{cap}) & b \neq x \land b \mbox{ free in }r & \lambda n \cdot (e[n/b])[r/x], \quad n \mbox{ fresh.}
\end{array}$$
We now express how this generalises to multiple bindings, meta-variables
and substitutions:
\begin{eqnarray*}
  (\lambda b_1,\ldots,b_m @ e)[e_1,\ldots,e_n/x_1,\ldots,x_n] && m,n \geq 0
\\  (\textsc{bnd})
    &\downarrow&
    \setof{x'_1,\ldots,x'_k}  = \setof{x_1,\ldots,x_n}\setminus\setof{b_1,\ldots,b_m}, 1 \leq k \leq n
\\ (\lambda b_1,\ldots,b_m @ e)[e'_1,\ldots,e'_k/x'_1,\ldots,x'_k] &&
\\ &\downarrow& \mbox{w.l.o.g.}, 1 \leq j \leq m
\\ (\textsc{fre})
   &\downarrow&
   \setof{b_1,\ldots,b_j} = \setof{b_1,\ldots,b_m}\setminus \mathcal{FV}\setof{e'_1,\ldots,e'_k}
\\ (\textsc{cap})
   &\downarrow&
   \setof{b_{j+1},\ldots,b_m} = \setof{b_1,\ldots,b_m}\cap \mathcal{FV}\setof{e'_1,\ldots,e'_k}
\\ (\lambda b_1,\ldots,b_j,\nu_{j+1},\ldots,\nu_m @ \qquad
   && \nu_{j+1},\ldots,\nu_m \mbox{ fresh w.r.t. } b_i, e, e'_i, x'_i
\\  e[\nu_{j+1},\ldots,\nu_m/b_{j+1},\ldots,b_m])
\\  {}[e'_1,\ldots,e'_k/x'_1,\ldots,x'_k]
\end{eqnarray*}
Any occurrence of $(L\setminus S)$ in a substitution can
be converted into a known set of variables.
If such an occurrence in a binder cannot be given a full denotation
(because $S$ contains `pattern' variables),
then we cannot perform the substitution at this point.

We can also give it an algorithmic formulation as follows:
\begin{eqnarray*}
  (\lambda B @ e) \sigma
  &\defs&
     {}\LET \sigma' = \sigma \setminus B \IN {}
\\&& {}\LET F = \fv(\ran~\sigma') \IN{}
\\&& {}\LET (B_F,B_C) = (B \setminus F, B \cap F) \IN {}
\\&& {}\LET \alpha :\in N_C \bij B_C \IN \lambda B_F \cup N_C @ (e~\alpha)\sigma'
\\&& {}\WHERE N_C \cap (B \cup F \cup \fv~e) = \emptyset
\end{eqnarray*}
The above description ignores variable ordering
in the binders, which of course has to be preserved.

This leads to a 2-phase algorithm:
\begin{description}
  \item[phase 1] compute $\sigma'$ and $F$ from $\sigma$ and $B$
   (determine substitution scope).
  \item[phase 2] With $F$ extended with $\fv~e$, work through $B$,
    classifying into $B_F$ (leave as is) or $B_C$ (replace with $N_C$),
   accumulating $\alpha$ as required
   (to avoid variable capture).
\end{description}