\section{Conclusions}\label{sec:conclusions}


We have looked at how commonplace modes of reasoning in UTP
have lead to the notions of having an object logic
with explicit meta (schematic) variables, list-variables
and explicit substition.
The equational reasoning style is supported by a matcher
that has to cope with the complexities that so arise,
the most important being the need to match modulo any permutations
in the ordering of variable lists.
This has lead to the notion of matching being separated into an initial
structural matching, deferring hard matching problems,
followed by a phase that resolves the non-deterministic choices
presented by the deferred material.
We have describes some of the heurisitics used to try to get
a usefully complete pattern matching facility.


In effect \UTP2 has turned into a tool for discovering the nature of the
underlying logic in common use for UTP.
By trying to mechanise what is a fairly standard approach in the literature
to writing and conducting UTP proofs,
we have exposed a range of subtleties, largely around all the implicit
information that is carried along in such proofs.


\subsection{Future Work}

Perhaps the most important thing that needs to be done
is to get the tool to a state where the wider community can experiment with
it and get feedback regarding its soundness and useability.
It would be hoped that many UTP theories from the literature
would get encoded for \UTP2, and that there could be a repository of these
made available, similar to Isabelle/HOL's Archive of Formal Proofs (AFP).

If a general non-deterministic matching framework emerges,
with side-conditions and prior bindings as constraints,
we would hope to see the matcher becoming much more simple.
This would assist in a longer-term goal,
to identify a minimal kernel logic, that might facilitate and LCF-style
approach to giving a sound safe kernel, so proofs can be trusted.

Other possibilities include links to other tools:
perhaps match completion could be handled by an SMT-solver (modulo set-theory)?
A facility to expert the logic, theories and proofs to a system like
Isabelle/HOL would increase the security and utility of this work.
