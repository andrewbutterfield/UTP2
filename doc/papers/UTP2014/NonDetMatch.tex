\section{Deferred Match Completion}\label{sec:nondet:matching}

Deferred match completion takes the context,
bindings and deferred matches from structural matching,
and attempts to find a single binding that covers all the deferred
matches, and is compatible with the structural binding.
It returns the merge of both bindings as the final result.
In addition to be compatible with the structural binding,
we must also ensure that any side-conditions are also obeyed,
so these are passed in
(We use typewrite font for the $Deferred$ arguments
as the correspond to actual Haskell code,
so we are flagging that we are close to the actual implementation here):
\begin{eqnarray*}
   && completeM : Context \fun  (Side\times Side) \fun Bind \fun Deferred \fun (1 + Bind)
\\ && completeM_{\Gamma}~(sc_T,sc_P)~\beta~(\texttt{qvms},\texttt{subms}) \defs \ldots
\end{eqnarray*}

Note that the name is a bit of a misnomer, as $completeM$ tries
to find a suitable binding, but is not guaranteed to succeed,
even if such a binding exists.

This algorithm is basically a series of heuristic steps
(9 in all)
that transform process the deferred data,
always looking for the earliest opportunity to fail,
or succeed.


\subsection{The search for completeness}


We shall present some of these heuristic steps here,
focussing on the deferred quantifier variable lists
(the expr/variable lists deferred from substitution matching
 follow along similar lines, with some extra complications
 that space does not permit us to discuss).
Also we have viewed the binding $\beta$ as monolithic,
but in fact it is composed of a number of individual binding tables.
Standard variables and meta-variables are bound to variables,expressions
and predicates as appropriate,
while list-variables, both generic and reserves are bound to lists of the
appropriate tables.
For this exposition we shall only consider three such tables:
\begin{eqnarray*}
   \texttt{ebind} &:& (Std)Var \ffun Expr
\\ \texttt{qsbind} &:& (Std)Var \ffun Var
\\ \texttt{qmbind} &:& (Lst)Var \ffun Var^*
\end{eqnarray*}

So our call looks like:
\begin{eqnarray*}
\lefteqn{completeM_{\Gamma}~(sc_T,sc_P)~(\texttt{ebind},\texttt{qsbind},\texttt{qmbind}\ldots)~(\texttt{qvms},\texttt{subms})}
\\ ~&\defs&  \mbox{see sketch below}\ldots
\end{eqnarray*}

\begin{enumerate}

\item

\textbf{Rule out Expression Matches}\\
 There may already be matches in \texttt{ebind} from pattern standard variables
 to target variables --- these should be removed from \texttt{qvms}
 and added into \texttt{qsbind}.
 This can fail if there is any conflict with pre-existing
 \texttt{qsbind} entries,
 and so serves as an early way to weed out matches that must eventually fail.

\item

\textbf{Identify Feasible Quantifier-List Matches}\\
  Argument \texttt{qvms} is a list of pairs, each pair of the form:
  \begin{eqnarray*}
   (~x_1,\ldots,x_k, \lst x_1,\ldots,\lst x_\ell
   &,&
     a_1,\ldots,a_n, \lst x_1,\ldots,\lst x_m~),
   \\ \mbox{pattern}  &,& \mbox{target}
  \end{eqnarray*}
  were we have ordered each list standard-variable first.

  From these we extract two tables, one for the standard pattern
 variables ($x_i$),
 the other for the pattern list-variables ($\lst x_j$)
  showing the \emph{possible} binding targets:
  \begin{eqnarray*}
    x_i &\mapsto& \setof{a_1,\ldots,a_n}
  \\ \lst x_j &\mapsto&
    \left\{
      \begin{array}{rc}
        \setof{\lst a_1,\ldots,\lst a_m}, & n=k \\
        \setof{a_1,\ldots,a_n, \lst a_1,\ldots,\lst a_m}, & n > k
      \end{array}
    \right.
  \end{eqnarray*}
  Entries for $x$ or $\lst x$
  from different \texttt{qvms}-pairs
  are merged by taking intersections of the relevant sets.

\item

\textbf{Prune Single Matches}\\
  \label{enum:prune}
  For every match of the form $x \mapsto \setof{a_1,\ldots,a_p}$
  in the standard table:
  \begin{enumerate}
    \item If the set is empty then FAIL
%    \item If $x$ is known, and $x \in \setof{a_1,\ldots,a_p}$
%       then record $x$ as targetted
%       and remove the mapping, else FAIL.
    \item if $x \notin \texttt{qbind}$ do nothing.
    \item if $\texttt{qbind}(x) = a$ then, if $a \in \setof{a_1,\ldots,a_2}$
     then record $a$ as targetted, and remove the mapping,
     otherwise FAIL.
  \end{enumerate}
  All targeted $a$s are removed
  from the range sets of both tables
  If any range set in the standard becomes empty, we FAIL.
  At the completion of this stage, without failing,
  then the standard table consists of pattern variables
  that can be still be bound to more than one target single-variable.

\item

\textbf{Resolving via Side-Conditions} \\
We have tables of the form:
\begin{eqnarray*}
  x_i &\mapsto& \setof{a_1,\ldots,a_n}
\\ \lst x_j &\mapsto& \setof{a_1,\ldots,a_n,\lst a_1,\ldots,\lst a_m}
\end{eqnarray*}
We then look at each pattern side-condition: $V~\mathcal R~\fv P$,
where $V=\setof{v_1,\ldots,v_n}$,
$\mathcal R \in \setof{=,\supseteq,\DISJ}$,
and $P$ is a standard pattern predicate meta-variable.
We use the bindings so far to translate $P$ into the corresponding
target predicate.
We then determine the free-variables, giving a set $F$
We then consider the relation $\mathcal R$  between pattern variables $V$
and target free variables $F$, and use it to adjust the standard table,
from which we get mappings of the form
\begin{eqnarray*}
 x  &\mapsto&  A, \mbox{ where } x \in V
\end{eqnarray*}
then we adjust the range set $A$ as follows:
\begin{eqnarray*}
   \mbox{if }\mathcal R = (\DISJ)      &,& A := A \setminus F
\\ \mbox{if }\mathcal R = (=)          &,& A := A \cap F
\\ \mbox{if }\mathcal R = (\subseteq)) &,& A \mbox{ unchanged}
\end{eqnarray*}
Again, we fail for the same circumstances that arise when pruning single matches.

\item

\textbf{Force Matches}
At this stage we have used pruning and side-conditions
to simplify as far as is practicable.
We are now going to resolve any remaining non-determinism
by basically forcing a final binding,
and hoping for the best%
\footnote{This is where incompleteness sneaks in.}%
!
We now invert both tables to get ones of the form:
 \begin{eqnarray*}
   a_i &\mapsto& \setof{x_{i1},\ldots,x_{in_i}, \lst x_{i1},\ldots,\lst x_{im_i}}
 \\ \lst a_j &\mapsto& \setof{ \lst x_{j1},\ldots,\lst x_{jn_j}}
 \end{eqnarray*}

We then process the single inverted table to construct a mapping,
that in some sense shows how ``popular'' a target standard variable
is---more popular means there are more pattern list-variables
that can include it in their binding list:
\begin{eqnarray*}
  a_i &\mapsto& n_i \qquad \mbox{number of $\lst x$}
\end{eqnarray*}
We use this mapping to sort the standard variable table range elements in ascending order.
We then do a form of greedy process to assign standard target variables
to pattern list variables, so we want to get the least popular done first.
We process each entry in the single table,
binding $x_i$ to the first entry ($a_{i1}$, greedy!) and then removing that entry
from the ranges of both tables:

For list-matching, simply working down the lists, binding each list variable
to its full range and then removing that from the rest
gives a bias towards alphabetically earlier list-variables.
Unfortunately this greedy approach can often fail
\\---e.g. trying to match lhs of axiom \AXorAllOScopeN:
$$\AXorAllOScope.$$
Ideally we should find choices that maximise the chance of side conditions
being satisfied.

We have developed a resolution mechanism
that is parameterised by an resolution strategy.
The idea is that we have a mapping of keys to lists of data items,
where a data-item can be associated with multiple keys.
We want to transform this into a mapping were each data item
appears exactly once.
Three strategies have been implemented,
as way to parameterise this forcing phase:
\begin{description}
  \item[All to First Key] very greedy,
  the first pattern list-variable is bound to everything,
   and all the rest are bound to the empty-list.
  \item[One Each, Rest to Last]
    each pattern list-variable is bound to one target variable,
    except the last which gets all that is leftover, if any.
  \item[Round-Robin]
    keep adding in one target variable to each pattern-list variable
    in a cyclic fashion until all are accounted for.
\end{description}
Currently the round-robin strategy is hardwired in,
but there are plans to allow the user change this.

\item

\textbf{Translate Law Side-Condition (In-Context) and Check}\\ % outer list
At this point we now have finalised bindings,
so what remains is to double-check that side conditions
are satisfied.
The bindings are used to transform pattern side-conditions
into corresponding target ones,
and a side-condition inference engine is used
to check that the target side condition
entails the translated pattern side-conditions.

\end{enumerate} % outer list


\subsection{Completion: a summary}

Match completion is a non-trivial activity,
and is by far the most complex part of \UTP2,
both in terms of code size and conceptual difficulty.
Experimentation with proofs shows that in fact most matching
rarely needs to go beyond the pruning step (\ref{enum:prune}).
Most matchings either succeed or fail at this point.

This code has been largely developed by attempting to build
a collection of theories from the ground up: propositional
theorems, quantifier theorems, laws of equality, arithmetic,
sets and lists, all followed by a generic base UTP theory
with standard notions like non-deterministic choice, refinement
and sequential composition.
Each attempt to move up the theory ``tower'' has lead to further extensions
of match completion, leading to the many phases we have just discussed,
and more.
What is becoming clear, however
(apart from the growing unmanageability)
is that there is a much simpler non-deterministic matcher,
with set-theoretic constraints that underlies all of this.
It is hoped that the nature of this will soon emerge.
