\section{Logic}\label{sec:logic}

The logic of \UTP2\ is an adaptation of the first-order equational logic
described by Tourlakis \cite{journals/logcom/Tourlakis01},
that fully formalises the logic of Dijkstra, Gries and Schneider \cite{gries.93}.
For this paper, we simply present the syntax and axioms of
a small first-order subset of
the whole logic,
sufficient to motivate and explain the required behaviour of the matcher.

We define our logic syntax
over a collection of given sets characterising different name-spaces:
\RLEQNS{
   x,y,z \in Var && (given) & \mbox{Variables}
\\ k \in Const && (given) & \mbox{Constants}
\\ f,g,h \in Name && (given) & \mbox{(Function) Names}
\\ E,F,G \in EName && (given) & \mbox{Expression Metavariable Names}
\\ P,Q,R \in PName && (given) & \mbox{Predicate Metavariable Names}
}
Constants and function names are as one would expect
in a logic with associated equational theories,
but we also have explicit meta-variables%
\footnote{Cover the role of ``schematic'' variables in some other logic presentations}
 for expressions
and predicates, in the object logic, as many UTP laws
are expressed using such.
Variables need special treatment, discussed further in Sec. \ref{sec:variables}.

Expressions denote values in the ``world of discourse'' (observations)
and are implicitly typed:
\RLEQNS{
   b,e \in Expr &::=& k | x  & \mbox{Expressions}
\\              &|& f~e & \mbox{Applications}
\\              &|& E & \mbox{Explicit Metavariable}
\\              &|& e[ e/x ] & \mbox{Explicit Obs. Substitution}
}
Expressions whose type is boolean ($b \in Expr$)
form the class
of \emph{atomic predicates}.
Predicates are defined much as expected:
\RLEQNS{
   p,q,r \in Pred &::=& \False & \mbox{Constant Predicates}
\\              &|& b & \mbox{Atomic Predicate}
\\              &|& p~ \lor q & \mbox{Disjunction}
\\              &|& P & \mbox{Explicit Metavariable}
\\              &|& \forall x,\ldots,x @ p & \mbox{Universal Quantifier}
\\              &|& p[ e,\ldots,e/x,\ldots,x ] & \mbox{Explicit Obs. Substitution}
}
One point to note is that the syntax includes explicit substitutions,
as these feature prominently in many UTP theories
(e.g. laws regarding assignment and sequential composition,
of the reactive healthiness condition \RR2).

We define a Law to be a pair,
consisting of a predicate and an associated side-condition ($Side$).
Side-conditions are a conjunction of zero or more basic conditions,
which typically capture relationships between given variables and
the free variables ($\fv$) of given predicates or expressions.
\begin{eqnarray*}
   sc \in Side &::=&  x \notin \fv.P
         | \setof{x,y,\ldots} = \fv.P
         | \setof{x,y,\ldots} \supseteq \fv.P
\\ Law &=& Pred \times Side
\end{eqnarray*}
