\section{Designs}\label{sec:designs}

\PLAN{A theory of designs is developed, showing implicit and explicit alphabets,
and the importance of side conditions.}

The UTP theory of Designs \cite[Chp 3]{UTP-book}
introduces two boolean observation variables ($ok, ok'$)
to model program start and termination, and new notation $P \design Q$
to represent a predicate with pre and post-conditions:
\begin{eqnarray*}
  ok,ok' &:& \Bool \\
  P \design Q &\defs& ok \land P \implies ok' \land Q
  , \qquad ok,ok' \notin \fv.P \cup \fv.Q
\end{eqnarray*}
A key feature to note is that in this theory we do not specify
the entire alphabet, but only stipulate that whatever it is,
it must contain $ok$ and $ok'$.
In this light we see an even stronger need for special list-variables
like $Obs$ as already introduced.

We can already capture this with our theories as described so far:
\begin{eqnarray*}
   obs(ok) &=& \Bool
\\ obs(ok') &=& \Bool
\\ lang(\design) &=& P \design P
\\ prec(\design) &=& (n,NonAssoc), \quad n \mbox{ is desired precedence}
\\ laws(\design\!-\!DEF)
   &=& (P \design Q \equiv ok \land P \implies ok' \land Q,
        ok,ok' \notin \fv.P \cup \fv.Q)
\end{eqnarray*}
Here we see some side-conditions that assert that neither $P$ nor $Q$ should
mention either $ok$ or $ok'$.
These are important side-conditions, without which we do not obtain the
desired behaviours (algebraic laws) for designs.
However, in proving properties of designs in UTP,
we find that the side-conditions play a more active role than encountered
in more traditional presentations of logic.
In many logics, side-conditions about free variables
are syntactic in nature and can always be checked/discharged
when applying a rule to a predicate in the logic.
In particular, when applying a rule like the one above, both $P$ and $Q$
will have been instantiated to concrete predicates, and so it will be
easy to establish the truthfulness of these side-conditions.
However in a UTP proof about the properties of designs,
we work with explicit meta-variables $P$ and $Q$
for which it is not possible to compute side-condition rules
at rule-application time.

Instead, we have to add a post processing stage to law matching.
Assuming that a target predicate match involving a law has succeeded
returning a binding,
We use that binding to translate any side-condition with the law
to a corresponding one in the target world.
We then need to show that the translated law-side condition
is a consequence of any side-conditions associated with the conjecture goal.

In effect, in addition to a syntax for side-conditions,
we have to implement a side-condition inference engine
that can deduce when one side-condition implies another.
Let $psc$ denote the translated pattern side-condition,
and $tsc$ denote the side-condition associated with the conjecture being proven.
We have to demonstrate that $tsc \implies psc$.
As side-conditions are a conjunction of a few basic primitive side-conditions,
we simply take both $tsc$ and $tsc \land psc$,
reduce both to a canonical normal form, and check for equality.


To illustrate all of this, here is a proof that $R \design S \equiv R \design R \land S$,
given that $ok, ok' \notin \fv.R \cup \fv.S$.
Here we deliberately state our conjecture using different meta-variables
to those used to define designs, to show the translation aspect at work.
Our proof strategy will be to take the lefthand side
and transform it into the righthand side%
\footnote{
The strategy in play is noted in the $Proof$ record.
}%
.

The first step  proceeds when a match of $R \design S$ succeeds
against pattern $P \design Q$ returning the binding $\mapof{P \mapsto R, Q \mapsto S}$.
However, we need to discharge the side-condition $ok,ok' \notin \fv.P \cup \fv.Q$.
We use the bindings to translate this to $ok,ok' \notin \fv.R \cup \fv.S$.
This then has to be implied by our conjecture side-condition, which in this case
is identical to the law condition, so
we can deduce that it holds.
The proof then proceeds as follows:
\begin{eqnarray*}
  && R \design S
\EQV{as just discussed above}
\\&& ok \land R \implies ok' \land S
\\&\quad\equiv\quad& ok \implies ( R \implies ok' \land S)
\\&\equiv& ok \implies ( R \implies R \land ok' \land S)
\\&\equiv& ok \land R \implies ok' \land S \land R
\EQV{see below}
\\ && R \design R \land S
\end{eqnarray*}
The last step up is similar to the first, as the matching of righthand sides
succeeds, and the bindings and translation are the same.
This raises a new and important issue to do with observational variables.
The variables $ok$ and $ok'$ mentioned above are not arbitrary,
but denote specific observations, and so it is important for UTP that
they only match themselves in laws, unlike general variables
that can match arbitrary expressions (including other variables).
This leads to the need to indicate that certain variables in patterns
stand for themselves. Such variables are described as being ``known''.
All $obs$ variables are known,
and there is also a facility for a user to give names to constants and expressions,
and so those names would also be considered ``known''.
We will not give further details here.

The structural matching rule for variable patterns needs to be modified,
using the context $\Gamma$ to check if a variable is known,
here written as $x \in \Gamma$:
\begin{eqnarray*}
  & \inferrule%
       {x \in \Gamma}%
       {\Gamma \vdash x \ddagger x }
\\& \inferrule%
      {x \notin \Gamma}%
      {\Gamma \vdash x \ddagger v |\setof{ x \mapsto v} }
\\& \inferrule%
      {x \notin \Gamma}%
      {\Gamma \vdash x \ddagger e |\setof{ x \mapsto e} }
\end{eqnarray*}
Note that when a known variable matches against itself, no binding entry
is produced.

At this point, given the hierarchy of Figure \ref{fig:hier-of-theory},
we have a theory called \texttt{Design},
which has access to the laws of logic, equality, arithmetic and sets,
as well as the definitions and associated laws of $\comp$, $\sqcap$, $\refinedby$,
$\cond c$ and $\design$, as well as the known observation variables $ok$ and $ok'$.
In particular, we stress that by being linked in the hierarchy shown,
the \texttt{Design} theory inherits all the material defined in \texttt{UTP}, and all its
ancestors.
This is quite abstract at this point, so now we move to ground it all a little more.

\subsection{Healthiness Conditions}

A key feature of UTP is the use of healthiness conditions,
expressed typically as monotonic idempotent predicate transformers.
To support this in \UTP2\, we need to extend the predicate syntax
to include notation for functions over predicates, and the application
of those to predicates, and appropriate axiomatisation:
\begin{eqnarray*}
   p,q,r \in Pred &::=& \ldots
\\              &|& \Lambda P @ p, \qquad  \mbox{Predicate Abstraction}
\\              &|& p(q), \qquad \mbox{Predicate Application}
\\ (\Lambda P @ p)(r) &\equiv& p[r/P]
\end{eqnarray*}
It is at this point that we definitely leave 1st-order logic behind
and move up towards 2nd- and higher-orders of logic.
At this point it is useful to have a facility to give names
to frequently used constructs like  healthiness conditions
or common predicate fragments, such as the predicates called $II$, $B$ and $J$
used in the definition of the Reactive theory \cite[Chp. 8]{UTP-book}.
In effect we want to give definitions like the following
(not necessarily from the theory of Designs):
\begin{eqnarray*}
   \H1 &\defs& \Lambda P @ ok \implies P
\\ J &\defs& (ok \implies ok') \land wait=wait \land tr'=tr \land ref'=ref
\end{eqnarray*}
We achieve this by adding in tables into a theory that allow us
to write such definitions,
and modifying the matching algorithm to treat all names in those
tables as ``known'':
\begin{eqnarray*}
Theory &=& \textbf{record} \ldots
\\ && preds : Name \pfun Pred
\\ && exprs : Name \pfun Expr
\\ && \ldots \textbf{end}
\end{eqnarray*}
So, for example,
in this theory of Designs we have $preds(\H1) = \Lambda P @ ok \implies P$.
The rest of the \UTP2\ machinery can then be used to reason about and use these
healthiness conditions in the normal way,
so for example, $\H1(q)$ can be converted into $ok \implies q$, and vice-versa.
