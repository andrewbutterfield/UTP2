\section{Matching}\label{sec:matching}

The matching algorithm implements a mechanism for implementing
judgements of the form:
\[
\Gamma\vdash P \ddagger T | \beta
\]
which can be read as saying:
\begin{quote}
  ``
  Target $T$ is an instance of pattern $P$, given matching context $\Gamma$,
  as witnessed by binding $\beta$.
  ''
\end{quote}
In effect,
this is implemented as a function that accepts $\Gamma$, $T$ and $P$
as inputs, along with side-conditions
(the target side-conditions are those of the conjecture being proved)
and either returns a failure indicator, or success along with a resulting
binding:
\begin{eqnarray*}
  && match : Context \fun (Pred\times Side) \fun (Pred\times Side) \fun (1 + Bind)
\\ && match_{\Gamma}~(T,sc_T)~(P,sc_P) \defs \ldots
\end{eqnarray*}
The context $\Gamma$ is essentially a stack of theories
going downwards from where pattern $P$ is defined,
refactored in ways that need not concern us here.
Typically we match $T$ against a long list of $P$s,
and each match is done by functional programming pattern-matching
on the structure or each $P$, so it is most pragmatic for it to be the last argument
of $match$.

The following principles are key:
\begin{description}
  \item[Fail Early]
    We typically match one $T$ against many $P$, so the quicker we
    can detect failure and move on, the better.
  \item[Bindings for All]
    Known variables only match themselves, e.g. $ok$ in the Design theory
    but we still require an explicit binding to be returned.
    This is to facilitate building the replacement,
    which uses the lack of bindings for any given variable to
    detect that that variable did not appear on the matched side of an equation
    but is only present on the replacement side.
    Consider using law $P \lor (P \land Q) \equiv P$
    as an expansion (right-to-left), with target $T$.
    It matches the rhs with binding $P \mapsto T$,
    but we have no binding from that match for $Q$.
    in fact, $Q$ can be any predicate, and so we need a way to provide one.
\end{description}
The Bindings for All principle leads us to highlight
the overall workflow in matching and apply a law in a proof:
\begin{enumerate}
  \item Match target $T$ against a list of $P$s.
  \item Select successful match of interest, with binding $\beta$
    and replacement $R$
  \item If some variables are in $R$ and not in $\beta$
      then prompt the user to supply suitable instances.
  \item Replace target $T$ by result of using binding $\beta$
   as a substitution for the variables in $R$
\end{enumerate}

We find that matching has to be split into two phases:
initial structural matching, followed by a ``completion'' phase.

In Section \ref{sec:struct:matching}
we describe the structural matching, and why it has to be incomplete.
Then, in Section \ref{sec:nondet:matching}
 we then talk about how to automatically fill in the gaps.
