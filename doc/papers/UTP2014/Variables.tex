\section{Variables in \UTP2}\label{sec:variables}

The notion of ``variable'' in \UTP2 (and by extension UTP)
is not simple, as exemplified by the following examples,
from the book\cite{UTP-book}:
\[\begin{array}{rcl@{\quad}r}
   P(\nu');Q(\nu) &~~=_{df}~~& \exists \nu_0 @ P(\nu_0)\land Q(\nu_0),
   out\alpha P = in\alpha' Q = \{\nu\}
   & [p49]
\\ \Skip_A &=_{df}& (\nu'=\nu), A = \{\nu,\nu'\}
   & [p50]
\\ x:=e &=_{df}& (true \design x'=e \land y'=y \land \ldots \land z'=z )
   &[p78]
\end{array}\]
and the following from the UTP semantics of \Circus\cite{journals/fac/OliveiraCW09}:
\[\begin{array}{rcl@{\quad}r}
   A_1 ; A_2 &~\defs~& \exists x_0 @ A_1[x_0/x'] \land A_2[x_0/x]
   & [p9]
\\ J &\defs& (okay\implies okay') \land tr'=tr \land wait'=wait
\\ && {} \land ref'=ref \land \nu'=\nu
   & [p9]
\end{array}\]
Here we have variables like $\nu$, $x$ that actually represent lists of variables,
usually those in the alphabet, or subscripted variants thereof.
We also have predicates like $\nu'=\nu$
that are a shorthand for loosely specified conjunctions of individual
variables, eg, $y'=y \land \ldots \land z'=z$.
We also note that despite their importance to UTP,
alphabets are often treated in proofs and derivations as implicit and understood.
A key design goal of \UTP2 is to support these aspects of UTP reasoning,
in do far as is practicable.
In effect this paper is a description of the impact of that decision
on the underlying complexity of the matcher.

At present we support the use of ``variable-list variables'' like $\nu$ of $x_0$
above, but not the style of using function notation to highlight free
variables, like $P(\nu')$.
Laws using this style can be formulated in \UTP2,
but pattern matching with the requisite level of generality
is extremely difficult.
Formulating laws using explicit substitution ($A_1[x_0/x']$
results in a much more tractable matcher (but still non-trivial).

We now present the \UTP2 approach to variables.

\subsection{Variable Categories}

Variables fall into the following categories:
\begin{description}
  \item[Standard]
    These are are ``ordinary'' variables, that denote observations or values,
    and can typically match in a law against other such variables, or expressions.
    They are represented as sequences of letters, e.g.: $ok, x, y, wait, \ldots$.
  \item[List]
    These are variables that represent a list of variables, in a context were
    variable lists occur, namely in quantifiers and substitutions.
    We can break these down into two distinct classes of list-variable:
    \begin{description}
      \item[Generic]
        These variables represent arbitrary variable-lists,
        and typically are used to denote remaining or left-over lists of variables.
        These are represented as sequences of letters with a $\lst{}$ postfix,
        e.g.: $\lst x, \lst{var}, \ldots$.
      \item[Reserved]
        These are special list variables that act as a shorthand, referring
        to the observational variables of a theory.
        It is important to realise that in many UTP theories
        we actually have two classes of observations: those associated with
        the values of variables in the program text under consideration (e.g. $x$, $y$ and $z$),
        and those that capture overall program properties, independent of any program variable
        (e.g., $ok$ and $ok'$, denoting termination).
        We shall refer to the former as Script variables and the latter as Model variables.
        \begin{eqnarray*}
           OClass &::=& Model | Script
        \end{eqnarray*}
        These are represented succinctly as single-character strings: $\OBS$
        (all observation variables in the current theory alphabet),
        $\MDL$ (all model (auxilliary, non-program) alphabet variables),
        and $\SCR$ (all script (e.g. program) alphabet variables).
    \end{description}
    With both type of list-variables, it is sometimes useful to talk about
    the ``meaning'' of such a variable with some nominated variables removed.
    For example, the meaning of $x:=e$ will want to say that $x'=e$,
    and that ``all the remaining program variables are unchanged''.
    We can assert the latter as predicate $\SCR'\less x = \SCR\less x$
    In general a list-variable can have the form $\lst v\less{x,y,z,\ldots}$,
    which is itself an expression denoting a modified list-variable.
    The nesting of this notation is not permitted.
\end{description}
Note: there have been some key notation changes in this area
since earlier publications
---see Appendix \ref{sec:syntax:change}.

In addition to all of the above,
variables can also have decorations: none; dash or subscripts.
So the following are all valid distinct variables:
\[
x~~~~x'~~~~x_1~~~~x_m~~~~\lst x~~~~\lst x'~~~~\lst x_1~~~~\SCR~~~~\SCR'~~~~\SCR_m
~~~~\SCR\less x~~~~\SCR'\less x~~~~\SCR_m\less x
\]
In the last three examples, we do not add the decoration to the subtracted
variable $x$, instead assuming it implicitly has the same decoration as the
list-variable---contrast $\SCR'\less{x,y}$ with $\SCR'\less{x',y'}$.
The subtraction decoration does not add anything essential.

\subsection{List-variable Semantics}

There are three places in our logic where lists of variables can occur:
\begin{description}
  \item[Quantifiers] $\forall x,y,z @ p$
  \item[Substitution Targets] $q[e_1,e_2,e_3/x,y,z]$
  \item[Substitution Replacements] $r[E,F,G/x,y,z]$
\end{description}
We can imagine capturing the above as following more general patterns:
\begin{description}
  \item[Quantifiers] $\forall \lst v @ p$
  \item[Substitution Targets] $q[\lst E/\lst v]$
  \item[Substitution Replacements] $r[\lst F/\lst v]$
\end{description}
We can recover our original specific instances by positing the
following mapping from list-variables to lists of variables
or lists of expressions:
\begin{eqnarray*}
   \lst v &\mapsto& x,y,z
\\ \lst E &\mapsto& e_1,e_2,e_3
\\ \lst F &\mapsto& E,F,G
\end{eqnarray*}
In effect, the meaning of a list-variable is ultimately
a set of possible lists of zero or more standard variables, or expressions.

For reserved list-variables, the meaning is fixed by the observation variables
of the relevant theory, so if we have a theory of designs with $ok$ as
as a model variable, and $x,y,z$ as program (script) variables, then
the binding is forced to
\begin{eqnarray*}
   \MDL &\mapsto& ok
\\ \SCR &\mapsto& x,y,z
\\ \OBS &\mapsto& ok,x,y,z
\end{eqnarray*}
Note that the mapping always respects the following law: $\OBS=\MDL\cup\SCR$.

If we are in a theory that does not define any particular observation variables,
then the only possible mappings for \OBS,\MDL, and \SCR\ are those to themselves,
or $\OBS \mapsto \MDL,\SCR$.
Such theories do occur, perhaps the best example being a UTP ``base'' theory
that defines ubiquitous concepts such as refinement, conditionals,
and sequential composition.

\subsection{Laws of Interest}
We shall look a small selection of axioms and definitions
focussing on those that make most use of the list-variables as just described.

\subsubsection{Axioms of Interest}

We consider 3 axioms to be worth noting,
either for the matching challenge they present,
or surprising consequences of how they can match:
\begin{description}
  \item[1 Std, and the rest]
     \[
       \AXallOInstN \qquad \AXallOInst
     \]
     This captures a very common idiom where a law basically wants
     to single out a single variable from a list for special treatment.
     It is worth noting that $\lst x$ can match an empty list of variables,
     so the more traditional form of this instantiation axiom
     $(\forall x@p)\implies p[e/x]$ matches the above.
  \item[Ours and Theirs]
     \[
          \AXorAllOScopeN \qquad  \AXorAllOScope
     \]
     This covers a more general case where we want to split a list into two
     parts, often as here, under the influence of a side-condition.
     This points out that the side-conditions have a key role to play
     in matching, as detailed later.
  \item[Hidden Depths]
     \[
       \AXanyODefN \qquad  \AXanyODef
     \]
     This is the standard way to define existential quantification
     in terms of universal, sometimes referred to as ``generalised deMorgan''.
     However, if we consider a matching in which $\lst x$ is matched against
     an empty list of variables, $\lst x \mapsto \emptyset$,
     then the target will look like
     $(\exists \emptyset @ p) \equiv \lnot(\forall\emptyset@\lnot p)$,
     which, if we drop the quantifiers, since they quantify nothing,
     reduces to $p \equiv \lnot\lnot p$--the law of involution!
     At present \UTP2 does perform this match,
     and in particular we have that any predicate $q$
     will match $\forall \lst x @ P$, with bindings $\lst x \mapsto \emptyset, P \mapsto q$.
     This can lead to what look like spurious matches,
     and it is not yet clear if it should be disabled, or retained.
\end{description}

\subsubsection{Definitions of Interest}

The following UTP definitions from various theories
were key drivers for the current architecture of the \UTP2 matching system:
\begin{description}
  \item[Reserved]
    \[\begin{array}{l@{\quad}l}
       \LNAME{$\comp$-def} & P \comp Q \defs \exists \OBS_m @ P[\OBS_m/\OBS'] \land Q[\OBS_m/\OBS]
    \end{array}\]
    This law motivates the need for reserved list variables.
  \item[Designs]
    \[\begin{array}{l@{\quad}l}
       \LNAME{$ok$-decl} & ok,ok' : \Bool \quad \mbox{(Model)}
    \\
       \LNAME{$\design$-def} & P \design Q \defs ok \land P \implies ok' \land Q
       , \qquad ok,ok' \notin \fv.P \cup \fv.Q
    \end{array}\]
    These declaration and definition shows the need for knowing that $ok$ and $ok'$
    are in the alphabet, and that they are not program variables.
  \item[Jay]
    \[\begin{array}{l@{\quad}l}
       \LNAME{$J$-def} & J \defs (ok \implies ok') \land wait'=wait \land tr'=tr \land ref'=ref
    \end{array}\]
    Definitions like these are common in UTP, so need to be supported.
    We also need to ensure that $J$ only ever matches itself or its definition rhs.
  \item[Assignment]
    \[\begin{array}{l@{\quad}l}
       \LNAME{$:=$-def} & v:=e \defs True \design v' = e \land \SCR'\less{v'}=\SCR\less v
    \end{array}\]
    Here we see the need for a reserved variable that only refers to program variables,
    and the need to remove nominated variables from it.
\end{description}
