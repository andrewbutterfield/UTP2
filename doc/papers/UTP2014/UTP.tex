\section{UTP}\label{sec:utp}

\PLAN{ Describing the more ubiquitous operators of UTP,
and why we want to handle alphabets implicitly.}

Some key concepts are common to most UTP theories,
namely sequential composition ($\comp$), non-deterministic choice ($\sqcap$),
refinement ($\refinedby$) and conditional ($\cond c$).
Importantly, in most theories these all have the same definition:
\begin{eqnarray*}
% \nonumber to remove numbering (before each equation)
  P \comp Q &\defs& \exists Obs_m @ P[Obs_m/Obs'] \land Q[Obs_m/Obs] \\
  P \sqcap Q &\defs& P \lor Q \\
  P \refinedby Q &\defs& [ Q \implies P ] \\
  P \cond c Q &\defs& c \land P \lor \lnot c \land Q
\end{eqnarray*}
The definitions for $\sqcap$, $\refinedby$ and $\cond c$ are unproblematical,
and are easily handled by the existing machinery,
with one key extension.
The definition of $\comp$ not only makes use of explicit substitution notation,
but also raises the question of how to interpret $Obs_m$, $Obs'$ and $Obs$.
Clearly they stand for the obervational variables of a UTP theory along with
appropriate decorations, but how do we support this?
In particular, how can we arrange matters so that
we only define $\comp$ once, in such a way that it can be used by many different theories?
We will first address the key extension alluded to above,
and then return to the problem of sequential composition.

\subsection{Defining your own language in \UTP2}

A key aspect of a UTP theory is the signature that captures the
abstract syntax of the language being defined.
This means that \UTP2\ needs to support user-defined languages.
This is achieved by having a table-driven parser for entering predicates,
and providing a facility for the user to add new entries
to the relevant tables:
\begin{eqnarray*}
Theory &=& \textbf{record} \ldots
\\ && precs : Name \pfun Precedence
\\ && lang : Name \pfun LangSpec
\\ && \ldots \textbf{end}
\end{eqnarray*}
The $precs$ table maps the name of an infix operator
to information about its parsing precedence and its associativity.
The $lang$ table maps a language construct name to a language specification
($LangSpec$) that describes the concrete syntactical structure of that construct.
A language specification is a mix of keywords denoting syntactical components
like variables (V) , expressions (E) , predicates (P),
 or various lists of such,
interspersed with concrete syntax symbols.
We won't give a full definition here but present some examples to give the
idea:
\begin{itemize}
  \item
    Refinement: we specify this as ``\verb@P |= P@'',
  which states that \verb@|=@ is an infix operator between two predicates.
  When this is entered into the $lang$ table,
  a corresponding entry is automatically created in the $precs$
  table with default values (mid-range precedence, non-associative)
  which can then be edited by the user to suit.
  Also entered is a dummy definition for the construct
  into the $laws$ table, which itself then needs to be edited.
  \item Assignment:
    specified as ``\verb@V := E@'', stating that \verb@:=@
    is an infix operator in-between a variable and expression,
    resulting in a predicate.
\end{itemize}
In general defining a language construct (resulting in a predicate)
 involves adding entries
to the $lang$ and $laws$ tables,
and possibly also to the $types$ and $precs$ tables, depending
on the precise nature of the construct.
Infix expression operators do not have $lang$ entries
but require $laws$, $precs$ and $types$ entries.

When we talk about developing a theory of Designs (Section \ref{sec:designs}),
we shall give a worked-out example of  a language definition.

\subsection{The problem with $\protect\comp$}

The definition of sequential composition,
\begin{eqnarray*}
P \comp Q &\defs& \exists Obs_m @ P[Obs_m/Obs'] \land Q[Obs_m/Obs]
\end{eqnarray*}
says in effect that for each observation, $x$, say, in $Obs$,
we replace any free occurrence of $x'$ in $p$ by $x_m$
and any free occurrence of $x$ in $q$ by $Obs_m$,
and use existential quantification to hide $x_m$.
In effect the rule above is really a rule-schema,
characterising an infinite number of rules,
one for each possible alphabet represented by $Obs$.
However, we don't want to repeatedly instantiate this rule
and reason about its consequences for each specific alphabet we use.
In fact, we want to use the definition in cases where only part
of the alphabet is known (Designs again, Section \ref{sec:designs}).
We would prefer to be able to do proofs with the definition
as given above, only instantiating $Obs$ where necessary,
and then perhaps only partially.
In fact, we want to support the following proof
(of the associativity of $\comp$)
which does not require any instantation of $Obs$:
{\small
\begin{eqnarray*}
  && P ; (Q ; R)
\\&\equiv& \exists Obs_m @ P[Obs_m/Obs']
           \land (Q;R)[Obs_m/Obs]
\\&\equiv& \exists Obs_m @ P[Obs_m/Obs']
           \land (\exists Obs_n @ Q[Obs_n/Obs'] \land R[Obs_n/Obs])[Obs_m/Obs]
\\&\equiv& \exists Obs_m,Obs_n @ P[Obs_m/Obs']
           \land Q[Obs_n/Obs'][Obs_m/Obs]
           \land R[Obs_n/Obs][Obs_m/Obs]
\\&\equiv& \exists Obs_m,Obs_n @
                  P[Obs_m/Obs'][Obs_n/Obs']
           \land Q[Obs_n,Obs_m/Obs',Obs]
           \land R[Obs_n/Obs]
\\&\equiv& \exists Obs_n @
                  (\exists Obs_m @ P[Obs_m/Obs'][Obs_n/Obs']
           \land Q[Obs_m/Obs][Obs_n/Obs'])
\\&& \qquad\qquad {} \land R[Obs_n/Obs]
\\&\equiv& \exists Obs_n @
                  (\exists Obs_m @ P[Obs_m/Obs']
           \land Q[Obs_m/Obs])[Obs_n/Obs']
           \land R[Obs_n/Obs]
\\&\equiv& \exists Obs_n @
                  (P ; Q)[Obs_n/Obs']
           \land R[Obs_n/Obs]
\\&\equiv&(P ; Q) ; R
\end{eqnarray*}
}
In effect we want to reason within our logic about ``schematic''
variables like $Obs$ and treat the substitution notation as part
of the object logic, rather than meta-notation describing
the behaviour of an inference rule.

To achieve this we have to add another linguistic innovation
to the logic.
A common shorthand in most presentations of logic is to
view $\forall x,y,z @ p$ (say)
as a shorthand for $\forall x @ \forall y @ \forall z @ p$.
Our innovation is not only to add the former as a full part
of the logic syntax, but also a further extension.
We want to be able to have quantifier variables (e.g. $Obs$)
that represent lists of ``ordinary'' quantifier variables.
We do this by splitting the list into two parts, separated by
a semi-colon, with those in the first part being ordinary,
whilst those in the second part denote lists of variables.
The revised syntax of $\forall$ is now:
$$
\forall x_1,\ldots,x_m \qsep xs_1, \ldots , xs_m @P
\qquad
m \geq 0, n \geq 0, m+n \geq 1
$$
Other observation (1st-order) quantifiers are modified similarly.
The $x_i$ and $xs_j$ above are ``quantifier variables'',
and will be disambiguated were necessary by referring to the $x_i$
(before the $\qsep$ sysmbol) as ``single variables''
and the $xs_j$ (after $\qsep$ as ``list variables'').
A list where $m=0$ is referred to as an ``ordinary list''.
The meaning of a quantifier variable list of the
form $x_1,\ldots,x_m \qsep xs_1, \ldots , xs_m$
is that it matches an ordinary list
of the form $y_1,\ldots,y_{m+k}, k \geq 0$
where each $x_i$ binds to one $y_j$,
each $xs_i$ binds to zero or more $y_j$,
and every $y_j$ is bound exactly once.
In principle the bindings associated with a variable like $xs_i$
are non-deterministic, albeit they must be consistent with bindings derived
from the match as a whole, i.e. the wider context in which that  variable occurs.
In practice, heuristics are used in the implementation to select
a binding that is hopefully as ``good'' as possible.

As our proof above largely depended on properties of (explicit)
substitution, we have to add it into our logic as well.
So we revise our syntax for predicates:
\RLEQNS{
   p,q,r \in Pred &::=& \ldots
\\              &|& \yen qvs @ p & \mbox{1st-order Quantifiers, } \yen \in \setof{\forall,\exists,\exists!}
\\              &|& p[ e/x ] & \mbox{Explicit Obs. Substitution}
\\              &|& p[ e/E ] & \mbox{Explicit E-var. Substitution}
\\              &|& p[ p/P ] & \mbox{Explicit P-var. Substitution}
\\ qvs \in QVars &&& \mbox{Quantifier Variable lists}
\\ &::=~& x_1,\ldots,x_m \qsep xs_1, \ldots , xs_m
   & m \geq 0, n \geq 0, m+n \geq 1
}
Explicit substitutions are also added to expressions as well.
Laws regarding explicit substitutions also need to be developed,
e.g.
$$
p[e/x][f/y] = p[e,f/x,y], \quad x \neq y, y \notin \fv.e
$$
but we do not list these here.

This extension allows us to introduce axioms like:
$$
\AXallOInst \qquad \AXallOInstN
$$
rather than relying on a simple single quantifier axiom
and the usual conventions regarding the $\forall x,y,z$ shorthand.
In essence what we have done is to formalise and automate this convention.

To support the definition of $\comp$ we need one further step.
The list variable $Obs$ does not stand for an arbitrary
list of single variables, but is instead intended to stand for
precisely those un-dashed variables that are present in the
alphabet of the current theory, even if that alphabet
has not been fully described.
Similarly, $Obs'$ stands for all the dashed variables,
and $Obs_m$ denotes the decoration of all the $Obs$ variables.
In effect we designate certain list variables (like $Obs$)
as having a special meaning.

The basic matcher described in Section \ref{sec:logic},
has to be enhanced to perform appropriate matching
where non-ordinary quantifier lists are present.
To make this work, we need to extend theories to have
a table that records the theory alphabet:
\begin{eqnarray*}
Theory &=& \textbf{record} \ldots
\\ && obs : Name \pfun Type
\\ && \ldots \textbf{end}
\end{eqnarray*}
The $obs$ table needs to become part of the matching context $\Gamma$,
and we introduce rules for matching quantifier lists:
\begin{eqnarray*}
  & \inferrule%
       {~}%
       {\Gamma \vdash \qsep Obs \ddagger \qsep Obs | \varepsilon }
\\
\\& \inferrule%
       {Obs(\Gamma)=\setof{o_1,\ldots,o_n}}%
       {\Gamma \vdash \qsep Obs \ddagger \setof{o_1,\ldots,o_n}
         | \setof{Obs \mapsto \setof{o_1,\ldots,o_n}}}
\end{eqnarray*}
The first rule allows $Obs$ to match itself, and so we can do proofs
that do not require it to be expanded to an ordinary list.
Note also that in this case an empty binding ($\varepsilon$) is returned.
Other matching rules not shown here, take care of decorations,
ensuring that $Obs$ matches $x,y,z$, if appropriate,
but not $x',y',z'$.

We can now define sequential composition in our revised logic as:
\begin{eqnarray*}
P \comp Q &\defs& \exists \qsep Obs_m @ P[Obs_m/Obs'] \land Q[Obs_m/Obs]
\end{eqnarray*}
and produce a proof as shown earlier.
There is an additional extension required to the logic to do this,
but we shall motivate and introduce it
in the section on Designs (Section \ref{sec:designs}).
