% easychair.tex,v 3.5 2017/03/15

%\documentclass{easychair}
\documentclass[EPiC]{easychair}
%\documentclass[EPiCempty]{easychair}
%\documentclass[debug]{easychair}
%\documentclass[verbose]{easychair}
%\documentclass[notimes]{easychair}
%\documentclass[withtimes]{easychair}
%\documentclass[a4paper]{easychair}
%\documentclass[letterpaper]{easychair}

\usepackage{doc}

% use this if you have a long article and want to create an index
% \usepackage{makeidx}

% In order to save space or manage large tables or figures in a
% landcape-like text, you can use the rotating and pdflscape
% packages. Uncomment the desired from the below.
%
% \usepackage{rotating}
% \usepackage{pdflscape}

% Some of our commands for this guide.
%
\newcommand{\easychair}{\textsf{easychair}}
\newcommand{\miktex}{MiK{\TeX}}
\newcommand{\texniccenter}{{\TeX}nicCenter}
\newcommand{\makefile}{\texttt{Makefile}}
\newcommand{\latexeditor}{LEd}

%\makeindex

%% Front Matter
%%
% Regular title as in the article class.
%
\title{User-Oriented Theorem Prover Design%
\thanks{Thanks!}}

% Authors are joined by \and. Their affiliations are given by \inst, which indexes
% into the list defined using \institute
%
\author{
Andrew Butterfield\inst{1}\thanks{SFI/Lero/UTP}
}

% Institutes for affiliations are also joined by \and,
\institute{
  Trinity College Dublin,
  Ireland\\
  \email{butrfeld@scss.tcd.ie}
 }

%  \authorrunning{} has to be set for the shorter version of the authors' names;
% otherwise a warning will be rendered in the running heads. When processed by
% EasyChair, this command is mandatory: a document without \authorrunning
% will be rejected by EasyChair

\authorrunning{Butterfield}

% \titlerunning{} has to be set to either the main title or its shorter
% version for the running heads. When processed by
% EasyChair, this command is mandatory: a document without \titlerunning
% will be rejected by EasyChair
\titlerunning{User-Oriented TP}

\begin{document}

\maketitle

\begin{abstract}
  What do I want to say?
  Why was Reiner Hahnle's article \cite{RH:UserExpr:2016}such a trigger?
  Are sound trustworthy user-centric theorem provers possible?
  Will automated theorem-proving always be the sole preserve of dedicated users,
  or can it ever become something that anyone with the relevant logic knowledge could just pickup and use?
  How can I best assess and disseminate my ideas about slightly(?) alternative ways
  to go about theorem proving?
\end{abstract}



%------------------------------------------------------------------------------
\section{Introduction}
\label{sect:introduction}


I do semantic theory developmewnt in the
Unifying Theories of Programming (UTP) paradigm\cite{DBLP:conf/RelMiCS/1998},
having looked at slotted-Circus, Handel-C, and probability.
Along the way I got so frustrated with long proofs that
I explored the use of theorem provers.
However, none suited my way of working,
so I took a big leap and started to roll my own
\cite{conf/utp/Butterfield10,conf/utp/Butterfield12,DBLP:journals/corr/Butterfield14}!


My semantic modeling and proof style uses predicate calculus,
classical, usually first order
but with occasional forays into higher orders.
A key point is that I prefer equational reasoning,
keeping the more ``sequent oriented'' material for high level proof planning.

I developed a rapid-prototyping predicate calculator\cite{DBLP:conf/utp/Butterfield16},
driven to it by trying to construct
my compositional semantics for shared-variable concurrency.
The key issue here was that my semantic theory went through a number of
distinct versions before I found one that worked.
Failure at each point emerged after some long drawn-out test calculation
would stall, either with no further progress possible,
or because the emerging result was clearly wrong.
What I needed was a tool to do most of the biolerplate calculation,
with an easy way to add definitions, and an almost complete disregard
for any soundess guarantees.

The key point was that I did not need (nor want) a theorem \emph{prover},
because almost every proposition I investigated was false, either in the logic,
or in my intended interpretation. In fact my starting point whould be
a predicate that was not a theorem, i.e. not universally true.
What I was hoping to do was to use equational calculation to simplify it down to
something that was then clearly correct (w.r.t. my intended interpretation).


The relevance of Reiner H\"{a}hnle\cite{RH:UserExpr:2016}:
I vastly prefer my prover to those in the Isabelle/HOL or Coq style.
I don't have to keep relearning how to do proofs after an absence.
My third year students got the proof bit almost immediately,
complaining instead about my choice of keyboard shortcuts.
It would be lovely to compare these automated proof styles
with mathematical users with no prior theorem prover experience.
To paraphrase the title of David S. Platt's notorious(?) book
about bad software\cite{Platt:2006}:
\begin{quote}
  Why Theorem-Provers Suck\dots and What You Can Do About It
\end{quote}
The main problem that he identifies can be summarised by the following
direct quote:
\begin{quote}
  ``
   Your.
   User.
   Is.
   Not.
   You.
  ''\cite{Platt:2006}
\end{quote}

\section{Discussion Points}

Let's get thought-provoking.

Some strong assertions suggested to me by my experience, but lacking any kind of hard evidence:
\begin{itemize}
  \item
     equational (and inequational) reasoning is easiest, particularly for those new to theorem proving.
  \item
    keep deductive/sequent reasoning styles for top-level proof strategies
  \item
    allow point and click/drag to be used to identify sub-goals
    ---
    the  pretty printers in most theorem proverbs are simply dreadful!
\end{itemize}



\label{sect:bib}
\bibliographystyle{plain}
%\bibliographystyle{alpha}
%\bibliographystyle{unsrt}
%\bibliographystyle{abbrv}
\bibliography{ARCADE2017}


%------------------------------------------------------------------------------
\end{document}
