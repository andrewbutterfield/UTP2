\section{The Logic}
\label{sec:logic:foundation}

\REVIEW2{(OPEN)\\
2) A more detailed description of the meta language and its
logic would be useful. In particular more on how the theory of
free and bound variable side conditions may be used in user
defined constructs.
}


%
% Presents the logic of  Saoithin with an emphasis on foundations
%

A subset of the syntax of the predicate logic of \STHN\ is
shown below:
  \begin{eqnarray*}
    P,Q \in Pred &::=&  \True | \False | \lnot P | P \land P | P \lor P | P \implies P | P \equiv P
  \\           & | & A | \forall \lst v @ P | \exists \lst v @ P
                       | M | P[\lst e/\lst v]
  \\ e,f \in Expr &::=& v | M | \mbox{expressions using }+,\times,\le,\cup,\cat \ldots, v \in Var
%  \\ v &\in& Var
  \end{eqnarray*}
In addition to the usual propositions and quantifiers,
we also have atomic predicates ($A$, boolean-valued expressions),
meta-variables denoting arbitrary predicates or expressions ($M$),
and an explicit substitution notation $P[\lst e/\lst v]$.
The predicate meta-variables allow us to write conjectures and laws true for
arbitrary predicates,
while explicit substitution is required
as many definitions in UTP use it in such an explicit manner.
\NEW{The
axiomatisation being used is one for equational logic
\cite{journal/logcom/Tourlakis01},
extended to support 2nd-order features,
with the inference rules (Liebniz, etc.) effectively
being implemented by the law matcher (\S\ref{sec:match:foundation}).
}
%It is also worth noting that all the boolean connectives live at the
%predicate level, and are not present in expressions.
%So, an atomic predicate is an expression whose type is boolean,
%but whose value is constructed by applying a boolean valued function
%to non-boolean arguments.

Predicates in \STHN\ can participate in a number of roles,
of which the most basic are \emph{laws} and \emph{conjectures}.
Law are either asserted to be true (\emph{axioms}),
or are conjectures that have been proven, and are now \emph{theorems}.
A theorem is then a conjecture coupled with a proof.
Some of the axioms/inference rules of predicate calculus have side-conditions,
but given explicit meta-variables and substitutions,
we find we can no longer treat side-conditions as statically checkable
at law-application time, necessitating an explicit representation:
  \begin{eqnarray*}
     V \in VSet &::=& \power Var
  \\ sc \in Side &::=& True | M = V
                     | M \subseteq V
                     | M \DISJ V
                     | \isCond M
                     | \bigwedge\setof{sc_1,\ldots,sc_n}
  \end{eqnarray*}
We read $\isCond M$ as asserting that $M$ is a \texttt{condition}
(no dashed free variables), and $M \DISJ V$ means the free variables of $M$
are disjoint from $V$.
\NEW{
The use of side-condition expressions is discussed further in
\S\ref{ssec:match:completion}.
}

The basic unit of work in \STHN\ is in fact a predicate coupled
with a side-condition and
from here on we use the terms conjecture and law to refer to such pairs.
So we can view a \STHN\ theory,
as a named collection of named conjectures, theorems and laws:%
\footnote{We use Z schema notation here for convenience%
---we are not presenting a formal Z model of UTP theories.
}
\begin{schema}{Theory}
   name : Name
\\ laws : Name \pfun Pred \times Side
\\ cnjs : Name \pfun Pred \times Side
\\ thms : Name \pfun Pred \times Side \times Proof
\end{schema}
%The theory name uniquely defines that theory and so lives
%in a \emph{theories namespace},
%whilst the names associated with laws, conjectures and theorems
%are in a common namespace local to that theory.
%In particular, conjecture names must be disjoint from those of theorems or laws.
Once a conjecture has been proven it is moved from conjectures
to theorems with the same name.

At present, \STHN\ is very much an experimental tool,
intended in the first instance to support foundational work in the UTP.
As a consequence of this, the axioms and inference rules have not,
in the main, been hard-coded,
but instead the user is free to add their own axioms.
This clearly is very dangerous, but does support experimentation.
For example, two axiomatisations of predicate calculus have been
developed based on \cite{Morgan1989} and \cite{Gries-Schneider94},
and initial experimentation suggests that the axiomatisation in the latter
leads to easier proofs, due to less choice being available at each step.
Versions of \STHN\ that disallow user addition of axioms have been
implemented for use in teaching, and we envisage future versions
of the tool being able to work in different ``user-experience'' modes.
