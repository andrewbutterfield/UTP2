\section{Adding Language}
\label{sec:language:foundation}

%
% Presents Language-building in  Saoithin with an emphasis on foundations
%

The main purpose of UTP is to allow the construction,
comparison and connection of theories about a variety of
modelling, specification and programming languages.
This requires us to be able to specify two key aspects:
\begin{enumerate}
  \item The types of observations of such systems that we wish to discuss,
   captured by the notion of \emph{observation variables};
  \item A description of the language under study: both its syntax and semantics.
\end{enumerate}

\subsection{Alphabets}

A key feature of UTP is its use of \emph{alphabetised} predicates,
where the alphabet is a collection of typed observation variables.
At present, the support in \STHN\
is limited to theories with fixed alphabets, such as the standard
reactive systems theories, or theories where all program variables
are encapsulated in a single $state : Name \pfun Value$ observation.
A proposal to broaden this to cover varying alphabets is under consideration
for a future revision of the prover.
At present, we simply record the (fixed) set of observation
variables and their types in a table:
\begin{schema}{Theory}
   \ldots
\\ obs : Name \pfun Type
\end{schema}
The variables listed in the domain of $objs$ are ``known'' to the theorem prover
and this influences the operation of the matcher.
This information is also fed into the type-inferencing system,
complementing the $types$ information.
We can specify this in the text file, as in the following example
for a theory of designs:
\begin{haskell}
OBSVAR
 ok : B .
 ok' : B
\end{haskell}


\subsection{User-defined Language Constructs}

In order to be able to describe a theory about the language,
we need to be able to describe the language,
so we provide facilities to allow the specification (to \STHN)
of the syntax of language constructs.

First, we note that the expression language is easily extendible
as the parser automatically converts token streams of the form $nm~e$
into the application of (function) $nm$ to expression argument $e$.
However their effective use often requires the user to give them a type,
entered in the $types$ table.
New infix operators however are not automatically parsed,
and need to be declared in advance, using a simple form associating
an operator name with its precedence and associativity.
This infix information is stored in a table in the theory:
\begin{schema}{Theory}
   \ldots
\\ precs : Name \pfun Precedence
\end{schema}
A \emph{predecence} is a pair of a number and an associativity (None/Left/Right),
with higher numbers denoting tighter binding strengths%
\footnote{Currently $\land:80, \lor:60, \implies:30, \equiv:20$}.
In a theory of designs, we want to introduce $\vdash$ as an infix operator
so we can declare it in a PRECEDENCES section:
\begin{haskell}
PRECEDENCES
 |- 55 None
\end{haskell}
A user-defined language construct is an interleaving of existing forms (variables, expressions, types, predicates)
with new tokens, including also some list forms with specified delimiters and separators.
So a theory needs to contain a table listing syntax specifications:
\begin{schema}{Theory}
   \ldots
\\ lang : Name \pfun SynSpec
\end{schema}
There is a simple ASCII syntax for defining new language constructs,
basically an interleaving of keywords
\verb"V" (variable),
\verb"T" (type),
 \verb"E" (expression) and \verb"P" (predicate),
keysymbols \verb"*" (list) and \verb"#" (counted-list)
with arbitrary syntactical elements.
The keysymbols follow a keyword and are themselves followed
by a token denoting a separator.
So \verb"E*," denotes a comma-separated list of expressions,
wherease \verb"V#," is a comma-separated list of variables,
whose length must match that of any other list present
also defined using \verb"#".
A theory of designs and imperative programming might declare its
syntax as follows:
\begin{haskell}
LANGUAGE
  "|-"   | " P |- P " .
  ":="   | " V := E " .
  "::="  | " V#, ::= E#, " .
  ";"    | " P ; P " .
  "**"   | " E ** P "
\end{haskell}
(Here \verb"::=" is intended to be simultaneous assignment).
All language constructs so specified are considered as new instances
of predicates, and so can themselves appear and be parsed as language elements,
so allowing easy nesting of such constructs.
Any specifications that define an infix operator can also have a precedence
declaration.

There are no extra facilities provided to describe the semantics
of user-constructs, as such laws are simply provided by the user as appropriate
axioms in the theory.
For our design theory we might give the semantics of $\vdash$ as follows:
\begin{haskell}
LAWS
  "DEF |-" |  P |- Q  ==  ok /\ P => ok' /\ Q
\end{haskell}
Note these axioms can be written using the full predicate calculus language
shown here, and hence cannot be considered as some form of conservative extension
of pre-existing axioms.
%Hence, care needs to be taken when given such a semantics,
%noting in particular that a pattern-matching semantics style,
%reminiscent of modern functional languages,
%is potentially dangerous, as \STHN\ provides no guarantee%
%\footnote{And never will!}
%about the
%order of such ``pattern matches''.
In addition,
for any law regarding a user-construct to be useable,
there must be a non-empty collection of laws satisfying the following conditions:
\begin{enumerate}
  \item
    Each law has a name prefixed with ``DEF\verb*" "$uname$'',
    where $uname$ is the name in $lang$ of the user-construct.
  \item
    Each law has the form $LHS \equiv RHS$
    where $LHS$ is an instance of the $uname$ construct.
  \item
    Any instance of the language construct must match at least one of these laws.
  \item
    The $RHS$s must not mention the construct explicitly,
    nor should it be possible to construct a cycle
    via mutual recursion with other user-constructs.
\end{enumerate}
The reason for this set of restrictions
is because these laws will be used to automatically expand language
definitions ``under the hood'' in order to evaluate free variables
and side-conditions (\S\ref{ssec:free-vars}).
They do not preclude our introducing laws with language recursion,
provided their names do not start with ``DEF\verb*" "$uname$''.

Having introduced our language constructs we then will want to give their
semantic definitions as axioms, posit some conjectures,
and hopefully prove them to be laws of the language.
However, the parser needs to know about language syntax before
the rest of the theory text file can be parsed,
as we are introducing new lexical elements.
So every UTP text file has to have a syntax preamble,
immediately after the first line gving the theory name.
The preamble has the following format:
{\obeylines
\texttt{SYNTAXFROM}
  list of zero or more theory names, separated by spaces
  optional \texttt{LANGUAGE} and \texttt{PRECEDENCES} sections in any order
\texttt{ENDSYNTAX}
}
\noindent
The theories listed are those with syntax sections defining language
constructs used in this theory in addition to the ones it introduces itself.

%\subsection{On Metavariables}
%
%An issues arises concerning the interaction of ``regular'' variables
%and expression meta-variables.
%Consider the follow two laws, one drawn from a theory of lists,
%whilst the other comes from a theory of a sequential imperative language:
%\begin{eqnarray*}
%  \ell_1 \cat (\ell_2 \cat \ell_3) &~~=~~& (\ell_1 \cat \ell_2) \cat \ell_3
%\\ x:=e ; y:=f &=& y:=f[e/x] ; x:=e, \qquad y \notin e
%\end{eqnarray*}
%The question then arises, which of $\ell_i$, $x$, $y$, $e$ and $f$
%are ordinary (observation/quantifiable) variables,
%and which are expression meta-variables?
%The solution adopted treats a variable as ordinary, except if it is ``known''
%as an expression meta-variable, or appears as such in a side condition.
%So, in the above, only $e$ needs to be treated as a metavariable.
