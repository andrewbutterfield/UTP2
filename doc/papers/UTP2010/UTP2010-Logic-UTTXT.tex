
The interface to the logic presented to the user is,
in the main, based on an ASCII-based concrete syntax,
that imitates the mathematical syntax as far as is practicable.
This ASCII syntax is used for both input and output.
For example, the predicates
\begin{equation*}
 [ (\exists c @ D \land L) \implies S ]
 \qquad
 P[\True/ok]
\end{equation*}
are written ASCII-style as:
\begin{center}
    \verb"[[ ( exists c @ D /\ L ) => S ]]       P[ TRUE // ok ]"
\end{center}
Similarly, side condition
$
   \bigwedge\setof{Q \subseteq \setof{a,b}, e \DISJ \setof{b,c} }
$
is written as
\begin{center}
    \verb"coverP Q a,b ; notinE e b,c"
\end{center}

An almost complete theory (less theorems)
can be input and output as a ASCII text file,
whose extension is \verb".uttxt" (UTP Text).
The file is structured as a series of sections,
each flagged by a header keyword.
The first collection of sections,
between SYNTAXFROM and ENDSYNTAX keywords
provide information to support the parser.
The remaining sections provide semantics information.
The first line of such text files is the theory name,
optionally followed by a version number.
So a simple theory defining some laws and conjectures about conjunction
and disjunction could be written as:
\begin{haskell}
ConjDisj 0
SYNTAXFROM
 Logic
ENDSYNTAX
LAWS
  "/\-comm" | P /\ Q == Q /\ P .
  "\/-def"  | P \/ Q == ~(~P /\ ~Q)
CONJECTURES
  "\/-comm" | P \/ Q == Q \/ P
END
\end{haskell}
