\section{Usage}
\label{sec:usage}

\REVIEW1{(CLOSED)\\
The author is supposed to describe the implementation algorithm
of the prover more clearly and give the performance evaluation
of this tool.
}

Here we show a proof
(that $x:=e;x:=f~=~ x:=f[e/x]$),
using the reduce both sides strategy (Fig.\ref{fig:UI:proof-example}).
\begin{figure}[tb]
  \includegraphics[scale=0.55]{\imgfile{proof-example}}
  \caption{\STHN\ Proof Example}
  \label{fig:UI:proof-example}
\end{figure}
A number of steps have already been performed, expanding
the definitions of $:=$ and $\vdash$, with the steps shown in reverse order
(first proof step last).
The user has right-clicked, bringing up a possible list of replacements.
Law matching can return a large number of successful matches,
and these need to be presented to the user in a manner that makes it easy
for them to select the law they wish to apply. There are basically two
ways for a user to request law matching:
one displays all the matches in a special window for the user to peruse,
whilst the other displays up to 20 of them
in a popup menu for immediate selection.
In both cases we need some way to rank matches so the most useful appear first.
In essence, we need some form of heuristics that can take a law match and compute
a ranking number.
There has been a limited amount of experimentation in this area,
and it is a difficult area to get right.
While at present the heuristics
are hard-coded, we expect to improve matters to give the users more control
in this area in future.
The performance of the law matching is excellent at present,
with a virtually instant return of results, while the GUI performance
can be a little slow when inspecting large theory tables,
so this is an obvious area for improvement.

Earlier versions of \STHN\ have been tested on 3rd-year Computer Science
students at TCD who have taken an elective module on Formal Methods,
which presents the subject through the UTP framework.
Initially they were shown the use of the prover using a logic module
with initial laws drawn from the axiomatization of equational logic
in \cite{Gries-Schneider94},
material that they studied in their first year.
Then they were asked to prove conjectures arising from useful library theories,
such as Sets or Lists.
Most problems reported by the students had to do with installation or user-interface
behaviour issues%
---most students found the proof aspects easy to apply once these initial hurdles
were overcome, and feedback from them has been used to improve the tool.
We have used the tool to produce a wide range of proofs of UTP relevance,
of which the following is just a short representative list:
\begin{mathpar}
(\sigma \cat \tau) - \sigma = \tau
\and S_1 \cap (S_2 \cup S_3) = (S_1 \cap S_2) \cup (S_1 \cap S_3)
\and P \vdash Q \equiv P \vdash P \land Q
\and  x:=e; y:=f \equiv y:=f[e/x]; x:=e, \quad y \notin e
\and \RR3(\RR3(P)) \equiv \RR3
\and \RR2 \circ \RR2 = \RR2
\end{mathpar}
