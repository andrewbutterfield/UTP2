\section{Proofs}
\label{sec:proof:foundation}

\REVIEW1{(OPEN)\\
The author is supposed to describe the implementation algorithm
of the prover more clearly and give the performance evaluation
of this tool.
}

\DRAFT{
 I'll add some ``implementation'' comments here.
 The performance will be commented on in the usage section
}

\REVIEW2{(OPEN)\\
3) A minimal axioms for the built assumptions for the matcher
and how, if at all, the theory of matching used compares to
"unification in a theory" (i.e. unification in the presence of
a background theory of operators, such as
associative-commutative unification). Here there is also an
opportunity to compare the matching strategies used in other
proof assistants such as Isabelle, Mizar or the B-tool.
}

\DRAFT{
  Will look up Unification/Unification wrt Theory
}
\DRAFT{
  Will look up matching in Isabelle/Mizar/B-tool
}

%
% Presents the Proof system of  Saoithin with an emphasis on foundations
%

Given a conjecture in a theory,
any attempt to prove it will take place in a general proof context,
adopting one of a number of available strategies,
and making use of a variety of builtin proof procedures.
In general, we do not work with a single theory, but rather
a stack of them: the bottommost are the most fundamental and general
whilst theories higher up the stack tend to be more specific and higher-level.
The key idea is that a proof of a conjecture can depend on its own theory,
as well as material from theories below it in the stack.
The stack concept is provided to assist in the encapsulation of theories,
and for this reason circular dependencies are not allowed.

When initially started, \STHN\ has a proof stack with a single entry,
the $\_ROOT$ theory, defining the precedences of the propositional infix
connectives. The user can then load up the desired theories,
and the current state of the stack is then displayed in
the application's main (``top'') window:
\begin{center}
  \includegraphics[scale=0.8]{\imgfile{top-window}}
\end{center}
This stack state is saved into a file upon exiting the application,
and restored automatically upon re-entry.
%A typical stack setup for a theory of Reactive Systems (say) in UTP
%might look like the following: Logic; Equality; UTP; Arithmetic; Sets; Lists; Reactive.
%Most are self-explanatory --- UTP is a theory with general definitions common
%to almost all UTP theories.
Double-clicking on a theory name opens up a tabbed window to
allow theory components to be inspected:
\begin{center}
  \includegraphics[scale=0.8]{\imgfile{theory-conj-window}}
\end{center}

There are a number of strategies we can employ to prove a goal conjecture $G$,
based on its top-level structure:
\begin{description}
  \item[Reduce/Deduce]
    We can simply attempt to take $G$ and use proof procedures
    to transform it to $\True$
    (or equivalently, an existing law).
  \item[LHS-to-RHS]
    If goal $G$ has the form $L \equiv R$,
    we can try to transform the LHS $L$ until it is equivalent
    to $R$
    (we could also try \textbf{RHS-to-LHS}).
  \item[Reduce-Both]
    Again, for the form $L \equiv R$, a way to proceed
    might be to try and transform both sides into something equivalent.
  \item[Law-Reduce]
    Rather than starting with the conjecture,
    we start with an instantiation of an existing law,
    and transform that until equivalent to our conjecture.
%    Because we start with an established law,
%    each step we make results in a theorem
%    so we can exploit
%    the following meta-theorem:
%    if any of the following is a theorem then they all are:
%    $$
%      P \qquad [P] \qquad \forall x_1,\ldots,x_n @ P,
%      \mbox{ for any }x_i
%    $$
%    This means that each check for equivalence between the current goal
%    and the conjecture can strip off any top-level universal quantification.
    This strategy is the basis for inductive proofs in \STHN.
  \item[Assume-then-\ldots]
    If the conjecture has the form $A \implies C$,
    we can assume the LHS antecedent $A$ as a law, adding it to the proof
    context, and have it available to assist in proving the consequent $C$
    using some other appropriate strategy.
\end{description}
\STHN\ supports the above strategies,
and will support more in the future.
In general, the chosen strategy may modify both the conjecture
and its proof context in some way,
and will have a completion criteria that depends on that modification.
We shall now discuss some of the concepts just introduced in a little
more detail.

Double clicking on a conjecture starts a proof by opening
up a proof window. From the Setup menu a strategy can be chosen,
for example Reduce:
\begin{center}
  \includegraphics[scale=0.8]{\imgfile{proof-window}}
\end{center}
In this case the entire goal appears, with the focus,
currently the whole goal, underlined.
Given that all atomic predicates are expressions of boolean type,
we find more useful typing information to be the notion
of \emph{underlying type}: the types of the sub-expressions
just below the top-level of the atomic predicate.
As the goal is an atomic predicate comparing two integers,
we see that its underlying type is $\Int \cross \Int$.
If attention moves down into sub-expressions,
then the full type of that component is displayed,
as the notion of underlying type
only applies to atomic predicates at the top-level.

\subsection{Transformations}

We mentioned taking a goal and transforming it in some way.
As \STHN\ favours the use of equational reasoning,
such transformations generally involve replacing
the goal, or a sub-part of it, by a logically equivalent expression.
So a proof becomes a chain of equivalences linking the
initial goal to one that is equivalent to the required final predicate,
the specifics of which are of course strategy-dependent.
%While not yet implemented, \STHN\ does have `hooks' to support
%inequational reasoning,
%where the transformations involve
%implication rather than equivalence---so for example,
%a goal of the form $A \implies C$ could be proven
%by starting with $A$ and transforming it to $C$ with a mixture
%of equivalent and implicative transforms.

The primary mechanism for effecting such transformations
is a process of selecting a sub-part of the current goal (the ``focus'')
and then applying the desired rule.
The rules available for application are dependent on the focus,
but always include the option to match it against the laws
in all the currently accessible theories.
The details of this matching is discussed later (\S\ref{sec:match:foundation}),
and for now we give a brief survey of other builtin transformers that can be used:
\begin{description}
  \item[Tidying]
    Basically a collection of builtin operations to flatten and sort
    predicates whose top-levels are disjunctions and conjunctions.
  \item[Simplify]
    Effectively constant-folding, doing fairly obvious propositional simplifications.
  \item[Normal Forms]
    Converting predicates to either disjunctive or conjunctive normal form
  \item[Application]
    Reducing applications of abstractions to arguments ($\beta$-reduction)
    or applying explicit substitutions to the underlying predicate.
  \item[Quantifiers]
    Specifying explicit $\alpha$-substitution, or existential witnesses.
    If using the Law-Reduce strategy, then we can also strip out top-level
    universal quantifiers, provided the focus is the whole goal.
  \item[Splitting]
    Given long conjunction/disjunctions,
    it is often useful to be able to re-structure them into
    specific groups in order to allow certain laws to become applicable.
%    For example, grouping terms with no bound variables would be a useful
%    prelude to a law that moves them outside a quantifier.
\end{description}
Most of these can be invoked by a single key shortcut (see Help menu in Proof Window)

A proof is complete when goal transformation
makes it ``equivalent'' to an end-point predicate
as determined by the strategy in use.
The notion of equivalence we use is that of
a slight generalisation of $\alpha$-equivalence,
that flattens nested quantifiers of the same type,
so that the following are considered equivalent:
\begin{equation*}
 (\forall x @ \forall y @ P)
 \qquad
 (\forall x,y @ P)
 \qquad
 (\forall y,x @ P)
 \qquad
 (\forall y @ \forall x @ P)
\end{equation*}



\subsection{Shorthands}\label{ssec:shorthands}

A useful facility in \STHN\ is the ability to define meta-variables
as shorthands for longer, more complex predicates.
Examples of this in UTP include the definition of $J$ in the
reactive theory \cite[Chp.8, p208]{UTP-book}:
\begin{eqnarray*}
  J &\defs& (ok \implies ok') \land wait'=wait \land tr'=tr \land ref'=ref
\end{eqnarray*}
These definitions can be made as part of a theory,
or can be introduced on the fly during a proof to replace
a long predicate by a shorthand.
During a proof it is then possible to replace such a meta-variable
by its definition, or even to recursively expand all such shorthands
in a predicate---even if the shorthands are themselves mutually recursive.
To support this facility, we must add tables to a theory to record
these shorthands, which can be introduced for type and predicate
meta-variables, as well as regular variables:
\begin{schema}{Theory}
   \ldots
\\ typedef : Name \pfun Type
\\ exprdef : Name \pfun Expr
\\ preddef : Name \pfun Pred
\end{schema}
All of the names in these tables are ``known'',
which affects how they participate in law matching.
These tables also have another use to ensure the soundness
of certain strategies, as detailed below.

\subsection{Assumptions}

The ``Assume-'' strategies are an implementation of the Deduction Theorem:
$$
  \Phi \infer A \implies C \quad \mbox{iff} \quad \Phi,A \infer C
$$
where the antecedent is elevated (temporarily) to a law
for the purposes of this proof alone.
However we need to be careful here \cite[p72]{Gries-Schneider94},
to ensure that all meta-variables in the ``assumption-now-law''
can only match against themselves---in essence any such variables
in $A$ above must be temporarily restricted to only stand for themselves.
We create a temporary theory on the top of the theory-stack,
and add in as new laws, the antecedent $A$, suitably decomposed (e.g.):
\begin{mathpar}
  \Phi \infer (A_1 \land A_2 \land A_3) \implies ((A_4 \land A_5 \land A_6) \implies C)
\\ \downharpoonleft
\\ \Phi,A_1,A_2,A_3,A_4,A_5,A_6 \infer C
\end{mathpar}
We then use the meta-variable tables just introduced in \S\ref{ssec:shorthands},
adding in entries of the form
$
  M \defs M
$
for every ``unknown'' meta-variable in $A$,
\NEW{and finally renaming all the $M$ in these tables, $A_i$ and $C$
to be fresh, so as to avoid clashes with other laws.
}
This effectively prevents the law matching from binding these meta-variables
to anything else during the proof.
Once the proof is complete,
the temporary theory is removed.


\subsection{Induction}

The current support for induction is via the Law-Reduce strategy,
using the appropriate induction axiom, instantiated appropriately
with the conjecture. This is then simplified until the the goal is itself reached.
For example, we may have the following inductive axiom for natural numbers:
$$
 M[0/x] \land (\forall y @ M[y/x] \implies M[y+1/x]) \implies \forall x @ M
$$
(Note that the assumption here is that $x$, if free in $M$ is a natural number).
Given addition defined recursively on its left argument,
we may wish to prove that $n+0 = n$, for all natural $n$.
If we invoke the Induction strategy in \STHN, and upon selecting the above
induction axiom, and identifying $n$ as our induction variable,
a goal is generated by substituting our conjecture for $M$,
and $n$ for $x$,
to obtain:
$$
 (n+0 = n)[0/n]
 \land (\forall y @ (n+0 = n)[y/n] \implies (n+0 = n)[y+1/n])
 \implies \forall n @ (n+0 = n)
$$
Switching to an Assume strategy here doesn't help as we want to manipulate
the antecedents. Successive transformations, applying substitutions
and using the definition
of $+$, plus other axioms associated with the natural numbers, allow us
to reduce the antecedents to true, leaving
$$
 \forall n @ (n+0 = n)
$$
Because we are reducing from an instance of an axiom,
this is a theorem, so we can strip off the top-level universal quantification
to get our original conjecture:
$$
n+0 = n
$$
At present \STHN\ requires us to handle induction proofs as one lump as just illustrated,
but future strategy/case-handling enhancements will make it possible to do a more
``traditional'' proof, where the base and inductive cases are effectively
proven separately.

\subsection{Caveat Emptor}

Before moving on to look at the law matching facilities
we need to point out an important aspect of the transformations discussed
in this section: they have built-in to them certain assumptions
regarding such notions as $\alpha$-equivalence, the associativity, commutativity
and idempotence of conjunction and disjunction, and implication trading,
to name but a few.
It is therefore important that any axioms introduced in theories
do not conflict with these internal assumptions regarding the logic%
\footnote{This is in fact an argument for having a built-in logic theory
that is compatible with the above assumptions.}%
.
