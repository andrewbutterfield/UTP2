\section{Conclusions and Future Work}
\label{sec:conclusions}

We have given motivation for, and presented, \STHN,
a theorem-proving assistant for UTP.
The current state of the tool is still experimental,
with considerable scope for enhancement and improvement,
but is already a useful tool for experienced\footnote{Not too trusting!}
developers of UTP foundations.

Apart from the obvious need for comprehensive library of useful theorems
from general logic, arithmetic, set, list and map theory,
there are still a few foundational issues that need to be resolved.
The current limitation to fixed alphabets is too restrictive,
but we have a plan, using special meta-variables known to the pattern matcher,
to be able to describe generic laws that cross a number of theories:
the gold standard here would be a definition of sequential composition\
\begin{center}
$
  P ; Q
  ~~\defs~~
  \exists OBS_m @ P[OBS_m/OBS'] \land Q[OBS_m/OBS],
   \quad OBS_m \mbox{ fresh}
$
\end{center}
that works in any theory whose observations are homogenous
(meta variables $OBS_m$, $OBS'$ and $OBS$ would need special pattern matching
and binding treatment).
The treatment of undefinedness in a way that limits its impact on general proving
is still under exploration, and we are exploring various ideas in this area
\cite{wkshp/mpf/arthan:96}.
Given the existing work on UTP and Circus using Proofpower-Z,
we also hope to explore the ability to take the GUI front-end of \STHN,
and couple it to ProofPower-Z as a backend, given some appropriate transformation
of how proofs and theorems are presented.

\subsection{Acknowledgments \& Availability}


We would like to thank Colm Bhandal, Karen Forde, Simon Dardis,
Jim Woodcock and the Formal Methods classes of 2009 and 2010
for their work on, and feedback about, \STHN. \STHN\ is written
in Haskell \cite{rep:Haskell:98}, managed using the distributed
source-code manager Mercurial \cite{bk:mercurial:BOS:09}, and
is released open-source under GPL v2, at
\texttt{\url{http://www.scss.tcd.ie/Andrew.Butterfield/Saoithin/}}.
The version available at the time of writing is 0.86$\alpha$10.
