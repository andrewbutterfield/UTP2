\section{Matching}
\label{sec:match:foundation}

\REVIEW2{(OPEN)\\
3) A minimal axioms for the built assumptions for the matcher
and how, if at all, the theory of matching used compares to
"unification in a theory" (i.e. unification in the presence of
a background theory of operators, such as
associative-commutative unification). Here there is also an
opportunity to compare the matching strategies used in other
proof assistants such as Isabell, Mizer or the B-tool.
}

\DRAFT{
 Reading \cite{wkshp/IWU/BL98}.
 Looks like we do simple syntactic matching,
 not even unification.
}
%
% Presents the matching of  Saoithin with an emphasis on foundations
%

The basic idea behind law application in \STHN\ is that
the goal focus is matched against one or many laws.
Those matches that succeed return a binding and a replacement template,
and the user can then select which successful match to apply.
The focus is replaced by the template, instantiated using the bindings.
\NEW{
  The matching we perform is basic structural matching
  of one test predicate against another treated as a pattern,
  with success returning a binding that matches pattern variables
  to test sub-predicates.
  We are not performing unification, either in isolation,
  or w.r.t. to some equational theory \cite{wkshp/IWU/BL98}.
  This is in contrast to Isabelle, for example,
  which uses higher-order unification \cite{tr:Isabelle:Paulson88}.
}
However, we do not simply match $F$ against all of $L$
---if $L$ has certain forms, we can match against part of the law,
offerring another part as a replacement:
\begin{description}
  \item[$L$ is  $P \equiv Q$]:
    If the law is an equivalence we look for matches against the LHS and RHS
    separately, offering the other side as a replacement.
%    Some laws have the form $F(M) \equiv M$,
%    for some meta-variable $M$.
%    Such laws will always succeed in matching their RHS against any focus predicate,
%    so to avoid vast numbers of spurious matches they are handled separately.
  \item[$L$ is $A \implies C$]:
    Given an implication law, a partial match against either the antecedent
    or consequent is still useful.
    Using equational reasoning, a match against $A$ can offer $A \land C$
    as a replacement, whilst a match against $C$ can be replaced by $C \lor A$.
%    If doing inequational reasoning we can have a transformation
%    linked by the appropriately ordered implication relation,
%    provided the focus is in a clear positive or negative position
%    in the overall goal.
\end{description}
So the upshot is that we can get a number of different types of matches
against laws with a certain structure.
We now proceed to discuss the process of matching.

\subsection{Basic Matching}

The first phase of matching is a basic structural comparison
of a test predicate $Q$ (the goal focus) against a pattern predicate $P$ (part of a law),
which either fails,
or succeeds and returns an incomplete binding,
mapping pattern meta-variables to corresponding fragments of the test predicate.
%If we denote a matching as $Q \approx P$, then some obvious matching outcomes
%are illustrated here:
%  \begin{eqnarray*}
%     \True \approx \True &\defs& \mapof{}
%  \\ Q_1 \land Q_2 \approx P_1 \land P_2
%    &\defs&
%    (Q_1 \approx P_1) \uplus (Q_2 \approx P_2)
%  \\ Q \approx M &\defs& \mapof{M \to Q}, M \mbox{ not known}
%  \\ M \approx M &\defs& \mapof{M \to M}
%  \\ Q \approx M &\defs& \mapof{M \to Q}, preddef(M) = Q
%  \end{eqnarray*}
A key feature to note is that pattern variables and meta-variables
match anything of the appropriate class (expr/type/predicate),
provided they are not ``known''.
Remember, a name is ``known'' if it appears in any of the following theory
tables: $obs$, $typedef$, $exprdef$ or $preddef$.
A known name can only match against either itself
or its definition as found in those tables.

None of the above gives any clue as to why basic matching
returns an incomplete binding.
The reason for this lie in three areas: our use of
explicit substitutions, the fact that ordering is not important
in substitutions or quantifier lists,
and our desire to support a lot of flexibility in matching quantifiers.
The motivation for the latter stems from the fact that \STHN\ is designed
to support foundational work in the UTP,
and so we want to be able to derive very general laws.
For example, we want to have general laws like:
$$
  (\forall x,x_1,\ldots,x_n @ P )
  \equiv
  (\forall x_1,\ldots,x_n @ P ), \quad x \notin P
$$
where we can somehow specify that $n \geq 0$ and avoid listing the $x_i$.
Also, given the above law
and the axiom
$$
  (\exists x_1,\ldots,x_n @ P)
  \equiv
  \lnot (\forall x_1,\ldots,x_n @ \lnot P )
$$
we would like to be able to prove
$$
  (\exists x,x_1,\ldots,x_n @ P )
  \equiv
  (\exists x_1,\ldots,x_n @ P ), \quad x \notin P
$$
To this end, the syntax of quantifiers in the logic of \STHN\ is in fact
more complex than suggested so far.
Instead we view a quantifier as being followed by
two comma-separated lists of variables,
with a semicolon separating the two lists:
  $$\begin{array}{lc@{\qquad}l}
     \forall x_1,\ldots,x_m~; xs_1,\ldots,xs_n @ P
     && m \mbox{ s-qvars, } n \mbox{ m-qvars}
  \\ \forall x_1,\ldots,x_m @ P
     && m \mbox{ s-qvars, } 0 \mbox{ m-qvars}
  \\ \forall ~~;xs_1,\ldots,xs_n @ P
     && 0 \mbox{ s-qvars, } n \mbox{ m-qvars}
  \end{array}$$
Either list may be empty.
The first list is of so-called ``single'' quantifier variables (\emph{s-qvars}),
each of which corresponds to a conventional quantifier variable,
and, when occurring in a pattern predicate, are required to match
distinct s-qvars from the test predicate.
The second list is of ``multiple'' quantifier variables (\emph{m-qvars}),
which are intended to represent many quantifier variables.
A m-qvar in a pattern can match any mixture of s-qvars and m-qvars,
including none.
Matching is now complicated by the fact that the ordering
of s-qvars, m-qvars  is irrelevant,
so in principle we have to try every possible permutation.
Basic matching  will succeed for simple cases, basically
those with only one pattern qvar,
but otherwise will return a deferred match in the form of
a test-qvars/pattern-qvars pair,
in addition to any bindings
from the quantifier bodies.
The basic matching is then followed by two phases,
one that tries to resolve the deferred matches,
and then a step to check the side-conditions.
Both of these phases need to know about the free variables of the goal.

\subsection{Free Variables}
\label{ssec:free-vars}

Many laws have side-conditions attached,
all of which are conditions regarding the free-variables.
However, the presence of explicit meta-variables and substitution
 complicates the calculation of the free variables of a predicate.
To see this, we first present some clauses of the definition of free variables
of a predicate:
  \begin{eqnarray*}
     \fv e &=& \mbox{all variables free in }e
  \\ \fv (P_1 \land P_2) &=& (\fv P_1) \cup (\fv P_2)
  \\ \fv (\forall \lst v @ e) &=& (\fv e) \setminus \lst v
  \\ \fv M &=& \fv.M
  \\ \fv P[e_1,\ldots,e_n/v_1,\ldots,v_n]
     &=& (\fv P)\setminus\setof{v_1,\ldots,v_n}
  \\ && {} \union
         \bigcup \setof{ i : 1 \ldots n | v_i \in \fv P @ \fv e_i }
  \end{eqnarray*}
We see that for meta-variables we need to return the fact that computation
of their free variables needs to be held until
they are instantiated at some later point in time.
With substitution, we see that the resulting set is contingent
upon the freeness or otherwise in $P$
of the substitution target variables ($v_1,\ldots,v_n$).
The upshot of all this is that we need to represent the free-variables
of predicates using a free-variable set-expression, rather than
just a simple variable enumeration.
The required syntax of such expressions is as follows:
  $$\begin{array}{rcll}
     s \in SetExpr & ::= & \setof{ v_1, \ldots, v_n} & \mbox{Enumeration, }n \geq 0
  \\               &  |  & \fv.M & \mbox{Meta free-variables}
  \\               &  |  & s_2 \setminus \setof{ v_1, \ldots , v_n}
                        & \mbox{Set Difference, } n \geq 0
  \\               &  |  & \bigcup (s_1,\ldots,s_n) & \mbox{Union, }n \geq 0
  \\               &  |  & m \sthen s & \mbox{Conditional Set}
  \\ m \in Member  & ::= & a \in s & \mbox{Element Membership}
  \\              &  |  & \bigwedge (m_1,\ldots,m_n) & \mbox{Conjunction, }n \geq 2
  \end{array}$$
Most are self explanatory, except conditional set membership
$m \sthen s$, which denotes set $s$ if $m$ is true, otherwise is empty.

A further complication arises with the user-defined language constructs.
The free variables of these can only be established by expanding out their
definitions.
If we were simply to report ``cannot tell'' for the free variables
of such constructs,
then we would be unable to match any law involving
a user-construct that had a side-condition,
and would be forced to expand them out as an explicit proof step in any case.
As one of the aims of UTP is to support laws at the language level,
without always having to expand out the underlying predicate form,
this is not a viable solution.

Accordingly, the algorithm for computing free variables
expands language definitions on-the-fly, to get to a predicate form
to establish its free-variables.
This is why every language construct has to have defining laws
with the name and form described in \S\ref{sec:language:foundation}.
This on-the-fly activity is invisible to the user, occurring behind the scenes.
This explains why the language definition laws cannot be mutually recursive,
otherwise this procedure could not be guaranteed to terminate.


\subsection{Match Completion}
\label{ssec:match:completion}

Once a basic match is done,
we want to try to complete it by using context information
to figure out a suitable qvar matching.
The context information we use includes the bindings already obtained,
some which may pre-determine how qvars should match,
as well as any information regarding the free variables of
the focus, as well as side-conditions for both the goal and the relevant law.
This process is quite complex and we omit any further details,
apart form noting that in complicated cases it may fail to find a valid matching
even if one exists.
However there is a work-around by using other proof steps
to re-organise quantifiers until matching does work.
For a similar reason, the checking of $\alpha$-equivalence
used to terminate a proof is also incomplete and may fail to detect
such an equivalence --- the same work-around can be used here as well.

Finally, we check that the law side-condition, if any,
when translated in terms of the goal using the bindings,
is actually a consequence of the goal side-condition, if any:
$sc_{law} \land sc_{goal} \equiv sc_{goal}$.
\NEW{This is achieved by code that automatically
evaluates/simplifies a side-condition
using free-variable set-expressions
and then establishes that the above equivalence holds.
This is facilitated by the existence of a normal-form for side-conditions.
}
