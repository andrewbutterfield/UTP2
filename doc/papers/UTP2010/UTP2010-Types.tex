\section{Types}
\label{sec:types:foundation}

\REVIEW2{(OPEN)\\
1) a fuller discussion of the type system e.g.
  a) does it include generics or some variety of polymorphism;
  b) what is (type) relationship between function types and powerset types (e.g. does the type checker generate side conditions?)
}

\DRAFT{
 Polymorphism (Haskell) a.k.a. Generic (Java),
 and we do not relate power set and function type at the type-checker level
}

%
% Presents expression typing in  Saoithin with an emphasis on foundations
%

We have a fairly simple notion of types
with booleans and integers making up basic types,
a basic type $Env$ that denotes
a program environment ($Name \pfun Value$),
type variables,
and then the capacity to build up set/sequence/map/free types on top
of these:
  $$\begin{array}{rclr}
     T \in Type &::=& \Bool | \Int  | Env & \mbox{Base Types}
  \\  &|&  \tau | ~? {} &\mbox{Type variables}
  \\  &|& \power T | T^* | T^+ | T \cross \cdots \cross T & \mbox{Composite types}
  \\  &|& T \fun T | T \pfun T | T \ffun T & \mbox{Function types}
  \\  &|& nm @ V~`\!|' \ldots `\!|' V & \mbox{Free Types}
  \\ V \in Variant &::=& nm~\ldata T \cross \cdots \cross T \rdata & \mbox{Free Type Variant}
  \\ nm &\in& Name & \mbox{Names}
  \end{array}$$
\NEW{The type-system supports Hindley-Milner style polymorphism,
and, for simplicity, treats powerset, sequence and function types as distinct%
\footnote{
  We could add axioms to a theory relating type-assertion predicates if we wished
  to link these types.
}
.
}
At present the main role played by types in the prover is to limit
the search for applicable laws to those that match the types of expression
involved.
To this end a type-inferencing algorithm is used to associate types
with all expressions. It uses user-supplied information about the types
of named functions to deduce the relevant types for entire expressions.
This information is stored in a table matching function names to their types,
which of course has to be an additional component of a theory:
\begin{schema}{Theory}
   \ldots
\\ types : Name \pfun Type
\end{schema}
%We should stress that the intended use of this table
%is to associate types with the names of static functions
%whose meaning and definition is fixed in a theory.
%Examples of such functions might the standard binary arithmetic operators
%like $+$,$-$,$\times$, or standard list operators and functions like $\cat$
%and
%$length$.
%Type-inferencing is performed at a predicate level,
%with all the expressions, extracted from its atomic predicates,
%being treated together.
%Given that our expression language does not contain boolean connectives
%then the only way an expression $e$ can have type $Bool$ are enumerated as follows:
%\begin{enumerate}
%  \item a boolean constant or variable ($b:\Bool$);
%  \item an application of a function of type $T \fun \Bool$,
%        where $T$ is a non-boolean type.
%\end{enumerate}

%In the first case above the underlying type is $\Bool$,
%while in  the second case it is $T$.
%If $\Gamma \infer e \ddagger T$ means that $e$ has underlying type $T$
%in context $\Gamma$, then the following rules allow this to be inferred:
%\begin{mathpar}
%   \INFER%
%     {\Gamma \infer v :: \Bool}
%     {\Gamma \infer v \ddagger \Bool}  [UVar]
%     \qquad
%   \INFER%
%     {\Gamma \infer k :: \Bool}
%     {\Gamma \infer k \ddagger \Bool}  [UConst]
%\\ \INFER%
%     {\Gamma \infer f~a :: \Bool
%      \qquad \Gamma \infer a :: T
%      \qquad \Gamma \infer f :: T \fun \Bool}
%     {\Gamma \infer f~a \ddagger T} [UApp]
%\end{mathpar}
