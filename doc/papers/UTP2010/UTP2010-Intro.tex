\section{Introduction}

\REVIEW3{(CLOSED)
It was not clear to me if Saoithin was a GUI front end for an
existing theorem prover, or if the authors had developed their
own core prover (which seems to me to be the case). Later in
the paper they mention that they may couple the GUI front end
with ProofPowerZ.
}

\STHN\footnote{Pronounced ``See-heen''.}
is an experimental proof assistant for the logic used in UTP,
supporting a notion of UTP theory, an intuitive prover interface
that supports the equational reasoning style that is prevalent in \cite{UTP-book}
and much other UTP-related literature,
with facilities for the user to define the syntax and semantics
of their own language constructs.
The tool consists of a core equational prover we have developed
in concert with a closely-coupled GUI interface.
At present its focus is on supporting foundational work in developing UTP,
rather than the application of the theory to a ``real-world'' design problem.


\subsection{Motivation}

We are doing foundational work in the UTP \cite{UTP-book},
which requires formal reasoning with not only predicates,
but also predicate transformers: $\RR3(P)$ $\defs$ $\Skip \cond{wait} P$
and predicates over predicates: $P = \RR3(P)$.
We also need to use recursion at the predicate level:
$ P \defs \mu Q \bullet F(Q)$,
as well as partially-defined expressions:
$s \le s \cat (tr'-tr) \equiv  tr \le tr'$.
The logic being used is therefore semi-classical
(two-valued logic, but expressions may be undefined)
and of least 2nd-order.
In addition, tool support for foundational work in UTP requires the ability
to easily describe new language constructs,
which can themselves be treated just like predicates,
in keeping with the ``programs are predicates''
philosophy \cite{conf/mlpl/Hoare85} of UTP.


\subsection{Structure}

We next discuss related work (\S\ref{sec:related}) in order to better justify
our decision to ``grow our own'' theorem proving assistant.
We then proceed to look at the logic (\S\ref{sec:logic:foundation},
type-system (\S\ref{sec:types:foundation}),
user defined language support (\S\ref{sec:language:foundation}),
proof procedures (\S\ref{sec:proof:foundation}),
and law matching (\S\ref{sec:match:foundation},
with emphasis on the underlying foundations.
We then discuss useability and experience (\S\ref{sec:usage})
and finally finish with  future work and conclusions (\S\ref{sec:conclusions}).
