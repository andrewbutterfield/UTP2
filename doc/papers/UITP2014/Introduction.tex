\section{Introduction}\label{sec:intro}

Unifying Theories of Programming (UTP) \cite{UTP-book},
is a framework that uses alphabetised predicates to define language
semantics in a relational calculus style, in a way that facilitates
the unification of otherwise disjoint semantic theories,
either by merging them, or using special linking predicates
that form a Galois connection. The framework is designed
to cover the spectrum from abstract specifications
all the way down to near-machine level descriptions,
and as a consequence the notion of refinement plays a key role.

We are doing foundational work in the UTP \cite{UTP-book},
which requires formal reasoning with not only predicates,
but also predicate transformers: $\RR3(P)$ $\defs$ $\Skip \cond{wait} P$
and predicates over predicates: $P = \RR3(P)$.
We also need to use recursion at the predicate level:
$ P \defs \mu Q \bullet F(Q)$,
as well as partially-defined expressions:
$s \le s \cat (tr'-tr) \equiv  tr \le tr'$.
The logic being used is therefore semi-classical
(two-valued logic, but expressions may be undefined)
and of least 2nd-order.
In addition, tool support for foundational work in UTP requires the ability
to easily describe new language constructs,
which can themselves be treated just like predicates,
in keeping with the ``programs are predicates''
philosophy \cite{predprog} of UTP.
In \cite{conf/utp/Butterfield10}
we gave an overview of the Unifying Theories of Programming Theorem Prover
(\UTP2)
that we are developing to support such theory development work%
\footnote{%
In that paper it was called \STHN, but the name has since changed to \UTP2
}%
.
The prover is an interactive tool, with a graphical user-interface,
designed to make it easy to define a UTP theory and to experiment
and perform the key foundational proofs.
The motivation for developing this tool,
rather than using an existing one,
has been discussed in some detail
in \cite{conf/utp/Butterfield10}, but key elements will be reprised here.
The logical and  technical underpinning was further elaborated
upon in \cite{conf/utp/Butterfield12},
which described as being an adapted and generalised version of
the equational reasoning system developed by Tourlakis \cite{journals/logcom/Tourlakis01},
itself inspired by the equational logic of David Gries and his colleagues \cite{gries.93}.

In this paper, we describe how the user interacts
with this theorem prover,
that was developed,
\emph{from the outset},
with the proof and reasoning styles typically used in UTP research and published work.

The key emphasis in development was to use window-based GUI techniques early on as the primary
mode of interaction, in stark contrast to most modern interactive theorem provers
that have essentially a command-line interface (most HOL flavours, CoQ, PVS, \ldots) sometimes wrapped with a
elaborate interface built on top of a highly configurable text editor
(e.g., Proof General on Emacs, new Isabelle/HOL interface on top of jEdit).
