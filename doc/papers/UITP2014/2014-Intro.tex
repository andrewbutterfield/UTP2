\section{Introduction}\label{sec:intro}

Unifying Theories of Programming (UTP) \cite{UTP-book},
is a framework that uses alphabetised predicates to define language
semantics in a relational calculus style, in a way that facilitates
the unification of otherwise disjoint semantic theories,
either by merging them, or using special linking predicates
that form a Galois connection. The framework is designed
to cover the spectrum from abstract specifications
all the way down to near-machine level descriptions,
and as a consequence the notion of refinement plays a key role.

In \cite{conf/utp/Butterfield10}
we gave an overview of the Unifying Theories of Programming Theorem Prover
(\UTP2)
that we are developing to support such theory development work%
\footnote{%
In that paper it was called \STHN, but the name has since changed to \UTP2
}%
.
The prover is an interactive tool, with a graphical user-interface,
designed to make it easy to define a UTP theory and to experiment
and perform the key foundational proofs.
The motivation for developing this tool,
rather than using an existing one has been discussed in some detail
in \cite{conf/utp/Butterfield10} and the  technical underpinning has been further elaborated upon in \cite{conf/utp/Butterfield12}.

Here we focus on a small subset of the logic and laws
in order to explain the key role of the extended matcher at the heart of \UTP2.
The focus is on the various matching and inference mechanisms and how they
interact to facilitate the application of laws in a proof at a level
that matches typical handwritten proof steps,
to the extent that this is feasible.

This paper assumes that the reader is familiar
with the basic ideas behind UTP, and does not give an introduction
to the subject. A good introduction is the key textbook written
by C.A.R. Hoare and He Jifeng \cite{UTP-book},
which is free to download from \texttt{unifyingtheories.org}.


\subsection{Matching: an overview}

The matching problem we consider in this paper can be summarised as follows:
during the course of a proof of some conjecture, we will have identified a sub-part
of the proof goal predicate that is of current interest---the \emph{target} focus predicate;
We will then indicate some collection of laws we hope might be applicable,
and then proceed to systematically match the target against those laws,
viewed as \emph{pattern} predicates.
The conjecture will have associated side-conditions,
as will each law---these are added to the match algorithm inputs.
A successful match returns a binding that maps all pattern variables
to the corresponding sub-parts of the target
($\pfun$ is a partial function arrow, while $\ffun$ is a finite partial function).
\[
match : (Pred\times Side) \fun (Pred\times Side) \pfun (Var \ffun \mbox{``Syntax''})
\]
Assuming some success, we can then choose from the range of successful matches,
and use this to apply the law.

In an equational reasoning system such as this, if we match a target
against a law successfully, then it is an instance of such a law,
hence universally true (this is what it means to be a law),
and so the only sensible replacement for the target is the constant predicate \True.
So, why do we return any bindings? Why not just a success/fail indicator?
There are two reasons, one algorithmic and pragmatic,
the other as a result of ensuring maximum flexibility
in the power of the prover:
\begin{enumerate}
  \item a match should fail if different sub-parts of the target
     match different sub-parts of the pattern, but with inconsistent bindings.
     So the attempt to merge these sub-bindings will uncover such problems.
  \item in an equational reasoning system, given law predicates of the
  forms $P \equiv Q$ or $P \implies Q$,
  it makes sense to try matches against either the lhs or rhs,
  with the other side viewed as a template for building the target replacement.
  The binding provides details of how the template should be instantiated.
\end{enumerate}
The process of
finding laws,
establishing if they have the special forms
just described,
 and working out the relevant replacement predicates
($\True$,or one of the law sides),
is fairly straightforward, and we do not discuss it further.

The real focus of this paper is on the function $match$ just introduced
above, and an exposition of how it is so much more than just an exercise
in structural template pattern-matching and merging the resulting bindings.

\subsection{Structure of this paper}

In Section \ref{sec:logic} we introduce a small subset of the logic of \UTP2,
and then in Section \ref{sec:variables} we discuss how the notion
of ``variable'' is neccessarily more complex than in most logic presentations.
In Sections \ref{sec:theories} and \ref{sec:contexts}
we describe how \UTP2 theories provide a context for matching,
while Section \ref{sec:matching} gives an overview of matching,
and Section \ref{sec:struct:matching} describes structural matching,
and its limitations.
The complexities of dealing with those limitation are described in Section \ref{sec:nondet:matching}.
Finally, in Sections \ref{sec:related} and \ref{sec:conclusions} we discuss related work and conclude.
