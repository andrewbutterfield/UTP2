\section{Motivation}\label{sec:motivation}

There are a lot of theorem provers in existence,
of which the most prominent feature in \cite{conf/tphol/2006provers}.
Of these, the most obvious candidates for consideration for UTP prover support
are Isabelle/HOL\cite{books/sp/NipkowPW02},
PVS\cite{conf/fmcad/Shankar96},
and CoQ \cite{bk:Coq'Art:04}.
They are powerful, well-supported,
with decades of development experience
and large active user communities.
They all support higher-order logic of some form, with a command-line interface,
typically based around tactics of some form. All three require functions to be total,
but support some kind of mechanism for handling partial functions
(e.g. dependent types in PVS).
Their reasoning frameworks are based on some form of sequent calculus,
and do not support equational reasoning in a native fashion.

There has been work done on improving the user interfaces
of theorem provers of this kind.
An interesting example was ``proof by pointing'' \cite{conf/tacs/BertotKT94}
for CoQ which allowed the user to select a subterm,
whereupon it would generate and apply a tactic based on the subterm's top-level
operator.
Whilst proof-by-pointing is not supported in more recent versions of CoQ,
it has been incorporated into ``Proof General'' \cite{conf/tacas/Aspinall00},
a general purpose user interface for theorem provers, built on top of Emacs.
It supports Isabelle and Coq, among others,
and is basically a proof-script management system.
In essence it supports the command-line tactics of the provers,
allowing the user to edit proof scripts at will,
whilst maintaining prover consistency behind the scenes.
Other explorations in this area include
INKA \cite{Hutter96}, Lovely OMEGA \cite{oai:CiteSeerXPSU:10.1.1.42.1864},
Window inference \cite{Staples95},
Generalized Rewriting in Type Theory \cite{journals/eik/Basin94},
The CoRe Calculus, \cite{conf/cade/Autexier05}
and the Jape Theorem Proving framework (\url{http://japeforall.org.uk/}.)
Of the above, \cite{conf/cade/Autexier05},\cite{journals/eik/Basin94} seems designed
to support equational reasoning, but lack any notion of a GUI. In \cite{Hutter96} and \cite{oai:CiteSeerXPSU:10.1.1.42.1864} we have GUIs, but the logic/proof style is tree based.
The window inference work \cite{Staples95} has a notion of ``focus'' similar to ours,
but has no GUI, and while capable of handling equational rewrites seems to be more general.
Jape has a GUI and facilities to encode logics, but again is deduction-biased, and has no easy 
way to extend the language.



