\section{Discussion}

Proofs done with \UTP2 are, in our opinion, more ``open'',
in that we can easily see the steps and laws used in a proof, in an equational style.
A consequence of this is readily seen when we consider the students taking the
Formal Methods course offered at Trinity College Dublin, that focusses on the
UTP, and uses \UTP2 for part of the classwork.
The feedback obtained from these students shows clearly that
(i) the learning curve to get good at \UTP2 proofs is fairly shallow---they almost never
get ``stuck'', once a few tricks are shown---experimentation is easy;
(ii) their concerns are regarding improvement to the GUI itself,
either in terms of how it looks, or having the flexibility to define their own keyboard
shortcuts.
A key feature that reduces the learning curve is the ability
of the prover to suggest possible next steps, by doing advance pattern-matching
and instantiation.

Proofs in CoQ or Isabelle/HOL are, again in this authors words, more ``procedural'',
and ``opaque'', but definitely more powerful. The disadvantage is that the learning
curve is much steeper, particularly when early success it obtained by tactics like
\texttt{auto}, \texttt{simp} or \texttt{sledgehammer}.
When these fail, the best approach is not so clear to the beginner.
However, there is undeniable power once that learning curve has been climbed.

\UTP2 was really developed to assist in the development of new semantic theories
within the UTP framework. Others have also put effort into doing this for UTP
using both ProofPowerZ\cite{conf/utp/ZC08}
and Isabelle/HOL\cite{conf/vstte/FeliachiGW12,conf/utp/FZW14}.
The price they pay is having to recast material in the ProofPower/HL style.
The benefit they tap into is the power of their proof engines.

The key questions raised here are:
\begin{itemize}
  \item Should point-n-click GUIs be added to existing provers?
  \item To what extent are front-ends like Proof-General or jEdit
  are step in this direction?
  \item
    Should more attention be paid to developing equational reasoning approaches?
  \item
  Can the \UTP2 front-end be fruitfully turned into a wrapper around Isabelle/HOL say?
  \item
  Should it use Isabelle/HOL as a way to check its proofs
  (would save trying to develop a small safe LCF-style kernel for \UTP2) ?
  \item
   Can we envisage proofs been done using gestures on a tablet?
\end{itemize}
Very recent work, presented as Tutorial 2 at FM2014 in Singapore, by Jim Woodcock, Simon Foster
and Frank Zeyda of the University of York, showed an encoding of UTP and some key theories
into Isabelle/HOL. One the negative side, they had to employ further nested quotation schemes,
but on the positive side, they used Isar in such a way that it may be relatively easy to use
Isabelle/HOL to check proof steps made by \UTP2. We hope to explore this connection in the near future.

\subsection{Obtaining Code}

\UTP2 is written in Haskell using the wxHaskell GUI library,
and is available open-source, currently under a GPL v2 license,
from \url{https://bitbucket.org/andrewbutterfield/saoithin}.
The screenshots in this paper were produced using version 0.98a.
