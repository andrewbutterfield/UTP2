\section{Related Work}
\label{sec:related}

There are a lot of theorem provers in existence,
of which the most prominent feature in \cite{conf/tphol/2006provers}.
Of these, the most obvious candidates for consideration for UTP prover support
are Isabelle/HOL\cite{books/sp/NipkowPW02},
PVS\cite{conf/fmcad/Shankar96},
and Coq \cite{bk:Coq'Art:04}.
They are powerful, well-supported,
with decades of development experience
and large active user communities.
They all support higher-order logic of some form, with a command-line interface,
typically based around tactics of some form. All three require functions to be total,
but support some kind of mechanism for handling partial functions
(e.g. dependent types in PVS).
Their reasoning frameworks are based on some form of sequent calculus,
and do not support equational reasoning in a native fashion.

There has been work done on improving the user interfaces
of the above theorem provers.
An interesting example was ``proof by pointing'' \cite{conf/tacs/BertotKT94}
for CoQ which allowed the user to select a subterm,
whereupon it would generate and apply a tactic based on the subterm's top-level
operator.
Whilst proof-by-pointing is not supported in more recent versions of CoQ,
it has been incorporated into ``Proof General'' \cite{conf/tacas/Aspinall00},
a general purpose user interface for theorem provers, built on top of Emacs.
It supports Isabelle and Coq, among others,
and is basically a proof-script management system.
In essence it supports the command-line tactics of the provers,
allowing the user to edit proof scripts at will,
whilst maintaining prover consistency behind the scenes.

Within the UTP community,
there has been considerable work using Proof{\-}Power-Z to
build models of UTP theories in Z in order to mechanise proofs.
Early work looked at deep embedding into Z of an imperative language whose semantics
were given using UTP \cite{conf/utp/NukaW06}.
Work extending this to a mechanisation of UTP itself was also undertaken,
driven by a desire to mechanically verify the semantics of Circus \cite{journals/fac/OliveiraCW09}.
Recent work has looked at re-working the mechanisation of UTP in order to better support
the hierarchical nature of UTP theory building \cite{journals/entcs/ZeydaC09},
where alphabetised predicates are restricted to relations, then designs, and so on.
Some support for Z-like theories from UTP (such as Circus)
can be found as extensions implemented in the Community Z Tools project
\cite{conf/zum/MalikU05}.


Whilst all of good pedigree, CoQ, Isabelle/HOL, ProofPower-Z and PVS
all have in common that they work best when used in the manner
for which they were designed%
---in none of these cases does this manner match the way
we wish to work in UTP, as described in the introduction.
Ironically, the key inspiration for the design of \STHN\ came not from the
above provers, but instead from the one provided as part
of the RAISE Development Method \cite{bk:Raise-dev-mthd:95}.
That theorem prover had mechanisms for selecting sub-expressions
and identifying applicable laws for that sub-expression,
a feature very close to that required for the proof style that \STHN\ supports.
