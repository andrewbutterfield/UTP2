\section{Soundness}\label{sec:soundness}

\PLAN{What basis does anyone have for trusting the output of \UTP2?}

Is \UTP2\ sound?
For now, the simple answer is no,
due mainly to two reasons.

Firstly, users can add their own laws (axioms),
and this always leads to the risk of defining a theory that is inconsistent.
As we consider the typical user to be a UTP practitioner with experience in logic
and axiomatics, developing foundational theories,
we feel it is reasonable to expect such (power) users
to be able to use their judgement to avoid such pitfalls.
Having said that, it will probably make sense in future versions
of the tool to support users at different levels of experience,
with the more advanced and dangerous features disabled for novices.

Secondly, the underlying proof engine is very complex,
reflecting the complexity of the logic required.
At present we are not in a position to guarantee soundness of every action
that can be invoked. However, in mitigation, we do point out that
the outcome of each basic proof step is highly visible in the tool's GUI.
It is clear that eventually we will have to pay serious attention
to ensuring the prover is sound (modulo any inconsistencies introduced
via user-defined axioms). We envisage two possible approaches:
\begin{enumerate}
  \item
    Identifying a very small core from which the whole logic can be developed
    conservatively, and producing a small piece of prover kernel code
    that can then be verified. This is the LCF approach adopted for prover systems
    like HOL\cite{books/sp/NipkowPW02} and Coq\cite{tr:Coq:manual:08}.
  \item
    Developing an encoding of the \UTP2\ logic into the logic of a system
    with a verified kernel, such as HOL or CoQ, and using those systems
    to do automated proof checks, possibly even for each proof step as it is
    done.
\end{enumerate}
