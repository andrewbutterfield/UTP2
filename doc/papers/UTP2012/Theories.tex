\section{Theories}\label{sec:theories}

\PLAN{Where we talk about the similarities and differences between UTP theories
and \UTP2\ ``Theory''s.}

A UTP theory is a coherent collection of the following items:
an alphabet defining the observations that can be made;
a set of healthiness conditions that characterise predicates that describe
realistic/feasible systems;
a signature that defines the abstract syntax of the language being defined;
definitions of the language constructs as healthy predicates;
and
laws that relate the behaviours of the various language components.
In \UTP2\ we use the term ``\texttt{Theory}'' to refer to
such collections, along with various other pieces of ancillary information,
as well as subsets of a full theory.
The ancillary information includes components to support
language parsing, local and temporary definitions,
as well as proof support in the form of conjectures, theorems and laws.
In effect a UTP theory may be constructed in \UTP2\ as
a layering of \texttt{Theory} ``slices'', each looking at a small part of the whole.

As an example, consider Figure \ref{fig:hier-of-theory}.
\begin{figure}[tb]
  \centering
    \begin{tikzpicture}[>=stealth,->,shorten >=2pt,looseness=.5,auto]
     \matrix[matrix of nodes,
             nodes={rectangle,draw,minimum size=4mm},
             column sep={1cm,between origins},
             row sep={1cm,between origins}]
      {              & |(X)|\texttt{Circus}  \\
        |(W)|\texttt{While} &              & |(C)|\texttt{CSP}                 \\
                     & |(D)|\texttt{Design} &              & |(R)|\texttt{Reactive}  \\
                     &              & |(u)|\texttt{UTP}              \\
                     &              & |(b)|\texttt{Base}                   \\
                     &              & |(r)|\texttt{\_ROOT}                 \\
      };
      \draw(X) -- (W) ; \draw (X) -- (C) ;
      \draw(W) -- (D) ; \draw (C) -- (D) ; \draw (C) -- (R) ;
      \draw (D) -- (u) ; \draw (R) -- (u) ;
      \draw (u) -- (b) ;
      \draw (b) -- (r) ;
    \end{tikzpicture}
  \caption{A Hierarchy of \texttt{Theory}s}
  \label{fig:hier-of-theory}
\end{figure}
Here we see theory slices organised as an acyclic directed graph,
where each slice inherits material from those below it.
At the bottom we have the \texttt{\_ROOT Theory} (slice),
which is hardwired in%
\footnote{
\texttt{\_ROOT} is the only thing hardwired---all other slices and their hierarchy
can be custom-built to suit the user.
}%
, and simply contains just the axioms of the underlying logic.
On top of this a full set of laws of predicate calculus are built
(by positing conjectures and proving them), as well as useful theories about equality,
and various datatypes, such as numbers, sets and sequences.
In \UTP2\ these are presented as a layer of \texttt{Theory} slices,
but here we simply imagine them all encapsulated into the \texttt{Base Theory},
that sits on top of \texttt{\_ROOT}.
On top of this we construct a slice (\texttt{UTP}) that presents the language constructs
that are common to most UTP theories,
e.g. sequential composition, and non-deterministic choice.
We then branch: a theory of Designs is implemented
by building a \texttt{Design Theory} slice on top of \texttt{UTP},
and hence incorporating \texttt{Base} and \texttt{\_ROOT}.
Similarly, we can define, independently of \texttt{Design},
a \texttt{Reactive Theory} slice over \texttt{UTP}.
It is this ability to re-use common material that motivates
the splitting of UTP theories into \UTP2\ Theory slices.
Figure \ref{fig:hier-of-theory} also shows how further slices
allow us to build a theory of a simple imperative language (\texttt{While}) on top of \texttt{Design},
as well as fusing \texttt{Design} and \texttt{Reactive} to get \texttt{CSP} \cite{Roscoe1997}.
A similar fusing of \texttt{While} with \texttt{CSP} gives \texttt{Circus}\cite{journals/fac/OliveiraCW09}.

In the sequel we shall no longer distinguish between ``proper'' UTP theories
and \UTP2's \texttt{Theory} slices, simply referring to them all as ``theories''.
We do not give details of the contents of a theory here,
but instead elucidate these details as we go through the paper.
