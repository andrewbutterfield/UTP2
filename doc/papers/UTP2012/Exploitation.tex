\section{Exploitation}\label{sec:exploitation}

\PLAN{Having a nice UTP theory of something gives intellectual satisfaction,
but how do we turn it into something useful?}

Assuming that we have addressed the soundness of the implementation of \UTP2,
and have used it to develop a nice theory of an interesting language,
how useful will the results be if we try to apply them to a real problem?
In principle, we could use \UTP2\ to prove properties of a program
written in the language described by our theory.
In fact some work has already been done exploring
a feature that allows us to take a predicate-transformer theory
(e.g. weakest precondition, as per \cite[Chp. 2, p66]{UTP-book}),
and a program,
and automatically generate proof obligations.
However, \UTP2\ is an interactive proof assistant, designed
to support UTP theory development, rather than theory use.
In practise, there is no way that \UTP2\ can realistically compete
with existing industrial-strength tools that can both generate
and discharge such proof obligations with a high degree of efficiency.

However what does seem to be feasible, is to develop a facility
whereby a UTP theory, once complete,
can be translated and exported as a theory useable by just such industrial-strength
provers. We are currently exploring building such a theorem-prover link
to
HOL, as recent work has looked at encoding UTP
in ProofPower/HOL\cite{conf/utp/OliveiraCW06,conf/utp/ZC08},
or Isabelle/HOL \cite{conf/utp/FeliachiGW10,conf/vstte/FeliachiGW12}.
We hope to be able to make use of these results
to build such a \UTP2-to-HOL bridge.
