\section{Introduction}\label{sec:intro}

\PLAN{An introduction to the paper as a rolling case-study.}

Unifying Theories of Programming (UTP) \cite{UTP-book},
is a framework that uses alphabetised predicates to define language
semantics in a relational calculus style, in a way that facilitates
the unification of otherwise disjoint semantic theories,
either by merging them, or using special linking predicates
that form a Galois connection. The framework is designed
to cover the spectrum from abstract specifications
all the way down to near-machine level descriptions,
and as a consequence the notion of refinement plays a key role.

Typically the development of a UTP theory involves determining the
key observational variables, so fixing the alphabet,
then defining healthiness conditions to characterise the predicates
that describe feasible behaviour, introducing the language under study
as a signature, and giving meaning to that signature using healthy predicates.
Algebraic laws of the language can then be developed.

In \cite{conf/utp/Butterfield10}
we gave an overview of the Unifying Theories of Programming Theorem Prover
(\UTP2)
that we are developing to support such theory development work%
\footnote{%
In that paper it was called \STHN, but the name has since changed to \UTP2
}%
.
The prover is an interactive tool, with a graphical user-interface,
designed to make it easy to define a UTP theory and to experiment
and perform the key foundational proofs.
The motivation for developing this tool,
rather than using an existing one has been discussed in some detail
in \cite{conf/utp/Butterfield10}.
We do not repeat it here in the introduction,
but this paper effectively gives a technical underpinning to that motivation.
In this paper we describe the logic behind \UTP2,
starting from 1st-order equational logic \cite{journals/logcom/Tourlakis01},
and gradually exposing the extensions required to facilitate the kind
of reasoning we require for foundational work.
In effect this paper explores the proof infrastructure needed
to reason about a theory of a simple imperative language ($While$),
built upon a theory of ``Designs'',
itself layered on top of a generic UTP base theory.
We start at the bottom looking at the logic and work up until
we can see what is needed for the $While$ language.

This paper assumes that the reader is familiar
with the basic ideas behind UTP, and does not give an introduction
to the subject. A good introduction is the key textbook written
by C.A.R. Hoare and He Jifeng \cite{UTP-book},
which is free to download from \texttt{unifyingtheories.org}.

In the rest of this paper, we use the term ``user'' to
refer to a UTP practitioner  involved in the development
of new UTP theories, and not a software developer
who might want to employ a formal method whose underlying semantics
derive from UTP.

\subsection{Structure of this paper}

Section \ref{sec:theories} talks about theories,
and gives a visual outline of much of this paper
in Figure \ref{fig:hier-of-theory}.
Section \ref{sec:logic} introduces the logic of \UTP2,
and
Section \ref{sec:utp} gives us an introduction to definitions
common to most theories.
In
Section \ref{sec:designs},
and
Section \ref{sec:programs},
we describe how \texttt{Theory}s can be layered up to present
a UTP Theory of Designs, as well as a theory for a simple While programming
language built as an extension on top of Designs.
Section \ref{sec:soundness}
and
Section \ref{sec:exploitation} discuss issues to do with the trustworthiness
and usefulness of \UTP2,
and finally,
Section \ref{sec:conclusions} concludes.
A collection of relevant rules can be found in
Appendix \ref{sec:rules}.
