\section{Conclusions}\label{sec:conclusions}

\PLAN{Wrap-up the paper.}

We can, in effect,
summarise the paper
 by giving a requirements list summarising all the special logic features we desire
 for \UTP2:
predicate and expression meta-variables;
user language definitions;
quantifier list variables, with specials to identify alphabets;
explicit substitutions;
``semantic'' side-conditions;
and
predicate transformers.

All the above could be implemented using Isabelle, or CoQ,
or PVS, or pretty much any higher-order theorem prover.
However any algorithm can, in principle, be written in the pure lambda
calculus, or expressed as a Turing machine,
but this does not make it
feasible, desirable or practical to use those notations.
Similarly we feel that encoding our requirements into one of the above higher-order
systems, at least to the extent that it would be visible to the user,
is not the way to meet our requirement for machine-assisted support
for UTP foundational reasoning.

The resulting logic is quite large, and space limitations have prevented
us from giving a complete description here.
More details can be found in a draft of the \UTP2\ Reference Manual
\cite{UTP2:Reference}.
