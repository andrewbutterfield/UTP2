\section{Programs}\label{sec:programs}

\PLAN{We now use Designs to develop a simple imperative language,
where our need to handle explicit substitutions becomes even more apparent.}

To get concrete, we are now going to define the semantics
for a simple imperative programming language (a.k.a. $While$),
as a UTP Design.
To keep things simple for now,
we assume the language has exactly three program variables:
 \texttt{x}, \texttt{y}, and \texttt{z}
(we look at the issue of many variables below in Section \ref{ssec:skip-n-assign}).
\RLEQNS{
   u,w \in While &::=& Skip & \mbox{do nothing}
\\              &|& v:=e & \mbox{Assignment, } v \in  \setof{x,y,z}, \fv.e \subseteq \setof{x,y,z}
\\              &|& u \comp w & \mbox{Sequential Composition}
\\              &|& u \cond c w & \mbox{Conditional, } \fv.c \subseteq \setof{x,y,z}
\\              &|& c \circledast w & \mbox{While-loop, } \fv.c \subseteq \setof{x,y,z}
}
The alphabet of this theory now contains $x,y,z,x',y',z'$
in addition to $ok,ok'$ inherited from the Design theory.
Also inherited are the definitions of $\comp$ and $\cond c$,
where now $Obs$ can bind to $ok,x,y,z,ok',x',y',z'$
in pattern matching.
We can use the language specification facility to introduce the syntax
to \UTP2, so in $While.lang$ we have:
\begin{eqnarray*}
   Skip & \mapsto & \texttt{Skip}
\\ :=  & \mapsto & \texttt{V := E}
\\ whl & \mapsto & \texttt{E ** P}
\end{eqnarray*}

\subsection{The \UTP2\ semantics of $Skip$ and $x:=e$}
\label{ssec:skip-n-assign}

We start to define the semantics of $Skip$,
and we could immediately write:
\begin{eqnarray*}
  Skip &\defs& True \design x'=x \land y'=y \land z'=z
\end{eqnarray*}
While correct, we may worry about what happens if the number of variables
increases, or if we want to have some dynamism regarding the number and
names of program variables. While we discuss another possible approach
to program variables later, for now let's see what we can do to improve things.
We could try to use special list variable $Obs$,
to get
\begin{eqnarray*}
  Skip &\defs?& True \design Obs' = Obs
\end{eqnarray*}
but this is not satisfactory, as $Obs$ ($Obs'$) includes $ok$ ($ok'$)
and these cannot occur in the design predicates, as per the side-condition
used in the Design theory.

The solution here is realise that in many UTP theories
we actually have two classes of observations: those associated with
the values of variables in the program text under consideration (here $x$, $y$ and $z$),
and those that capture overall program properties, independent of any program variable
(here $ok$ and $ok'$, denoting termination).
We shall refer to the former as script variables and the latter as model variables,
and add in two new special list-variables called $Scr$ and $Mdl$
to match against the two classes.
So in this theory, $Scr$ can match $x,y,z$, while $Mdl$ matches $ok$.
Also $Obs$ can now match $Scr,Mdl$, or combinations such as $Scr,ok$.
This requires us to modify the $obs$ table in a theory slightly
as we must now record observation class, as well as its type:
\begin{eqnarray*}
Theory &=& \textbf{record} \ldots
\\ && obs : Name \pfun Type \times OClass
\\ && \ldots \textbf{end}
\\ OClass &::=& Model | Script
\end{eqnarray*}
So, for example, in theory \texttt{Design} we have $obs(ok) = (\Bool,Model)$,
while in theory \texttt{While} we have $obs(x) = (t,Script)$, where $t$ is some type.
We can now define the semantics of $Skip$ as:
\begin{eqnarray*}
  Skip &\defs& True \design Scr'=Scr
\end{eqnarray*}
This definition will now work in a range of theories,
provided the observations are classified appropriately.
However it does also require a further extension
of the law matching algorithm.
This has to be modified to allow a pattern like $Scr'=Scr$,
given bindings $Scr \mapsto x,y,z$ and $Scr' \mapsto x',y',z'$,
to match against a predicate fragment like $x'=x \land y'=y \land z'=z$.
This feature is quite easily implemented as part of the structural matcher.

We now turn our attention to the definition of assignment.
The following is \emph{not} satisfactory:
\begin{eqnarray*}
  x:=e &\defs?& True \design x' = e \land Scr'=Scr
\end{eqnarray*}
First, as $x$ is known, this rule will only match assignments whose
variable is $x$, so we would need a different definition for each program
variable---not a good idea!
Secondly, $Scr'=Scr$ will match $x'=x \land y'=y \land z'=z$
as already described, and so we can match $x'=e \land x'=x$
which reduces to $x=e$, and then probably $False$.
We could try to make the matching of $Scr'=Scr$ against
$x'=x \land y'=y \land z'=z$ ``context sensitive'',
only matching an equality if both sides do not appear ``elsewhere'',
but it is currently very unclear if this is at all feasible.
Instead, we extend the list-variable notation to allow modifiers,
so we can write the following satisfactory definition for assignment:
\begin{eqnarray*}
  v:=e &\defs& True \design v' = e \land (Scr'\setminus v')=(Scr\setminus v)
\end{eqnarray*}
The law/pattern variable $v$ is not known, so it will match any of $x$, $y$ or $z$,
and even $ok$.
However as $ok$ cannot appear in the predicates in a design,
any matching of $v$ to $ok$ will lead to a proof that eventually freezes
up because the side-condition defining $\design$ won't be satisfiable.
Imagine we are matching the righthand side of the above definition
with $y'=f \land x'=x \land z'=z$.
The matching algorithm will attempt match $y'$ against $v'$,
returning a binding $v' \mapsto y'$.
This binding gives us enough information
to be able to match
$(Scr'\setminus v')=(Scr\setminus v)$
against $x'=x \land z'=z$.

A further complication arises when we try to prove laws such
as:
\begin{eqnarray*}
   (v:=e \comp v:=f ) &\equiv&( v := f[e/v])
\\ (u:=e \comp v:=f) &\equiv& (v := f[e/u] \comp u := e),
    \quad v \notin \fv.e
\end{eqnarray*}
We will not elaborate on details here,
but we find the need to use special list variables like $Scr$ and $Scr'$
in substitutions, so the matching algorithm needs to handle those cases as well.

\subsection{Merging program variables}

Another way to handle program variables
is to group them together into an environment,
a mapping from variable names to values:
\begin{eqnarray*}
% \nonumber to remove numbering (before each equation)
  \rho \in Env &=& Var \pfun Val
\end{eqnarray*}
We can then introduce $Model$ variables
called $state$ and $state'$.
This simplifies the alphabet handling, as it is now fixed,
and we can model variable declarations with map extensions.
In effect we have no script variables, just model ones,
with the consequence that the theory of the alphabet is now
independent of the program script.
The added complexity now emerges in the type system,
because $Val$ needs to include all types in $Type$,
and the definition of assignment now requires an $eval$
function of type $Env \fun Expr \pfun Val$
(here $\oplus$ denotes map override):
\begin{eqnarray*}
  v:=e &\defs& True \design state'=state \oplus \setof{v \mapsto eval(state)(e)}
\end{eqnarray*}
\UTP2\ can support either style of program variable handling,
although the environment-based approach requires a theory
of finite maps, and laws defining $eval$ for every expression construct,
with an added complication of having to handle explicit expression \emph{syntax} in laws.
However, the provision of such an $eval$ function is not quite as onerous as it sounds
as laws providing the meaning of all expression constructs
are required in any case.


%\subsection{The \UTP2\ semantics of $c \circledast w$}

We are not going to elaborate too much on how to give
a semantics to the while-loop construct here,
apart from noting that it requires a fixpoint construct
in the logic syntax, and an appropriate axiomatisation of fixpoint theory.
Then the loop can be defined as the least fixed point of
the appropriate functional.
\begin{eqnarray*}
   p,q,r \in Pred &::=& \ldots
\\              &|& \mu P @ F(P) \qquad \mbox{Fixpoint Operator}
\\ c \circledast w &\defs& \mu W @ (w \comp W) \cond c Skip
\end{eqnarray*}
