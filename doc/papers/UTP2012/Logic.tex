\section{Logic}\label{sec:logic}

\PLAN{Describing the logic of \UTP2,
and why it isn't just plain old predicate calculus (first order).}

The logic of \UTP2\ is an adaptation of the first-order equational logic
described by Tourlakis \cite{journals/logcom/Tourlakis01},
that fully formalises the logic of Dijkstra, Gries and Schneider \cite{gries.93}.
\subsection{\UTP2\ Logic Syntax}

We define our logic syntax
over a collection of given sets characterising different name-spaces:
\RLEQNS{
   x,y,z \in Var && (given) & \mbox{Obs. Variables}
\\ k \in Const && (given) & \mbox{Constants}
\\ f,g,h \in Name && (given) & \mbox{(Function) Names}
\\ E,F,G \in EName && (given) & \mbox{Expression Metavariable Names}
\\ P,Q,R \in PName && (given) & \mbox{Predicate Metavariable Names}
}
Variables, constants and function names are as one would expect
in a logic with associated equational theories,
but we also have explicit meta-variables for expressions
and predicates, in the object logic, as many UTP laws
are expressed using such.

Expressions and Predicates are defined by mutual induction,
because both may contain instances of the other.
Expressions denote values in the ``world of discourse'' (observations)
and are typed. Expressions whose type is boolean ($c \in Expr$)
form the class
of \emph{atomic predicates}:
\RLEQNS{
   c,e \in Expr &::=& k | x  & \mbox{Expressions}
\\              &|& f~e & \mbox{Applications}
\\              &|& \lambda x @ e & \mbox{Obs. Abstraction}
\\              &|& \Lambda E @ e & \mbox{E-var. Abstraction}
\\              &|& \Lambda P @ e & \mbox{P-var. Abstraction}
\\              &|& \setof{ x | p @ e } & \mbox{Comprehension}
\\              &|& E & \mbox{Explicit Metavariable}
%\\              &|& e[ e/x ] & \mbox{Explicit Obs. Substitution}
%\\              &|& e[ e/E ] & \mbox{Explicit E-var. Substitution}
%\\              &|& e[ p/P ] & \mbox{Explicit P-var. Substitution}
}
Predicates are defined much as expected:
\RLEQNS{
   p,q,r \in Pred &::=& True | False & \mbox{Constant Predicates}
\\              &|& e & \mbox{Atomic Predicate (Boolean-valued Expr.)}
\\              &|& \lnot p & \mbox{Negation}
\\              &|& p~ \maltese q & \mbox{Composites, }\maltese \in \setof{\land,\lor,\implies,\equiv}
\\              &|& P & \mbox{Explicit Metavariable}
\\              &|& \yen x @ p & \mbox{1st-order Quantifiers, } \yen \in \setof{\forall,\exists,\exists!}
\\              &|& \yen P @ p & \mbox{higher-order Quantifiers, } \yen \in \setof{\mathbf\forall,\mathbf\exists}
\\              &|& \yen E @ p & \mbox{higher-order Quantifiers, } \yen \in \setof{\mathbf\forall,\mathbf\exists}
\\              &|& [ p ] & \mbox{Universal Closure (over observations)}
%\\              &|& p[ e/x ] & \mbox{Explicit Obs. Substitution}
%\\              &|& p[ e/E ] & \mbox{Explicit E-var. Substitution}
%\\              &|& p[ p/P ] & \mbox{Explicit P-var. Substitution}
}
%One point to note is that the syntax for both expressions
%and predicates include explicit substitutions,
%as these feature prominently in many UTP theories
%(e.g. laws regarding assignment and sequential composition,
%of the reactive healthiness condition \RR2).

The axioms of the logic are shown in Appendix \ref{sec:rules}
(\ref{ssec:prop-axioms}, \ref{ssec:non-prop-axioms}).
The axioms are stored in the hardwired \texttt{\_ROOT} theory,
in the \emph{laws} component of the theory,
which maps law-names to laws,
where a law is a predicate and a side-condition.
Side-conditions are a conjunction of zero or more basic conditions,
which typically capture relationships between given variables and
the free variables ($\fv$) of given predicates.
\begin{eqnarray*}
Theory &=& \textbf{record}
\\ && laws : Name \pfun Law
\\   && \ldots \textbf{end}
\\ Law &=& Pred \times Side
\\ Side &=&  x \notin \fv.P | \setof{x,y,\ldots} = \fv.P | \setof{x,y,\ldots} \supseteq \fv.P | \ldots
\end{eqnarray*}
Here the notation $A \pfun B$ denotes a partial finite function from $A$
to $B$, and so is effectively a table using a key of type $A$ to lookup
a value of type $B$.

The inference rules (\ref{ssec:inference})
are implemented, in the main, by a pattern matching mechanism
that takes a current proof goal and sees which laws can apply,
and a process that allows the user to select and apply the desired one,
storing the changed goal in a list that is assumed to be chained
together by logical equivalence.
The basic structural match has a judgement $\Gamma\vdash P \ddagger T | \beta$
that asserts that,
given matching environment $\Gamma$,
test predicate $T$ matches pattern predicate $P$, with resulting bindings $\beta$.
Bindings map variables to well-formed expressions or predicates,
as appropriate.
If we ignore $\Gamma$ for now, then a representative collection
of structural matching rules are:
\begin{eqnarray*}
   & \inferrule%
      {}%
      {\Gamma \vdash x \ddagger e |\setof{ x \mapsto e} }
   & \LNAME{match-var}
\\
\\ & \inferrule%
     {\Gamma\vdash P_i \ddagger T_i  | \beta_i
       \qquad
       \beta_1 \cong \beta_2}%
     {\Gamma \vdash P_1 \land P_2 \ddagger T_1 \land T_2 | \beta_1 \uplus \beta_2}
   & \LNAME{match-$\land$}
\\
\\ & \inferrule%
        {P \ddagger Q  | \beta_1
          \qquad xs \ddagger ys | \beta_2
          \qquad \beta_1 \cong \beta_2}%
        {\forall xs @ P \ddagger \forall ys @ P | \beta_1 \uplus \beta_2}
   & \LNAME{match-$\forall$}
\end{eqnarray*}
The $\cong$ predicate asserts that two bindings
do not map the same variable to different things.
The $\uplus$ operator merges two bindings, provided they satisfy $\cong$.
An attempted match of $T$ against $P$ fails if no rules apply,
or an attempt is made to apply $\uplus$ to two bindings that do not satisfy $\cong$.

In order to facilitate proof, the theory has two components, one
for conjectures, which can be viewed as aspirant laws
(posited, hopefully true, but not yet proven),
and theorems, which are conjectures with proofs:
\begin{eqnarray*}
Theory &=& \textbf{record} \ldots
\\ && conjs : Name \pfun Law
\\ && thms : Name \pfun Proof
\\ && \ldots \textbf{end}
\\ Proof &=& \textbf{record}
\\ && goal : Pred
\\ && sc : Side
\\ && done : \Bool
\\ && \ldots \textbf{end}
\end{eqnarray*}
The workflow is as follows: conjectures can be entered by the user
and accumulated in $conjs$. A proof can then be started by selecting
a conjecture, which creates a corresponding entry in $thms$,
with $goal,sc$ set to match the conjectured law, and the $done$ flag
set to false. More than one proof can be active at any one time.
A proof is carried out using all the $laws$ accessible from
the theory.
Once a proof is complete, the $done$ flag is set true,
the corresponding conjecture is deleted, and, usually,
a corresponding entry is made into $laws$.

The mechanism as described so far is adequate for proving all and any conjectures
based on propositional logic.
However it needs extensions to cater for non-propositional logic,
and the datatype theories.
We will address the non-propositional extensions in the next section
on generic UTP.
Here we discuss briefly some practical issues with datatype theories.
We can define a theory of natural number arithmetic using Peano axioms,
for example---the tool supports the creation of a new named empty
theory, and the addition of appropriate axioms%
%\footnote{
%Exercise: you need five axioms --- what are they?
%}
 by the user into the $laws$ table.
Operations on natural numbers can be defined axiomatically by adding further
laws as required.
From this it is possible to prove a range of theorems about natural number
operations, e.g. $m + 0 = m$.
A similar exercise can be done for sets, and sequences,
resulting in laws like $S \cup \emptyset = S$ and $s \cat \nil = s$.
The problem is that we do not just match against whole laws, but can also
match against just the lefthand or righthand sides of an equality or
equivalence---so the righthand sides of all three laws above will match
an arbitrary expression $e$, offering $e+0$, $e \cup \emptyset$ and $e \cat \nil$
as replacements.
To prevent such spurious matches, we introduce a type system for expressions,
and a type-inference engine, that uses context information to deduce the types
of expressions like $e$, and serves to reduce spurious matches to a considerable degree.
A theory contains tables to support this feature:
\begin{eqnarray*}
Theory &=& \textbf{record} \ldots
\\ && type : Name \pfun Type
\\ && \ldots \textbf{end}
\\ t \in Type &::=& \Bool | \Int | \tau | \power t | \ldots
\end{eqnarray*}
The $Name$s in $type$ are typically names of variables or functions.
