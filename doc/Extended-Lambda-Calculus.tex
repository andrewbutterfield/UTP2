We present the extended $\lambda$-calculus that is a model for our logic:
$$\begin{array}{rclr}
   v,w,x &\in& V & \mbox{Standard Variables, given.}
\\ E &\in& M & \mbox{Expression Meta-Variables, given.}
\\ \lst q,\lst r &\in& Q & \mbox{List Variables, given.}
\\ e,f,g \in L
   &::=& v  & \mbox{Variables}
\\ &|& e~e  & \mbox{Application}
\\ &|& E & \mbox{Expression Metavariables}
\\ &|& \lambda \vec v,\lstvec q @ e & \mbox{Extended $\lambda$ Abstraction} 
\\ &|& e[\vec f,\lstvec r/\vec v,\lstvec q] & \mbox{Explicit Substitution}
\end{array}$$
The variable lists $\vec v,\lstvec q$ are well-formed when both have no duplicates.
Also note that the $v_i$ and $\lst q_j$ can be mixed up in practise,
but that listing standard- before list-variables is the canonical way
of representing them here.
In explicit substitutions, $\vec f$ and $\vec v$ must have the same length,
as so must both $\lstvec r$ and $\lstvec q$.

With such lists we can associate an interpretation $I_{\lambda}$ that maps list-variables
to sequences of standard variables:
\begin{eqnarray*}
   I_{\lambda} &:& Q \pfun V^*
\end{eqnarray*}
The interpretation $I_{\lambda}$ is consistent 
with list $\vec v,\lstvec q$ if no element of $\vec v$ appears in
the interpretation of any element of $\lstvec q$,
and those interpretation sets are all mutually disjoint.
\begin{eqnarray*}
  i &\in& 1 \ldots \len~ \vec v
\\ j,k &\in& 1 \ldots \len~ \lstvec q
\\ v_i &\notin& I_{\lambda}(\lst q_j)
\\ j\neq k &\implies& I_{\lambda}(\lst q_j) \cap I_{\lambda}(\lst q_k) = \emptyset
\end{eqnarray*}

We also note, in (explicit) substitutions,
that there should be no duplicates in $\lstvec r$,
which are list-variables, each denoting a list of zero or more
substitution replacement expressions.
There is also an interpretation that maps these instances of list variables to 
sequences of expressions:
\begin{eqnarray*}
   I_{\sigma} &:& Q \pfun L^*
\end{eqnarray*}
Given two interpretations $(I_{\sigma},I_{\lambda})$,
we say they are consistent with respect to  an explicit substitution
$[\vec f,\lstvec r/\vec v,\lstvec q]$, if $I_{\lambda}$ is consistent with
$\vec v,\lstvec q$, and for each element of $\lstvec r$,
its interpretation under $I_{\sigma}$ has the same length as that for the 
corresponding element of $\lstvec q$.
\begin{eqnarray*}
  \len~\lstvec r &=& \len~\lstvec q
\\ i &\in& 1 \ldots \len~ \lstvec r
\\ \len(I_{\lambda}(\lst q_i))
   &=&
   \len(I_{\sigma}(\lst r_i))
\end{eqnarray*}

We now introduce meta-notation for substitution
$$
  e[\vec v,\lstvec q:=\vec f,\lstvec r]
$$
This is a syntactical operator, transforming an extended $\lambda$-calculus
expression into another one that is equivalent, w.r.t to \emph{all} interpretations
consistent with $\vec v,\lstvec q$.
As a general rule, we can say that
$$
 e[\vec v,\lstvec q:=\vec f,\lstvec r]
 =
 e[\vec f,\lstvec r/\vec v,\lstvec q]
$$
but we would like to do better than this.

We shall explore it's full definition by first considering single substitutions,
one of a standard variable, the other of a list variable.
First, the standard case:
\begin{eqnarray*}
   \_[\_:=\_] &\mbox{as}& L \fun (V \times L) \fun L
\\ v[x:=g] &\defs&  g \cond{x=v} v
\\ E[x:=g] &\defs&  E[g/x]
\\ (f~e)[x:=g] &\defs& (f[x:=g])~(e[x:=g])
\\ (\lambda \vec v,\lstvec q @ e)[x:=g]
   &\defs&
   \left\{
     \begin{array}{ll}
       (\lambda \vec v,\lstvec q @ e), & x \in \vec v \\
       (\lambda \vec v,\lstvec q @ e)[g/x], & \hbox{otherwise.}
     \end{array}
   \right.
\\ (e[\vec f/\vec v])[x:=g]
   &\defs&
   \left\{
     \begin{array}{ll}
       e[\vec f[g/x]/\vec v], & x \in \vec v \\
       e[g,\vec f[g/x]/x,\vec v], & \hbox{otherwise.}
     \end{array}
   \right.
\\ (e[\vec f,\lstvec r/\vec v,\lstvec q])[x:=g]
   &\defs&
   (e[\vec f,\lstvec r/\vec v,\lstvec q])[g/x]
\\&\textbf{or?}&
   \left\{
     \begin{array}{ll}
       e[\vec f[g/x],\vec{\lst r}[g/x]/\vec v,\lstvec q], & x \in \vec v \\
       e[\vec f[g/x],\vec{\lst r}[g/x]/\vec v,\lstvec q], & x \notin e \\
       (e[\vec f,\lstvec r/\vec v,\lstvec q])[g/x], & \hbox{otherwise.}
     \end{array}
   \right.
\end{eqnarray*}
The last alternative requires a further extension to the syntax to
allow (nested) substitutions in the $\lstvec r$ list. Not sure we gain much.

Next, the list case:
\begin{eqnarray*}
   \_[\_:=\_] &\mbox{as}& L \fun (Q \times L^*) \fun L
\\ v[\lst p:=\vec g] &\defs&  v[\vec g/\lst p]
\\ E[\lst p:=\vec g] &\defs&  E[\vec g/\lst p]
\\ (f~e)[\lst p:=\vec g] &\defs& (f[\lst p:=\vec g])~(e[\lst p:=\vec g])
\\ (\lambda \vec v,\lstvec q @ e)[\lst p:=\vec g]
   &\defs&
   \left\{
     \begin{array}{ll}
       (\lambda \vec v,\lstvec q @ e), & \lst p \in \lstvec q \\
       (\lambda \vec v,\lstvec q @ e)[\vec g/\lst p], & \hbox{otherwise.}
     \end{array}
   \right.
\\ (e[\vec f/\vec v])[\lst p:=\vec g]
   &\defs&
   (e[\vec f/\vec v])[\vec g/\lst p]
\\ (e[\vec f,\lstvec r/\vec v,\lstvec q])[\lst p:=\vec g]
   &\defs&
   (e[\vec f,\lstvec r/\vec v,\lstvec q])[\vec g/\lst p]
\end{eqnarray*}

Now, the general case:
\begin{eqnarray*}
   \_[\_:=\_] &:& L \fun (V^m \times Q^n \times L^m \times Q^n) \fun L
\\ e[\nil:=\nil] &\defs& e, \mbox{ the empty case}
\\ v[\vec x,\lstvec p:=\vec g,\lstvec s] 
  &\defs& x_i, \qquad \textbf{if}~ v = x_i
\\ (f~e)[\vec x,\lstvec p:=\vec g,\lstvec s] 
  &\defs& (f[\vec x,\lstvec p:=\vec g,\lstvec s]~e[\vec x,\lstvec p:=\vec g,\lstvec s])
\\ (\lambda \vec v,\lstvec q @ e)[\vec x,\lstvec p:=\vec g,\lstvec s]
   &\defs&
   (\lambda \vec v,\lstvec q @ e)
   [\vec x\setminus\vec v,\lstvec p\setminus\lstvec q
    :=
    \vec g|_{\vec x\setminus\vec v},\lstvec s|_{\lstvec p\setminus\lstvec q}]
\\ e[\vec x,\lstvec p:=\vec g,\lstvec s] &\defs& e[\vec g,\lstvec s/\vec x,\lstvec p]
\end{eqnarray*}
Here the notation
$$
\vec g|_{\vec x\setminus\vec v}
$$
denotes $\vec g$ restricted to the corresponding elements of $\vec x$
that remain when the elements of $\vec v$ are removed.
