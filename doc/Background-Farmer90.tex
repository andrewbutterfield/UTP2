
\subsection{\textbf{PF} (Farmer 1990)}

This is the \textbf{PF} logic of William M. Farmer \cite{Farmer90},
with modernised notation.

\subsubsection{Formal Language of PF}

\paragraph{Types}
$$\begin{array}{rcl@{\qquad}r}
   \alpha,\beta  \in T &::=& \iota | b | \beta \fun \alpha & \iota, *, (\alpha\beta)
\\ \alpha_\iota  \in T_\iota &::=& \iota | \beta \pfun \alpha_\iota
\\ \alpha_b  \in T_b &::=& b | \beta \fun \alpha_b
\\ T &=& T_\iota \cup T_b
\end{array}$$

\paragraph{Primitive Symbols}

for all $\alpha \in T$:
$$\begin{array}{rcl@{\qquad}r}
   f_\alpha,
   g_\alpha,
   h_\alpha,
   \ldots,
   x_\alpha,
   y_\alpha,
   z_\alpha &:& \alpha & \mbox{typed variables} (V)
\\ =        &:& \alpha \fun \alpha \fun b & \mbox{logical constant} (C)
\\ c_\alpha &:& \alpha & \mbox{non-logical constants} (C)
\end{array}$$


\paragraph{Expressions and Formulas}

$$\begin{array}{rcl@{\qquad}r}
   A_\alpha, \ldots Z_\alpha
   &::=& x_\alpha | c_\alpha
\\ &|& F_{\beta \fun \alpha}A_\beta & \mbox{application}
\\ A_{\beta\fun\alpha}
   &::\!\!+& \lambda x_\beta @ M_\alpha & \lambda\mbox{-expression}
\\ A_b  &&& \mbox{formula}
\\ A_\alpha[B_1,\ldots, B_n/x_1,\ldots,x_n]
  && type(B_i)=type(x_i) & A_\alpha[{x_1 \over B_1}{\cdots\over\cdots}{x_n \over B_n}]
\end{array}$$

\subsubsection{Semantics of PF}

\begin{description}
  \item[Principle 1]
    $A_\iota$ may denote partial function
  \item[Principle 2]
    $A_b$ always denotes a standard truth-value
  \item[Principle 3]
    $x_\alpha,\;c_\alpha$ and $\lambda x_\alpha @ M_\beta$ always denote.
  \item[Principle 4]
    $F_{\beta\fun\alpha}A_\beta$ denotes iff $F$ and $A$ do ($f$ and $a$ resp.),
    and $f$ is defined at $a$.
  \item[Principle 5]
    $P_{\alpha\fun b}A_\alpha$ is false if $A$ is non-denoting.
\end{description}

\paragraph{Frames}

\def\D{\mathcal{D}}
\def\F{\textsc{f}}
\def\T{\textsc{t}}
\begin{eqnarray*}
  \mbox{frame} && \setof{ \alpha \in T | \D_\alpha }
\\ \D_b &=& \setof{ \T,\F }, \quad ( \T \neq \F )
\\ \alpha \in T_\iota &\implies& D_{\beta\fun\alpha} = \D_{\beta} \pfun \D_\alpha
\\ \alpha \in T_b &\implies& D_{\beta\fun\alpha} = \D_{\beta} \fun \D_\alpha
\end{eqnarray*}
There is no disjointness requirement between $\D_\alpha$ and $\D_\beta$,
e.g. the everywhere undefined function can be a member of all functional
domains over $T_\iota$.

blah about interpretations, etc.

\subsubsection{The Logic}

\def\TT{\mathbf{T}}
\def\FF{\mathbf{F}}

\paragraph{Abbreviations}

$$\begin{array}{l@{\mbox{~~~for~~~}}l}
    A = B & =_{\alpha\fun\alpha\fun b}A_\alpha B_\alpha
\\  \TT_b & (=_{b \fun b \fun b}) = (=_{b \fun b \fun b})
\\  \FF_b & (\lambda x_b @ \TT) = (\lambda x_b @ x_b)
\\ \FF_{\alpha\fun\beta} & \lambda x_\alpha @ \FF_\beta, \quad \beta \in T_b
\\ \Pi & =(\lambda x_\alpha \TT_b)
\\ \forall x_\alpha @ A_b & \Pi (\lambda x_\alpha @ A_b)
\\ \land & \lambda x,y @ (\lambda g @ g\;\TT\;\TT) = (\lambda g @ g\;x\;y)
\\ A \land B & \land A\;B
\\ \implies & \lambda x,y @ x = x \land y
\\ A \implies B & \implies A\;B
\\ \lnot & = \FF
\\ \lor & \lambda x,y @ \lnot(\lnot x \land \lnot y)
\\ A \lor B & \lor A\;B
\\ \exists x_\alpha @ A_b & \lnot(\forall x_\alpha @ \lnot A_b)
\\ A \neq B & \lnot(A = B)
\\ A_\downarrow & (\lambda x @ \TT) A
\\ A_\uparrow & \lnot(A\downarrow)
\\ A \simeq B & (A\downarrow \lor B\downarrow) \implies A=B
\end{array}$$

\paragraph{Axioms}

\def\LN#1{\mbox{[\textsc{#1}]}}

$$\begin{array}{rrcl@{\qquad}r}
   1& (g\;\TT \land g\;\FF) &=& \forall x @ g\;x &\LN{Truth Values}
\\ 2& x=y &\implies& h_b(x) = h_b(y) &\LN{Liebniz}
\\ 3& f=g &=& \forall x @ (f\;x \simeq g\;x) & \LN{Extensionality}
\\ 4& A\downarrow
   &\implies&
   (\lambda x @ B)A \simeq B[A/x], \quad A \mbox{ free for }x\mbox{ in }B &\LN{$\beta$-Reduction}
\\ 5& A_\alpha\downarrow && \mbox{where }\alpha \in T_b
   &\LN{Formulas Denote}
\\ 6& x_\alpha\downarrow && \mbox{where }x \in V
   &\LN{Variables Denote}
\\ 7& c_\alpha\downarrow && \mbox{where }c \in C
   &\LN{Constants Denote}
\\ 8& (\lambda x@A)\downarrow &&
   &\LN{Lambdas Denote}
\\ 9& (A_{\beta\fun\alpha}\;B)\downarrow
   &\implies
   & A\downarrow \land B\downarrow, \quad \alpha \in T_\iota
   &\LN{Propagate Divg.}
\\ 10& B\uparrow &\implies& A_{\beta\fun\alpha}\;B = \FF, \quad \alpha \in T_b
   &\LN{Divg. Predicate}
\end{array}$$

\paragraph{Inference}

Rule $R$:
$$
\INFER%
  {A_\alpha \simeq B_\alpha \\ C_b}
  {C_b[A_\alpha \mapsto B_\alpha]}
$$
Here we simply do a replace of one instance
of $A_\alpha$ (no consideration of free/bound etc),
(except we don't replace a binding occurrence !).

Proof of $A_b$ in \textbf{PF} is
$$
  A_0,A_1,\ldots,A_{n-1},A_n
$$
where $A_n$ is $A_b$ and for $i \in 0\ldots n$, $A_i$ is an axiom,
or obtained by applying rule $R$ to some of $A_0,\ldots,A_{i-1}$.

We now define
$$
 C_b[A_\alpha / B_\alpha]
 \mbox{ as }
 C_b[A_\alpha \mapsto B_\alpha]
$$
but where replaced $A_\alpha$ does not occur in $C_b$ within
a $\lambda x @ D$, where $x$ free in $\Gamma$ and $A_\alpha \simeq B_\alpha$.

Proof of $A_b$ in \textbf{PF} from $\Gamma$ ($\Gamma \infer A_b$)is:
\begin{eqnarray*}
  \mathcal{F}_1 \frown \mathcal{F}_2
\\ \mbox{where}
   && \mathcal{F}_1 \mbox{ is a proof in \textbf{PF}}
\\ &\mbox{and}& last(\mathcal{F}_2) \mbox{ is } A_b
\\ &\mbox{and}& \mbox{for each } D_b  \mbox{ in } \mathcal F_2
\\ && \qquad        D_b \in \Gamma
\\ && \quad {} \lor D_b \in \mathcal F_1
\\ && \quad {} \lor D_b \mbox{ is } C_b[A_\alpha / B_\alpha],
                    \quad C_b,(A_\alpha \simeq B_\alpha) \in \mathcal F_2
\end{eqnarray*}



Rule $R'$:
$$
\INFER%
   {\Gamma \infer A_\alpha \simeq B_\alpha \\ \Gamma \infer C_b}
   {\Gamma \infer C_b[A_\alpha / B_\alpha]}
$$

Tautology Theorem:
\INFER%
 {A_1 \land \ldots \land A_n \implies B ~ \Gamma \infer {A_i}}
 {\Gamma \infer B}

Relation $ A \approx_\Gamma B$ which holds iff $\Gamma \infer A \simeq B$,
is an equivalence relation.

Universal Generalisation:
\INFER%
  {\Gamma \infer A \\ x \notin \Gamma}
  {\Gamma \infer \forall x @ A}

Qualified Generalisation:
\INFER%
  {\Gamma \infer A \implies B \\ x \notin A,\Gamma}
  {\Gamma \infer A \implies \forall x @ B}

Universal Instantiation:
\INFER%
  {\Gamma \infer \forall x @ B
   \\
   \Gamma \infer A\downarrow
   \\
   A \mbox{ free for }x\mbox{ in }B}
  {\Gamma \infer B[A/x]}

Deduction Theorem:
\INFER%
  {\Gamma,A \infer B}
  {\Gamma \infer A \implies B}

Quasi-Equal (Strong) Argument Theorem:
\INFER%
  {\Gamma \infer B \simeq C}
  {\Gamma \infer A\;B \simeq A\;C}

Quasi-Equal (Strong) Function Theorem:
\INFER%
  {\Gamma \infer A \simeq B}
  {\Gamma \infer A\;C \simeq B\;C}

Extensionality:
\INFER%
  {\Gamma \infer A\downarrow
   \\
   \Gamma \infer B\downarrow
   \\
   \Gamma \infer A\;x \simeq B\;x
  }
  {\Gamma \infer A = B}



\subsubsection{Propositional Connectives}


\subsubsection{Axioms}


\subsubsection{Inference Rules}


\subsubsection{Derived Rules}


\subsubsection{Valuations}


\subsubsection{Meta-theorems}
