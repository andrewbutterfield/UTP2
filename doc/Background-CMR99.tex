\subsection{Partial Logics (Cerioli/Mossakowski/Reichel 1999)}
Based on \cite{CMR99}.

The key idea is to have a two-valued logic that handles
terms which may not denote.

An important point to note here is that the axioms and predicates
are \emph{``alphabetised''}.

\subsubsection{Syntax}

The define formulas over signatures $\Sigma$ and (free) variable sets
$X$ as belonging to $Form(\Sigma,X)$.

They define \emph{first order axioms} as a pair $X.P$ where $P \in Form(\Sigma,X)$,
so $X$ must cover all free variables in $P$.
If omitted, $X$ is assumed equal to the free variables of $P$.
If $X=\emptyset$ the axiom is \emph{closed}.

\subsubsection{Propositional Connectives}

\subsubsection{Substitution}

\RLEQNS{
   \__{\dagger\_} &\defs& Substn \times \power Variable \fun Substn & \elabel{Subs:id-ovr:sig}
\\ \theta_{\dagger X} &\defs& \theta \override Substn(id_X) & \elabel{Subs:id-ovr:def}
\\ \_[\_] &:& Predicate \times Substn \pfun Predicate & \elabel{Subs:p-pred:sig}
\\ (t_1 =_e t_2)[\theta] &\defs_E& t_1[\theta]=_e t_2[\theta] & \elabel{Subs:p-pred:eq:def}
\\ p(e_1,\ldots,e_n)[\theta] &\defs_E& p(e_1[\theta],\ldots,e_n[\theta]) & \elabel{Subs:p-pred:pred:def}
\\ False[\theta] &\defs_E& False &\elabel{Subs:p-pred:F:def}
\\ (P \land Q)[\theta] &\defs_S& P[\theta] \land Q[\theta]  &\elabel{Subs:p-pred:and:def}
\\ (P \implies Q)[\theta] &\defs_S& P[\theta] \implies Q[\theta]  &\elabel{Subs:p-pred:imp:def}
\\ (\forall \vec x @ P)[\theta]
   &\defs_S&
   \left\{
     \begin{array}{ll}
       \forall \vec x @ P[\theta_{\dagger\vec x}], & \nocapture~\theta~\vec x \\
       undefined, & \hbox{otherwise.}
     \end{array}
   \right.
   & \elabel{Subs:p-pred:all:def}
}
Here $Substn(\vartheta)$ denotes the obvious lifting of $\vartheta:Variable \pfun Variable$
to a substitution (range injection into $Expr$).

\begin{itemize}
  \item Substitution Lemma for $\models^t$ \elabel{Subst-Lemma:models-t}
  \\ Given $\tval:Y \fun A$, $\theta:X \fun term(Y)$ and $P$ (over $X$), and the conditions:
    \begin{itemize}
      \item $\tval^{\#} \circ \theta : X \fun A$ is total, i.e forall $x \in X$
       we have $\tval \valsat \theta(x)\defined$
      \item $P[\theta]$ is defined
    \end{itemize}
    then
    $\tval^{\#} \circ \theta \valsat P
       ~\IFF~
       \tval \valsat P[\theta]
    $
  \item Substitution Lemma for $\models^p$ \elabel{Subst-Lemma:models-p}
  \\ Given $\pval:Y \pfun A$, $\theta:X \fun term(Y)$ and $P$ (over $X$), and the condition:
    \begin{itemize}
      \item $P[\theta]$ is defined
    \end{itemize}
    then
    $\pval^{\#} \circ \theta \valsat P
       ~\IFF~
       \pval \valsat P[\theta]
    $
\end{itemize}

\subsubsection{Derivation Rules}

Abbreviations:
\RLEQNS{
   e\defined &\defs& e =_e e & \elabel{abrv:defd:eeq}
\\ e =_w e' &\defs& (e\defined \land e'\defined) \implies e =_e e'
   & \elabel{abrv:weq:eeq}
\\ e =_s e' &\defs& (e\defined \lor e'\defined) \implies e =_e e'
   & \elabel{abrv:seq:eeq}
\\ e =_w e' &\defs& (e\defined \land e'\defined) \implies e =_s e'
   & \elabel{abrv:weq:seq}
\\ e =_e e' &\defs& e\defined \implies e =_s e'
   & \elabel{abrv:eeq:seq}
\\ e =_e e' &\defs& e\defined \land e'\defined \land e =_w e'
   & \elabel{abrv:eeq:weq:defd}
\\ e =_s e' &\defs& (e\defined \lor e'\defined)
                    \implies
                    e\defined \land e'\defined \land e =_w e'
   & \elabel{abrv:seq:weq:defd}
\\ \lnot P &\defs& P \implies False & \elabel{abrv:not:implies:F}
\\ P \lor Q &\defs& \lnot(\lnot P \land \lnot Q) & \elabel{abrv:or:deMorgan}
\\ P \equiv Q &\defs& (P \implies Q) \land (Q \implies P)
  & \elabel{abrv:equiv:bi-implies}
\\ \exists \vec x @ P &\defs& \lnot(\forall \vec x @ \lnot P) & \elabel{abrv:exists:deMorgan}
}
Derivation Rules, assuming $\Phi,\Phi_1,\Phi_2$ and $\Phi_3$
as sets of formulae over $X$.
\RRULES{
   \INFER{~}{\Phi \infer^t_{\Sigma,X} P \qquad P \in \Phi}
   &  \elabel{rule:Assumption:total}
\\\\ \INFER%
     {\Phi_1 \infer^t_{\Sigma,X} P
      \qquad
      \Phi_2 \infer^t_{\Sigma,X} Q}
     {\Phi_1\cup\Phi_2 \infer^t_{\Sigma,X} P\land Q}
   & \elabel{rule:And-intro:total}
\\\\ \INFER%
     {\Phi \infer^t_{\Sigma,X} P\land Q}
     {\Phi \infer^t_{\Sigma,X} P}
   & \elabel{rule:And-left-elim:total}
\\\\ \INFER%
     {\Phi \infer^t_{\Sigma,X} P\land Q}
     {\Phi \infer^t_{\Sigma,X} Q}
   & \elabel{rule:And-right-elim:total}
\\\\ \INFER%
     {\Phi_1 \cup \setof{P} \infer^t_{\Sigma,X} Q
      \\\\
      \Phi_2 \cup \setof{(P \implies False)}\infer^t_{\Sigma,X} Q}
     {\Phi_1\cup\Phi_2 \infer^t_{\Sigma,X} Q}
   & \elabel{rule:Tertium-non-datur:total}
\\\\ \INFER%
     {\Phi \infer^t_{\Sigma,X} False}
     {\Phi \infer^t_{\Sigma,X} P}
   & \elabel{rule:Absurdity:total}
\\\\ \INFER%
     {\Phi_1 \infer^t_{\Sigma,X} \bigwedge_i P_i \implies Q_x
      \\\\
      \Phi_2 \infer^t_{\Sigma,Y}\bigwedge_j Q_j \implies R}
     {\Phi_1\cup\Phi_2 \infer^t_{\Sigma,X\cup Y}
         \bigwedge_i P_i \land \bigwedge_j Q_j \implies R, j\neq x}
   & \elabel{rule:Cut:total}
\\\\ \INFER%
     {\Phi \cup \Phi_1 \infer^t_{\Sigma,X} P}
     {\Phi \infer^t_{\Sigma,X} \bigwedge\Phi_1 \implies P}
   & \elabel{rule:Implies-intro:total}
\\\\ \INFER%
     {\Phi\infer^t_{\Sigma,X} \forall \vec y @ P}
     {\Phi \infer^t_{\Sigma,X\cup\vec y} P}
   & \elabel{rule:Forall-elim:total}
\\\\ \INFER%
     {\Phi\infer^t_{\Sigma,X\cup\vec y} P
      \quad
      \vec y \cap \fv~\Phi = \emptyset}
     {\Phi \infer^t_{\Sigma,X} \forall \vec y @  P}
   & \elabel{rule:Forall-intro:total}
\\\\ \INFER%
     {~}{\Phi \infer^t_{\Sigma,X} x =_e x \qquad x \in X}
   & \elabel{rule:Reflexivity:total}
\\\\ \INFER%
     {\Phi \infer^t_{\Sigma,X} P
      \\\\
      \theta:X\fun term(Y) \\ P[\theta]\defined}
     {\Phi \infer^t_{\Sigma,X\cup Y} (\bigwedge_{x \in X}x=_e\theta(x))\implies P[\theta]}
   & \elabel{rule:Congruence:total}
}

\RRULES{
     \INFER%
     {\Phi \infer^t_{\Sigma,X} P
      \\\\
      \theta:X\fun term(Y) \\ P[\theta]\defined \\ \Phi[\theta]\defined}
     {\Phi[\theta] \infer^t_{\Sigma,Y} (\bigwedge_{x \in X}\theta(x)\defined)\implies P[\theta]}
   & \elabel{rule:Substitution:total}
\\\\ \INFER%
     {\Phi\infer^t_{\Sigma,X} e_1 =_e e_2 \\ e \mbox{ sub-term of }e_1\mbox{ or }e_2}
     {\Phi \infer^t_{\Sigma,X} e\defined}
   & \elabel{rule:Function-strict:total}
\\\\ \INFER%
     {\Phi\infer^t_{\Sigma,X} p(e_1,\ldots,e_n)}
     {\Phi \infer^t_{\Sigma,X} e_i\defined}
   & \elabel{rule:Predicate-strict:total}
\\\\ \INFER%
     {\Phi\infer^t_{\Sigma,X} \bigwedge_i e_i\defined\\ f\mbox{ total}}
     {\Phi \infer^t_{\Sigma,X} f(e_1,\ldots,e_n)\defined}
   & \elabel{rule:Totality:total}
}

For a system based on Strong Equality and Definedness,
drop \ecite{rule:Reflexivity:total} and \ecite{rule:Congruence:total}
and add:
\RLEQNS{
     \INFER%
     {~}{\Phi \infer^t_{\Sigma,X} e =_s e \qquad e \in term(X)}
   & \elabel{rule:S-Reflexivity:total}
\\\\ \INFER%
     {\Phi \infer^t_{\Sigma,X} e_1 =_e e_2}
     {\Phi \infer^t_{\Sigma,X} e_1 =_s e_2}
   & \elabel{rule:S-Equality1:total}
\\\\ \INFER%
     {\Phi \infer^t_{\Sigma,X} e_1\defined
      \\
      \Phi \infer^t_{\Sigma,X} e_1 =_s e_2}
     {\Phi \infer^t_{\Sigma,X} e_1 =_e e_2}
   & \elabel{rule:S-Equality2:total}
\\\\ \INFER%
     {\Phi \infer^t_{\Sigma,X} e_1\defined \implies False
      \\
      \Phi \infer^t_{\Sigma,X} e_2\defined \implies False}
     {\Phi \infer^t_{\Sigma,X} e_1 =_s e_2}
   & \elabel{rule:S-Equality3:total}
\\\\ \INFER%
     {\Phi \infer^t_{\Sigma,X} P
      \\\\
      \theta:X\fun term(Y) \\ P[\theta]\defined}
     {\Phi \infer^t_{\Sigma,X\cup Y} (\bigwedge_{x \in X}x=_s\theta(x))\implies P[\theta]}
   & \elabel{rule:S-Congruence:total}
}

The partial calculus:
\RRULES{
   \INFER{~}{\Phi \infer^p_\Sigma P \qquad P \in \Phi}
   &  \elabel{rule:Assumption:partial}
\\\\ \INFER%
     {\Phi_1 \infer^p_\Sigma P
      \qquad
      \Phi_2 \infer^p_\Sigma Q}
     {\Phi_1\cup\Phi_2 \infer^p_\Sigma P\land Q}
   & \elabel{rule:And-intro:partial}
\\\\ \INFER%
     {\Phi \infer^p_\Sigma P\land Q}
     {\Phi \infer^p_\Sigma P}
   & \elabel{rule:And-left-elim:partial}
\\\\ \INFER%
     {\Phi \infer^p_\Sigma P\land Q}
     {\Phi \infer^p_\Sigma Q}
   & \elabel{rule:And-right-elim:partial}
\\\\ \INFER%
     {\Phi_1 \cup \setof{P} \infer^p_\Sigma Q
      \\\\
      \Phi_2 \cup \setof{(P \implies False)}\infer^p_\Sigma Q}
     {\Phi_1\cup\Phi_2 \infer^p_\Sigma Q}
   & \elabel{rule:Tertium-non-datur:partial}
\\\\ \INFER%
     {\Phi \infer^p_\Sigma False}
     {\Phi \infer^p_\Sigma P}
   & \elabel{rule:Absurdity:partial}
\\\\ \INFER%
     {\Phi\infer^p_\Sigma \forall \vec y @ P}
     {\Phi \infer^t_\Sigma (\bigwedge_{y \in \vec y} y\defined) \implies P}
   & \elabel{rule:Forall-elim:partial}
\\\\ \INFER%
     {\Phi\infer^t_\Sigma (\bigwedge_{y \in \vec y} y\defined) \implies P
      \quad
      \vec y \cap \fv~\Phi = \emptyset}
     {\Phi \infer^p_\Sigma \forall \vec y @  P}
   & \elabel{rule:Forall-intro:partial}
\\\\ \INFER%
     {~}
     {\Phi \infer^p_\Sigma e_1 =_e e_2 \implies e_2 =_e e_1}
   & \elabel{rule:Symmetry:partial}
\\\\ \INFER%
     {\Phi \infer^p_\Sigma P
      \\\\
      \theta:X\fun term(Y) \\ P[\theta]\defined \\ \Phi[\theta]\defined}
     {\Phi[\theta] \infer^t_\Sigma P[\theta]}
   & \elabel{rule:Substitution:partial}
\\\\ \INFER%
     {\Phi\infer^p_\Sigma e_1 =_e e_2 \\ e \mbox{ sub-term of }e_1\mbox{ or }e_2}
     {\Phi \infer^p_\Sigma e\defined}
   & \elabel{rule:Function-strict:partial}
\\\\ \INFER%
     {\Phi\infer^p_\Sigma p(e_1,\ldots,e_n)}
     {\Phi \infer^p_\Sigma e_i\defined}
   & \elabel{rule:Predicate-strict:partial}
\\\\ \INFER%
     {\Phi\infer^p_\Sigma \bigwedge_i e_i\defined\\ f\mbox{ total}}
     {\Phi \infer^p_\Sigma f(e_1,\ldots,e_n)\defined}
   & \elabel{rule:Totality:partial}
}


\subsubsection{Derived Rules}


\subsubsection{Valuations/Satisfaction}

\begin{itemize}
  \item Partial Signature \elabel{Part-Sig:def}
    \\ $\Sigma = (S,\Omega,P\Omega,\Pi)$
       of sorts, total operations, partial operations and predicates,
       all well-sorted.
  \item Partial Structure \elabel{Part-Str:def}
    \\ $(\carrier A,\mathcal I^{w,s}_A,\mathcal J^{s_1,\ldots,s_n}_A)$
    of carrier sets,
    function interpretations
    and predicate interpretations,
    defined w.r.t. partial signatures (\ecite{Part-Sig:def})
    \\ For $f \in \Omega$, $pf \in P\Omega$
       and $p \in \Pi$, we define shorthands
       for their respective interpretations as $f_A$, $pf_A$ and $p_A$.
  \item Total Valuation \elabel{Tot-Val:def}
    \\ $\tval : Variable \pfun \carrier A$ associates a denoting value
       with every variable in its domain.
       We write $\tval_X$ when we want to make the domain explicit.
  \item Partial Valuation \elabel{Part-Val:def}
    \\ $\pval : Variable \pfun \carrier A_\bot$ may associate a non-denoting value
       with some variables in its domain.
       We write $\pval_X$ when we want to make the domain explicit.
  \item Term Valuation \elabel{Term-Val:def}
    \\ By $\tval^\#$ and $\pval^\#$ we denote respectively
    the ``obvious'' extensions of $\tval$ and $\pval$ to
    terms built over the variables in their domains.
  \item Satisfaction w.r.t Valuation \elabel{Val:Sat:def}
    \\ Let \val\ stand for either $\tval$ or $\pval$ in:
    \RLEQNS{
       \_ \valsat \_ &:& Valuation \times Predicate \fun Bool
       & \elabel{Val-Sat:sig}
    \\ \val \valsat t_1 =_e t_2
       &\defs&
       \val^\#(t_1) =_e \val^\#(t_2) & \elabel{Val-Sat:eq:def}
    \\ \val \valsat p(t_1,\ldots,t_n)
       &\defs&
       \val^\#(t_1)\defined \land \ldots \land \val^\#(t_n)\defined
    \\&& {} \land (\val^\#(t_1),\ldots,\val^\#(t_n)) \in p_A
    & \elabel{Val-Sat:p:def}
    \\ \val \valsat False &\defs& F & \elabel{Val-Sat:F:def}
    \\ \val \valsat P \land Q
       &\defs&
       \val\valsat P \AND \val\valsat Q
       & \elabel{Val-Sat:and:def}
    \\ \val \valsat P \implies Q
       &\defs&
       \val\valsat P \IMPLIES \val\valsat Q
       & \elabel{Val-Sat:imp:def}
    \\ \val \valsat \forall \vec x @ P
       &\defs&
       \FORALL \tval'_{\vec x} \WEHAVE \val\override\tval' \valsat P
       & \elabel{Val-Sat:imp:def}
    }
  \item Total Valuation Axiom Satisfaction
    \\ Partial $\Sigma$-structure $A$ satisfies axiom $X.P$ w.r.t. total valuations
    \RLEQNS{
       A \models^t_\Sigma X.P
       &\defs&
       \FORALL \tval_X \WEHAVE \tval_X \valsat P
       & \elabel{Axiom:Sat:total}
    }
  \item Partial Valuation Axiom Satisfaction
    \\ Partial $\Sigma$-structure $A$ satisfies axiom $X.P$
    w.r.t. partial valuations
    \RLEQNS{
       A \models^p_\Sigma X.P
       &\defs&
       \FORALL \pval_X \WEHAVE \pval_X \valsat P
       &  \elabel{Axiom:Sat:partial}
    }
\end{itemize}


\subsubsection{Meta-theorems}

\begin{itemize}
  \item Universal Closure of Satisfaction
     \RLEQNS{
        A \models^t_\Sigma \vec x.P
        \quad\IFF
        &A \models^t_\Sigma \emptyset.\forall \vec x @ P&
        \IFF\quad
        A \models^p_\Sigma \emptyset.\forall \vec x @ P
        & \elabel{Sat:univ:closure}
     }
     Note, however that
     $A \models^p_\Sigma \emptyset.\forall x @ x=_e x$,
     but
     $A \not\models^p_\Sigma \setof x. x=_e x$,
  \item Satisfaction Coverage
    \RLEQNS{
       \fv~P \subseteq X \cap Y
       &\IMPLIES&
       A \models^p_\Sigma X.P
       \IFF
       A \models^p_\Sigma Y.P
       & \elabel{Sat:p:cover}
    \\ \fv~P \subseteq X \cap Y \AND \carrier A \neq \emptyset
       &\IMPLIES&
       A \models^t_\Sigma X.P
       \IFF
       A \models^t_\Sigma Y.P
       & \elabel{Sat:t:cover}
    }
    Here $\carrier A \neq \emptyset$ should be read as all carrier sets are non-empty.
  \item Semantical Consequence
    \\
    Given axiom $X.P$, let $\Phi = \setof{ \ldots Y.Q \ldots}$ be an axiom set,
    and $Z = X \cup \bigcup_{Y.Q \in \Phi} Y$, in
    \RLEQNS{
       \Phi \models^t_\Sigma P
       &\IFF&
       \FORALL \tval_Z \WEHAVE  & \elabel{Sem:Cons:total}
    \\ && \quad (\FORALL Y.Q \in \Phi \WEHAVE \tval_Z \valsat Q)
    \\ && \qquad\IMPLIES \tval_Z \valsat P
    \\ \Phi \models^p_\Sigma P
       &\IFF&
       \FORALL \pval_Z \WEHAVE  & \elabel{Sem:Cons:partial}
    \\ && \quad (\FORALL Y.Q \in \Phi \WEHAVE \pval_Z \valsat Q)
    \\ && \qquad\IMPLIES \pval_Z \valsat P
    }
  \item Coverage gives Partial Consequence
    \\Let $\Phi \cup \setof P$ be axioms over alphabet $X \cap Y$, then
    \RLEQNS{
       \Phi \models^p_\Sigma Y.P
       &\IFF&
       \Phi \models^p_\Sigma X.P
       & \elabel{Sem:Cons:p-coverage}
    }
    This property does not hold for $\models^t$.
    \par
    We can therefore ignore alphabets for partial valuations,
    and treat $P$ as an abbreviation for $\fv~P.P$.
  \item Models
    \\\emph{Presentation} $(\Sigma,AX)$ where $AX$ is a set of axioms over $\Sigma$,
    has as model any partial structure $A$ such that $A \models^p_\Sigma AX$.
    We denote the class of models by $Mod_\Sigma(AX)$.
  \item Semantic Consistency
    \\ Presentation $(\Sigma,AX)$ is semantically consistent if $Mod_\Sigma(AX)$ is non-empty.
    \RLEQNS{
       && \Phi \cup \setof{P_1,\ldots,P_n} \models^p_\Sigma Q
       & \elabel{Sem:Cnsq:Cnstnt:partial}
    \\ &\IFF& \Phi  \models^p_\Sigma  P_1 \land \ldots \land P_n \implies Q
    \\ &\IFF& (\Sigma,\Phi\cup \setof{P_1,\ldots,P_n,\lnot Q}) \mbox{ is inconsistent}
    }
  \item Derivation
    \\ Finite sequence of judgments $\Phi \infer^t_{\Sigma,X} P$,
       each an axiom, or obtained from earlier by a rule.
  \item Derivation of Formula
    \\ Derivation of $X.P$ is derivation ending in $\emptyset \infer^t_{\Sigma,X} P$
  \item Soundness \& Completeness of $\infer^t_{\Sigma,\_}$
    \RLEQNS{
        \Phi \models^t_\Sigma X.P
     &  \IFF
     &  \Phi \infer^t_{\Sigma,X \cup Y} P
     & \elabel{Sem:Sound-Complete:rules:total}
    }
    where $Y$ is all the variable-sets in $\Phi$
  \item Soundness \& Completeness of $\infer^p_\Sigma$
    \RLEQNS{
        \Phi \models^p_\Sigma X.P
     &  \IFF
     &  \Phi \infer^p_\Sigma P
     & \elabel{Sem:Sound-Complete:rules:partial}
    }
\end{itemize}


