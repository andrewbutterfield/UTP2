\textbf{Force Matches}
We have (``inverted'') variable tables of the form:
 \begin{eqnarray*}
   a_i &\mapsto& \setof{x_{i1},\ldots,x_{in_i}, \lst x_{i1},\ldots,\lst x_{im_i}}
 \\ \lst a_j &\mapsto& \setof{ \lst x_{j1},\ldots,\lst x_{jn_j}}
 \end{eqnarray*}
and a ``regular'' substitution table of the form
\begin{eqnarray*}
   (\lst x_i,\lst e_i) &\mapsto&
     \setof{(a_{k+1},f_{k+1}),\ldots,(a_n,f_n), (\lst a_1,\lst f_1),\ldots,(\lst a_m\lst f_m)}
\end{eqnarray*}
We can interpret this latter table as being the following two
\begin{eqnarray*}
   \lst x_i &\mapsto&
    \setof{a_{k+1},\ldots,a_n,\lst a_1,,\ldots,\lst a_m}
\\ \lst e_i &\mapsto&
    \setof{f_{k+1},\ldots,f_n, \lst f_1,\ldots,\lst f_m}
\end{eqnarray*}
but noting that $\lst x$ and $\lst e$ in the latter linked.
so we eliminate any occurrences of these $\lst x$ from the first two tables.

We can invert the substitution tables to get variable tables like:
 \begin{eqnarray*}
    a_s &\mapsto& \setof{\lst x_{s1},\ldots,\lst x_{sm_s}}
  \\ \lst a_t &\mapsto& \setof{ \lst x_{t1},\ldots,\lst x_{tn_t}}
 \end{eqnarray*}
We merge this by taking intersections with the inverted variable
tables, as both represent matching constraints that must be satisfied.

The list variables can be broken into general, and reserved.
  Reserved variables here in the inverted tables
  match anything because they are in a quantifier,
  but we do have information about how many standard variables they can
  match:
  \begin{eqnarray*}
     \#_R &:& Var \fun \Nat
  \\ \#_R(R) &\defs& \#_D(\sem{R}_\Gamma)
  \\ \#_D &:& (\power Var)^2 \fun \Nat
  \\ \#_D(V\ominus \emptyset ) &=& \# V
  \\ \#_D(V \ominus x_1,\ldots x_n) &=& (\# V) - n, \qquad n \leq \# V, \quad 0~ \otherwise.
  \\ \#_D(V \ominus \setof{\ldots,\lst v,\ldots} &\leq& \# V
  \end{eqnarray*}
  Reserved variables in the substitution table may not be bound by a quantifier
  and so can only match something with the same denotation.

We then process the single inverted table to construct a mapping \texttt{citab}:
\begin{eqnarray*}
  a_i &\mapsto& n_i \qquad \mbox{number of $\lst x$}
\end{eqnarray*}
We then un-invert the tables to get back to the form:
\begin{eqnarray*}
  x_i &\mapsto& \setof{a_{i1},\ldots,a_{in_i}}
\\ \lst x_j &\mapsto& \setof{a_{j1},\ldots,a_{jn_j}, \lst a_{j1},\ldots,\lst a_{jm_j}}
\end{eqnarray*}
We use \texttt{citab} to sort the \texttt{stab} range elements in ascending order.
We then process the sorted table to update the bindings.
We then do a greedy assignment of mappings from  \texttt{mtab},
using size information for reserved variables.
Finally we do something similar with the substitution matches.
