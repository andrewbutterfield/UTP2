\section{Proof Miscellany}

\subsection{Substitution}

We present a brief theory of substitution to allow us
to have a small calculus for combining same.
We take a simple expression language with components and binders:
\begin{eqnarray*}
   u,v \in Var && \mbox{unbounded set of variables}
\\ e \in Expr &::=& v | (e~e) | \lambda v @ e
\end{eqnarray*}
We define a substitution as an variable to expression
functional map:
\begin{eqnarray*}
   \sigma,\tau \in Subs &=& Var \pfun Expr
\end{eqnarray*}
We now define the outcome of a substitution:
\begin{eqnarray*}
   \_ ! \_ &:& Expr \times Subs \fun Expr
\\ v ! \sigma
   &\defs& \sigma(v) \cond{v \in \sigma} v
\\ (e_1~e_2) ! \sigma &\defs& ((e_1 ! \sigma)~(e_2 ! \sigma))
\\ (\lambda v @ e) ! \sigma
   &\defs&
   \lambda v @ (e ! \sigma \hide \setof v)
\end{eqnarray*}
Note that we do not address the notion of variable capture here
---
this is a ``raw'' substitution operator.
We now present a number of laws, with proofs.
The first characterise identity substitutions:
\begin{eqnarray*}
   e ! \mapof{} &=& e
\\\textsf{Proof} && trivial~induction~on~e \qed
\end{eqnarray*}
\begin{eqnarray*}
   e ! \mapof{u \to u} &=& e
\\\textsf{Proof} && trivial~induction~on~e \qed
\end{eqnarray*}
In general we say that substitutions are equivalent
if they have the same effect on all expressions:
\begin{eqnarray*}
 \sigma_1 \simeq \sigma_2
  &\defs&
  \forall e @ e ! \sigma_1 = e ! \sigma_2
\end{eqnarray*}
We can ignore identity substitutions:
\begin{eqnarray*}
   \sigma \cup \mapof{u \to u} &\simeq& \sigma
\\\textsf{Proof} && induction~on~e
\\\textsf{Base} && v!(\sigma \cup \mapof{u \to u})=v!\sigma
\\ v!(\sigma \cup \mapof{u \to u})
   &=& (\sigma \cup \mapof{u \to u})(v)
       \cond{v \in \sigma \cup \mapof{u \to u}}
       v
\\ &=& (\sigma \cup \mapof{u \to u})(v)
\\  && \cond{ v \in \sigma}
\\  && ( (\sigma \cup \mapof{u \to u})(v)
         \cond{v \in \mapof{u \to u}}
         v )
\\ &=& \sigma(v)
       \cond{ v \in \sigma}
       ( (\sigma \cup \mapof{u \to u})(v)
         \cond{v =u}
         v )
\\ &=& \sigma(v)
       \cond{ v \in \sigma}
       ( \mapof{u \to u}(v)
         \cond{v=u}
         v )
\\ &=& \sigma(v)
       \cond{ v \in \sigma}
       ( u \cond{v=u} v )
\\ &=& \sigma(v) \cond{ v \in \sigma} v
\\ &=& v!\sigma \qquad (note~indep.~of~\sigma)
\\\textsf{Step 1} && e = e_1~e_2, \quad trivial
\\\textsf{Step 2} && e = \lambda v @ e_b
\\ (\lambda v @ e_b)!(\sigma \cup \mapof{u \to u})
   &=& \lambda v @ (e_b ! (\sigma \cup \mapof{u \to u})\hide\setof v)
\\\textsf{Case 2.1} && u=v, ~v \not\in \sigma
\\ \lambda v @ (e_b ! (\sigma \cup \mapof{u \to u})\hide\setof v)
   &=& \lambda v @ (e_b ! \sigma)
\\ &=& \lambda v @ (e_b ! (\sigma \hide\setof v))
\\ &=& (\lambda v @ e_b) ! \sigma
\\\textsf{Case 2.2} && u \neq v
\\ \lambda v @ (e_b ! (\sigma \cup \mapof{u \to u})\hide\setof v)
   &=& \lambda v @ (e_b ! ((\sigma\hide\setof v) \cup \mapof{u \to u}))
\\ &=& \lambda v @ (e_b ! (\sigma\hide\setof v)) \qquad (hyp.~with~diff.~\sigma)
\\ &=& (\lambda v @ e_b) ! \sigma
\\ \qed
\end{eqnarray*}
We define substitution composition as follows:
$\sigma_1;\sigma_2$ is the substitution that satisfies, for any $e$,
the identity:
$$
  e!(\sigma_1;\sigma_2) = (e!\sigma_1)!\sigma_2
$$
Let $\tau = \sigma_1;\sigma_2$.
We can attempt to derive an expression for $\tau$ in terms of $\sigma_1$ and $\sigma_2$
by doing a proof of the identity above, using induction on $e$:
\begin{eqnarray*}
   P(e) &\defs& e!\tau = (e!\sigma_1)!\sigma_1)
\end{eqnarray*}
We start with the base-case:
\begin{eqnarray*}
   v!\tau &=& (v!\sigma_1)!\sigma_2
\\ \tau(v) \cond{v \in \tau} v
   &=&
   (\sigma_1(v) \cond{v \in \sigma_1} v)!\sigma_2
\\ &=&
   \sigma_1(v)!\sigma_2 \cond{v \in \sigma_1} v!\sigma_2
\\ &=&
   \sigma_1(v)!\sigma_2 \cond{v \in \sigma_1} (\sigma_2(v) \cond{v \in \sigma_2} v)
\\ &=&
   (\sigma_1(v)!\sigma_2 \cond{v \in \sigma_1} \sigma_2(v))
   \cond{v \in \sigma_1 \lor v \in \sigma_2}
   v
\end{eqnarray*}
From this we can identify the following identities:
\begin{eqnarray*}
   v \in \tau &\equiv& v \in \sigma_1 \lor v \in \sigma_2
\\ \tau(v) &=& \sigma_1(v)!\sigma_2 \cond{v \in \sigma_1} \sigma_2(v)
\end{eqnarray*}
These suggest the following definition:
\begin{eqnarray*}
 \sigma_1;\sigma_2 &\defs& (id \pfun \lambda e @ e!\sigma_2)\sigma_1 \cup (\sigma_2\hide\sigma_1)
\end{eqnarray*}
We can check this, by proving
$P(e) \defs e!(\sigma_1;\sigma_2) = (e!\sigma_1)!\sigma_2$,
by induction on $e$.
Expanding the definition of $;$ means we show:
$$
 e!((id \pfun \lambda e @ e!\sigma_2)\sigma_1 \cup (\sigma_2\hide\sigma_1))
 =
 (e!\sigma_1)!\sigma_2
$$
\begin{eqnarray*}
   && \textsf{Base Case : } e = v
\\ \lefteqn{v!((id \pfun \lambda e @ e!\sigma_2)\sigma_1 \cup (\sigma_2\hide\sigma_1))}
\\ &=& ((id \pfun \lambda e @ e!\sigma_2)\sigma_1 \cup (\sigma_2\hide\sigma_1))(v)
\\  && \quad \cond{v \in ((id \pfun \lambda e @ e!\sigma_2)\sigma_1 \cup (\sigma_2\hide\sigma_1))}
             v
\\ &=& (((id \pfun \lambda e @ e!\sigma_2)\sigma_1)(v)
         \cond{v \in \sigma_1}
         \sigma_2(v))
\\  && \quad \cond{v \in \sigma_1 \lor v \in \sigma_2}
             v
\\ &=& (\sigma_1(v))!\sigma_2
       \cond{v \in \sigma_1}
       (( \sigma_2(v)) \cond{v \in \sigma_2} v )
\\ &=& (\sigma_1(v))!\sigma_2
       \cond{v \in \sigma_1}
       v!\sigma_2
\\ &=& (\sigma_1(v) \cond{v \in \sigma_1} v)!\sigma_2
\\ &=& (v!\sigma_1)!\sigma_2
\\ && \textsf{Induction Step 1 : }  e = e_1~e_2
\\ \lefteqn{(e_1~e_2)!((id \pfun \lambda e @ e!\sigma_2)\sigma_1 \cup (\sigma_2\hide\sigma_1))}
\\ &=& (e_1!((id \pfun \lambda e @ e!\sigma_2)\sigma_1 \cup (\sigma_2\hide\sigma_1)))
\\  && (e_2!((id \pfun \lambda e @ e!\sigma_2)\sigma_1 \cup (\sigma_2\hide\sigma_1)))
\\ &=& ((e_1!\sigma_1)!\sigma_2)~((e_2!\sigma_1)!\sigma_2) \qquad ind.~hyp.
\\ &=& ((e_1!\sigma_1)~(e_2!\sigma_1))!\sigma_2
\\ &=& ((e_1~e_2)!\sigma_1)!\sigma_2
\\ && \textsf{Induction Step 2 : }  e = \lambda v @ e_b
\\ \lefteqn{(\lambda v @ e_b)!((id \pfun \lambda e @ e!\sigma_2)\sigma_1 \cup (\sigma_2\hide\sigma_1))}
\\ &=& \lambda v @ e_b!(((id \pfun \lambda e @ e!\sigma_2)\sigma_1 \cup (\sigma_2\hide\sigma_1))\hide v)
\\ &=& \lambda v @ e_b!( ((id \pfun \lambda e @ e!\sigma_2)\sigma_1)\hide v \cup (\sigma_2\hide\sigma_1)\hide v)
\\ &=& \lambda v @ e_b!( ((id \pfun \lambda e @ e!\sigma_2)(\sigma_1\hide v)) \cup ((\sigma_2\hide v)\hide(\sigma_1\hide v)) )
\end{eqnarray*}
This fails --- we cannot define substitution composition for a
raw-substitution example!
E.g. take
\begin{eqnarray*}
   \sigma_1 &=& \mapof{x \to 3y, y \to 2+x}
\\ \sigma_2 &=& \mapof{x \to 10y}
\\ \sigma_1;\sigma_2 &=& \mapof{x \to 3y, y \to 2+10y}
\\  && (\lambda x @ (x,y))!\sigma_1!\sigma_2
\\ &=& \lambda x @ (x,y)!\sigma_1\hide x ! \sigma_2 \hide x
\\ &=& \lambda x @ (x,2+x) ! \sigma_2 \hide x
\\ &=& \lambda x @ (x,2+x)
\\ &\not=& (\lambda x @ (x,y))!(\sigma_1;\sigma_2)
\\ &=& \lambda x @ (x,y)!((\sigma_1;\sigma_2)\hide x)
\\ &=& \lambda x @ (x,2+10y)
\end{eqnarray*}
We can make it work for a substitution algorithm that
$\alpha$-renames with fresh variables to avoid capture:
\begin{eqnarray*}
    && (\lambda x @ (x,y))!\sigma_1!\sigma_2
\\ &=& \lambda a @ (a,y)!\sigma_1!\sigma_2
\\ &=& \lambda a @ (a,2+x)!\sigma_2
\\ &=& \lambda a @ (a,2+10y)
\\ &=& (\lambda x @ (x,y))!(\sigma_1;\sigma_2)
\\ &=& \lambda a @ (a,y) ! (\sigma_1;\sigma_2)
\\ &=& \lambda a @ (a,2+10y)
\end{eqnarray*}


\newpage
\subsection{Matching Normal Forms}

We want to be able to match predicates like
$P\land(P\lor Q)$,
$P\land(Q\lor P)$,
$(Q\lor P)\land P$,
and $(P\lor Q)\land P$,
against the single, representative pattern $A \land (A \lor B)$ (say).
Can we find a ``normal form'' that does not depend
on ordering by Pvar names?

Laws needing this treatment typically have conjunction and/or disjunction,
with some kind of side-condition.
Such laws include:
\begin{eqnarray*}
   A \land (A \lor B) &\equiv& A
\\ A \lor (A \land B) &\equiv& A
\\ \exists x_i,y_j @ \bigwedge(x_i=e_i) \land A
   &\equiv&
   \exists y_j @ A[e_i/x_i]
   \qquad i \in I, j \in J, x_i \notin e_{i'}
\end{eqnarray*}
