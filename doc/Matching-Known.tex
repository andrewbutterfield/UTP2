Matching needs more than just the pattern and test:
it also requires a matching context ($\Gamma$)
that supplies information about various variables in order to
determine how they match.
The basic idea is that a pattern variable $v$ matches any corresponding
object, unless that variable is ``known'' ($v \in \Gamma$).
If a variable is known, then what it matches is governed
by what is known about it ($\Gamma(v)$).

It should be remembered that a pattern variable that is known
may occur in a pattern underneath a quantifier binding that variable,
in which case that instance of the variable is not known,
and is free to match any corresponding object.
A pattern variable in a quantifier binding list 
is not ``known'', but is restricted to matching only other
binding list variables.

\subsubsection{Known Names}

We supply as a context the following trie lists:
\begin{itemize}
  \item observation variables
  \item type variables denoting types
  \item variables denoting constants
  \item e-variables denoting expressions
  \item p-variables denoting predicates
\end{itemize}
the first being observation variables,
whilst the rest are the known (named) 
types/variables/expressions/predicates in each context.
A pattern variable not listed as known, matches anything of the corresponding type:
\begin{itemize}
  \item \texttt{Tvar}: any \texttt{Type}
  \item \texttt{Var}: any \texttt{Var}
  \item \texttt{Evar}: any \texttt{Expr}
  \item \texttt{Pvar}: any \texttt{Pred}
\end{itemize}
If a \emph{pattern} variable is defined (indicates a known entity),
it only matches itself, or what it is defined to be.
