\section{Foundations}

\subsection{1st-order Equational Logic (Tourlakis)}\label{ssec:1EL:Tourlakis}

We shall adapt the formalisation of 1st-order equational logic (\textbf{EL1})
due to Tourlakis \cite{journals/logcom/Tourlakis01},
as reprised here (with minor syntactic tweaks).

\subsubsection{EL1 Syntax}\label{sssec:1EL:syntax}

\RLEQNS{
   x \in ObjVar && (given) & \mbox{Object Variables}
\\ p \in PropVar && (given) & \mbox{Propositional Variables}
\\ a \in Const && (given) & \mbox{Constant symbols}
\\ P \in PredRel && (given) & \mbox{Predicate and Relation symbols}
\\ f \in Func && (given) & \mbox{Function symbols}
\\ t \in Term &::=& x | a | f~t_1~t_2~\ldots~t_n & \mbox{Terms}
\\ \alpha \in AF &::=& true|false|p & \mbox{Atomic Formulas}
\\                 &|& t_1=t_2  & \mbox{Equality Assertion}
\\                 &|& P~t_1~t_2~\ldots~t_n & \mbox{Pred.-Symbol application}
\\ A,B \in WFF &::=& \alpha & \mbox{Well-Formed Formulas}
\\             &|& \lnot A
\\             &|& (A\equiv B) | (A \implies B)
\\             &|& (A \land B) | (A \lor B)
\\             &|& (\forall x @ A)
}

\subsubsection{EL1 Rules}\label{sssec:1EL:rules}

\begin{description}
  \item[Ax1.] All propositional axioms from \cite{gries.93}.
  \item[Ax2.] $A \lor (\forall x @ B) \equiv (\forall x @ A \lor B), x \notin A$
  \item[Ax3.] $(\forall x @ A) \implies A[x:=t]$ ($t$ substitutable in $x$ (in $A$))
  \item[Ax4.] $x = x$
  \item[Ax5. (Liebniz)] $x = t \implies (A \equiv A[x:=t])$
     ($t$ substitutable in $x$ (in $A$))
  \item[Inf1. (Substitution)] $\INFER{A}{A[p:=B]}$ (no capture)
  \item[Inf2. (Leibniz)] $\INFER{A\equiv B}{C[p:=A] \equiv C[p:=B]}$
  \item[Inf3. (Equanimity)] $\INFER{A, A \equiv B}{B}$
  \item[Inf4. (Transitivity)] $\INFER{A\equiv B, B \equiv C}{A \equiv C}$\\
\end{description}

\subsubsection{EL1 Definitions}\label{sssec:1EL:defns}

\RLEQNS{
  (\exists x @ A) &\defs& \lnot(\forall x @ \lnot A)
  & \mbox{Existential}
}

\subsubsection{EL1 Proof}\label{sssec:1EL:proof}

\begin{description}
  \item[Logical Axioms (LA)]
     are the closure under \textbf{Inf1}
     of \textbf{Ax1} and all instances of \textbf{Ax2-Ax5}.
  \item[Non-Logical Axioms]
    are a supplied set of assumed formulas, here denoted by $\Gamma$.
  \item[$\Gamma$-Theorems]
   is the set obtained by closing $\textbf{LA} \cup \Gamma$ under \textbf{Inf2-Inf3}.
  \item[$\Gamma$-Proof $A_1,\ldots,A_n$]
    is valid if either $A_i \in \textbf{LA} \cup \Gamma$
    or is the result of using \textbf{Inf2-Inf3} on $A_j$,  $j < i$.
  \item[$\Gamma \vdash A$] if $A$ is at the end of some $\Gamma$-proof.
\end{description}

\subsubsection{EL1 Metatheorems}\label{sssec:1EL:meta}

We use the numbering in \cite{journals/logcom/Tourlakis01}.
\begin{description}
  \item[Remark 3.8(2)] if $A_1,\ldots,A_n$ have $\Gamma$-proofs
   and $B$ has a $\{A_1,\ldots,A_n\}$-proof, then $B$ has a $\Gamma$-proof.
  \item[Remark 3.8(3)] If $\Gamma \subseteq \Delta$ and $\Gamma \vdash A$,
  then $\Delta \vdash A$.
  \item[Meta 4.1] Rule \textbf{Inf4} is derived.
  \item[Meta 4.2] $\Gamma \vdash A$ iff $\Gamma \vdash A \equiv true$.
  \item[Meta 4.3] $\vdash (\forall x @ true) \equiv true$.
  \item[Meta 4.4] $\Gamma \vdash A$ iff $\Gamma \vdash (\forall x @ A)$
       --- $\forall$ Intro/Elim.
  \item[Corr 4.5] $A \vdash \forall x @ A$.
  \item[Corr 4.7] $A[x_1,\ldots,x_n] \vdash A[t_1,\ldots,t_n]$
       --- Substitution.
  \item[Meta 4.8]
     $\vdash (A \implies \forall x @ B) \equiv (\forall x @ A \implies B)$,
     provided $x \notin A$.
  \item[Corr 4.9] $\vdash A \equiv (\forall x @ A)$, provided $x \notin A$.
  \item[Remark 4.10] $\vdash A \equiv (\exists x @ A)$, provided $x \notin A$.
  \item[Meta 4.11] $\Gamma \vdash A \implies B$ iff $\Gamma \vdash A \implies (\forall x @ B)$,
    provided $x \notin A$.
  \item[Corr 4.12] $\Gamma \vdash B \implies A$ iff $\Gamma \vdash (\exists x @ B) \implies A$,
    provided $x \notin A$.
  \item[Meta 4.13] $\vdash (\forall x @ A) \equiv (\forall z @ A[x:=z])$,
    provided $z$ not free or bound in $\forall x @ A$
    --- dummy renaming.
  \item[Lemma 4.22] $\vdash (\forall x @ A \implies B)
                            \implies (\forall x @ A)
                            \implies (\forall x @ B)$.
  \item[Lemma 4.23] $A \implies (B \equiv C)
                     \vdash
                     A \implies ((\forall x @ B) \equiv (\forall x @ C))$,
                     provided $x \notin A$.
  \item[Meta 4.24] If $A$ is closed and $\Gamma,A \vdash B$, then $\Gamma \vdash A \implies B$
       --- The Deduction Theorem.
   \\ also $\Gamma \vdash A \implies B$ implies that $\Gamma, A \vdash B$,
     with no restrictions.
  \item[Meta 4.25]
    Extend language with new constant $e$ and let $\Gamma'$
    be augmented theory with the same non-logical axioms as $\Gamma$.
    Then $\Gamma' \vdash A[e]$ implies $\Gamma \vdash A[x]$,
    provided $x$ is not free our bound anywhere in the $\Gamma'$-proof.
  \item[Corr 4.26]
    Extend language with new constant $e$ and let $\Gamma'$
    be augmented theory with the same non-logical axioms as $\Gamma$.
    Then for any variable $y$ we have $\Gamma' \vdash A[e]$ iff $\Gamma \vdash A[y]$.
  \item[Remark 4.27]
    Let the free variables of $A$ and $B$ be included in $\{x_1,\ldots,x_n\}$,
    and $\Gamma'$ be obtained by adding $n$ new constants $e_1,\ldots,e_n$.
    \\
    Then from $\Gamma',A[e_1,\ldots,e_n/x_1,\ldots,x_n]
               \vdash B[e_1,\ldots,e_n/x_1,\ldots,x_n]$
    we can get $\Gamma \vdash A \implies B$.
  \item[Meta 4.28]
    From $\vdash A$ we get $\vdash A[p:=W]$ for any $W$ provided no free var of $W$
    becomes bound during the substitution.
\end{description}

\newpage
\subsection{\UTP2\ Logic Syntax}

We now define our logic syntax as follows:
\\\textbf{NEEDS REVISING TO USE UTP2LogicDefs TO KEEP THINGS CONSISTENT}
\RLEQNS{
   x \in Var && (given) & \mbox{Obs. Variables}
\\ k \in Const && (given) & \mbox{Constants}
\\ f \in Names && (given) & \mbox{(Function) Names}
\\ E \in ENames && (given) & \mbox{Expression Metavariable Names}
\\ P \in PNames && (given) & \mbox{Predicate Metavariable Names}
\\ e \in Expr &::=& k | x  & \mbox{Expressions}
\\              &|& f~e & \mbox{Applications}
\\              &|& \lambda x @ e & \mbox{Obs. Abstraction}
\\              &|& \Lambda E @ e & \mbox{E-var. Abstraction}
\\              &|& \Lambda P @ e & \mbox{P-var. Abstraction}
\\              &|& \setof{ x | p @ e } & \mbox{Comprehension}
\\              &|& E & \mbox{Explicit Metavariable}
\\              &|& e[ e/x ] & \mbox{Explicit Obs. Substitution}
\\              &|& e[ e/E ] & \mbox{Explicit E-var. Substitution}
\\              &|& e[ p/P ] & \mbox{Explicit P-var. Substitution}
\\ p \in Pred &::=& True | False & \mbox{Constants Predicates}
\\              &|& e & \mbox{Atomic Predicate (Boolean-valued Expr.)}
\\              &|& \lnot p & \mbox{Negation}
\\              &|& p~ \maltese q & \mbox{Composites, }\maltese \in \setof{\land,\lor,\implies,\equiv}
\\              &|& P & \mbox{Explicit Metavariable}
\\              &|& \yen x @ p & \mbox{1st-order Quantifiers, } \yen \in \setof{\forall,\exists,\exists!}
\\              &|& \yen P @ p & \mbox{higher-order Quantifiers, } \yen \in \setof{\mathbf\forall,\mathbf\exists}
\\              &|& \yen E @ p & \mbox{higher-order Quantifiers, } \yen \in \setof{\mathbf\forall,\mathbf\exists}
\\              &|& p[ e/x ] & \mbox{Explicit Obs. Substitution}
\\              &|& p[ e/E ] & \mbox{Explicit E-var. Substitution}
\\              &|& p[ p/P ] & \mbox{Explicit P-var. Substitution}
}

\newpage
\subsection{The Deduction Theorem}\label{ssec:Deduction}

A key part of the logic we present involves the validity
of the Deduction Theorem:
\begin{eqnarray*}
   \INFER%
     {\Gamma,P \infer Q}%
     {\Gamma \infer P \implies Q}
\end{eqnarray*}
The implementation of this needs to be done very carefully.
Given a variable denoting a predicate (\texttt{Pvar}),
we recognise two distinct cases:
\begin{itemize}
  \item (Closed \texttt{Pvar}) the variable is a shorthand name for some other predicate,
   as indicated by an entry in a predicate table;
  \item (Open \texttt{Pvar}) the variable denotes an arbitrary predicate.
\end{itemize}
During matching, the former (closed) only matches a \texttt{Pvar} of the same name,
whilst the latter (open) matches any predicate at all.
We shall indicate an open Pvar by underlining it in the sequel,
so in $\underline P \land GROW$, \verb'Pvar "P"'
is open and \verb'Pvar GROW' is closed.

A law with open \texttt{Pvar}s effectively denotes an infinite law schema,
one law instance for every possible instantiation of them.
We need to be careful when invoking the deduction theorem,
to ensure that every assumption made has its open \texttt{Pvar}s closed.
by the simple expedient of uniformly renaming them to be fresh
and then entering a mapping from those renamed \texttt{Pvar}s
to themselves in the predicate table.
Once a predicate with open Pvars is assumed true for a proof,
all the Pvars must stop denoting arbitrary predicates and
start denoting just themselves.

Failure to do this results in an unsound proof, as exemplified by the following ``proof''
of the (non-)``theorem'':
$$
 \underline A \implies \underline B
$$
(where $A$ and $B$ are open).

(non-)proof:
Assume $H1: \underline A$, with goal $ \underline B $
\begin{eqnarray*}
  && \underline B
\EQV{$H1$ with $\underline A \mapsto \underline B$,
 as $\underline A$ in $H1$ is open}
\\&& True
\end{eqnarray*}
Now, if
we use an overline to denote the ``closing''%
\footnote{
By ``closing'' we mean renaming to something fresh,
and then adding a self-definition to the appropriate table.
}
of a predicate,
in which all open \texttt{Pvar}s become closed,
then the deduction theorem becomes:
\begin{eqnarray*}
   \INFER%
     {\Gamma,\overline P \infer \overline Q}%
     {\Gamma \infer P \implies Q}
\end{eqnarray*}
The non-proof above now fails:
Assume $H1: \overline A$, with goal $ \overline B $
\begin{eqnarray*}
  && \overline B
\\ &\not\equiv& \coz{We can't match against $H1$,
as we have to match $\overline B$ against $\overline A$}
\\&& True
\end{eqnarray*}
The following proof works just fine:
$$
  \underline A \implies \underline A \lor \underline B
$$
Assume $H1: \overline A$, with goal $\overline A \lor B$:
\begin{eqnarray*}
  && \overline A \lor B
\EQV{$H1$, where $\overline A$ (closed) matches $\overline A$}
\\&& True \lor B
\EQV{$\lor$-zero}
\\&& True
\end{eqnarray*}

All of this is basically how we implement Remark 4.27 (\S\ref{sssec:1EL:meta}) for the Tourlakis logic.
