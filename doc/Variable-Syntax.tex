A variable has a \emph{name}.

A variable will have one of the following \emph{kinds}:
\begin{description}
  \item[Observational]
    Corresponds to some observation that can be made of a program,
    and hence belongs, in UTP,  to the alphabet of the corresponding predicate.
  \item[Schematic]
    Stands for an arbitrary chunk of abstract syntax,
    of which we recognise two broad classes:
    \begin{description}
      \item[Expression] denotes an arbitrary expression
      \item[Predicate] denotes an arbitrary predicate
    \end{description}
  \item[Script]
    Represents an actual program variable itself,
    rather than any associated value it may take.
  \item[List]
   In a context where a list of variables is required,
   this can stand for zero or more variables,
   all belonging to precisely one of the categories detailed above.
\end{description}
The above kinds all have different roles in the logic underlying UTP.

In addition, with most of the variable kinds above,
we can associate one of the following \emph{roles}:
\begin{description}
  \item[Static]
    variables whose values a fixed for the life of a (program) behaviour
  \item[Dynamic]
    variables whose values are expected to change over the life of a behaviour.
    \begin{description}
      \item[Pre]
        variables that record the values taken when a behaviour starts
      \item[Post]
        variables that record the values taken when a behaviour ends
      \item[Intermediate]
        variables that record values that arise between successive behaviours
    \end{description}
\end{description}
These distinct roles do not effect how the underlying logic handles
variables, but are used to tailor definitional shorthands that
assume that these are enacting the relevant UTP variable conventions.

WE SHALL REVISIT SYNTACTICAL ISSUES LATER.
