An assertion we hope is true is a \emph{conjecture},
whilst one that is ``self-evidently true'' (i.e. an axiom)
or demonstrated to be so (by proof), is called a \emph{law}.
With laws, we also want to associate a \emph{provenance}
indicating the origin of
such a law:
\begin{description}
  \item[SOURCE] The law is generated internally by this program
   and has (hopefully) been subject to extensive checking and validation.
   With this provenance we associate a name identifying the source
   of the law
  \item[PROOF] The law is as a result of a proof,
    and its trustworthiness is related to that of
    the least trustworthy step in that proof
    (again, we may consider some steps to
    be more trusted than others,
       e.g. the matching against a simple law vs.
       using a complex built-in simplifier).
  \item[USER] The law was specified by a user,
    exploiting the experimental nature of this tool.
    Clearly trustworthiness here is low, and dependent on the user.
    \begin{description}
      \item[CONSERVATIVE] The law was specified by a user
      as the \emph{sole} definition of the meaning of a language construct,
      satisfying the following criteria:
      \begin{itemize}
        % \item The language specification uses no list elements
        \item The law name is ``\texttt{DEF }\emph{\texttt{name}}'',
        where \emph{\texttt{name}} is the language construct name.
        \item The law has the form $LHS \equiv RHS$
        \item The $LHS$ is an instance of the language specification
         with every element given a unique metavariable
         (so it matches any instance of the construct).
         \\Note: any list is represented by a single ``multi-''metavariable.
        \item The $RHS$ makes no mention or use of the language construct
         itself.
      \end{itemize}
      Such laws are conservative definitions and are as safe
      as anything else in the system.
      In effect a construct defined by a law as above is simply
      a macro, or syntactic-sugar for some pre-existing predicate.
    \end{description}
\end{description}
