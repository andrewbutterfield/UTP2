We shall define a normal form  for side-conditions as follows:
\begin{itemize}
  \item
    $\TRUE$ and $\FALSE$ are normal forms
    (and this is the only way in which they can arise in a normal form)
  \item
    $\bigwedge S$ is a normal form if each element of $S$
    is a \emph{single-MV normal-form}, each over a distinct meta-variable.
  \item
    A single-MV normal-form over meta-variable $M$
    either has the form $\isFresh M$,
    or is a \emph{minimal} conjunction over terms of the following forms:
     $\isCond M,~~M \DISJ V,~~M=V,~~M \subseteq V$,
     subject to the following constraints:
     \begin{itemize}
       \item
         the terms are consistent
         (otherwise the top-level normal form should simply be $\FALSE$).
       \item
         the terms are minimal in the sense that no one term implies another,
         and indeed terms are merged to the greatest extent possible.
     \end{itemize}
\end{itemize}
Terms can be inconsistent/merged in a variety of ways:
\begin{eqnarray*}
   M=V_1 \land M=V_2  &\equiv& M = V_1 \cond{V_1 = V_2} \FALSE
\\ M=V_1  \land M \subseteq V_2 &\equiv& M = V_1 \cond{V_1 \subseteq V_2} \FALSE
\\ M=V_1  \land M \DISJ V_2 &\equiv& M = V_1 \cond{V_1 \DISJ V_2} \FALSE
\\ M \subseteq V_1  \land M \subseteq V_2 &\equiv& M \subseteq V_1 \cap V_2
\\ M \subseteq \emptyset  &\equiv& M = \emptyset
\\ M \subseteq V_1  \land M \DISJ V_2 &\equiv& M \subseteq (V_1 \setminus V_2)
\\ M \DISJ V_1 \land M \DISJ V_2 &\equiv& M \DISJ (V_1 \cup V_2)
\end{eqnarray*}
We see that terms of the form $M = V_i$ subsume all others,
because we can always evaluate a consistency condition over
other $V_i$ present.
Terms of the form $M \subseteq V_i$ can be merged using set intersection
over the $V_i$ whilst $M \DISJ V_i$ can be merged with set union.
Also a term of the form $M \subseteq V_1$ subsumes
any term of the form $M \DISJ V_2$, if $V_1$ and $V_2$ are disjoint.
\emph{In fact $M \subseteq V$ can always be tweaked to subsume $M \DISJ U$
(simply replace it with $M \subseteq (V \setminus U)$).
}
We also look at the interaction between $\isCond M$
and the above forms:
\begin{eqnarray*}
   M=V\land \isCond M &\equiv&  M=V \cond{\isCond V} \FALSE
\\ M \subseteq V\land \isCond M &\equiv& M \subseteq V\!|_{\isCond{\_}}
\\ S\!|_p &\defs& \setof{x | x \in S \land p(x)}
\\ M \DISJ V\land \isCond M &\equiv& M \DISJ V\land \isCond M
\end{eqnarray*}
So, a single-MV normal form can have one of the following configurations:
$M = V, \quad M \subseteq V, \quad M \DISJ V$
each possibly conjoined with $\isCond M$.

Freshness is very strict and rules out a lot:
\begin{eqnarray*}
   \isFresh V  &\implies& V\DISJ M  \mbox{ for any } M
\end{eqnarray*}
Here we assume $V$ is non-empty.
