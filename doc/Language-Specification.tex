The textual form of the language specification
is one that matches how the construct would be written,
as an interleaving of language elements (basic and list)
with tokens, which may be absent.
The following character have special roles:
\begin{description}
  \item[Basic Elements]
     letters \texttt{V}, \texttt{T}, \texttt{E} and \texttt{P}
  \item[List Element]
  letter \texttt{*} or \texttt{\#} immediately after a basic element.
  \item[Whitespace] ignored/skipped
\end{description}
Anything else is interpreted as a token,
even if it contains the above special characters.
The only error is if two tokens occur one after another,
with everything else being interpreted as a valid language specification.

For illustration, here are some specifications
that correspond to well-known language constructs:

\begin{tabular}{|l|c|c|}
  \hline
    Construct & Specifier & Example
  \\\hline
    Logical-And & \texttt{P*/\BS} &  \texttt{P /\BS Q /\BS R}
  \\\hline
    Logical-Or & \texttt{P*\BS/} &  \texttt{P \BS/ Q \BS/ R}
  \\\hline
    Pred. Forall & \texttt{Forall V,* @ P} & \texttt{Forall P,Q @ P => Q}
  \\\hline
    Assignment & \texttt{V := E} & \texttt{x := y + z}
  \\\hline
    Sim.--Assignment & \texttt{V\#, := E\#,}
      & \texttt{x,y := y + z,y-1}
  \\\hline
\end{tabular}
