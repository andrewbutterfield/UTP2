In determining free observation variables we have to deal with the
presence of meta-variables denoting arbitrary predicates and expressions,
as well as the presence of explicit substitutions, and quantifier list-variables
that can denote lists of variables or expressions.
This means that the free variables of a predicate or expression
are contingent on not just the meta/list-variables,
but whether or not certain observation variables are free/present in (any
instantiation of) those meta/list-variables.

We shall use the lambda calculus to illustrate the consequences
of having explicit substitution and quantifier-list matching in our logic.

We start with the untyped lamdba-calculus ($L_0$) where we have an unbounded set of variables ($v \in V$):
\begin{eqnarray*}
   v,w,x \in V && \mbox{Given.}
\\ e,f \in L_0 & ::=& v | e~e| \lambda v @ e
\end{eqnarray*}
with $\lambda x_1,x_2,\ldots,x_n @ e$
as syntactic sugar for
$\lambda x_1 @ \lambda x_2 @ \ldots \lambda x_n @ e$,
itself often shortened to $\lambda \vec x @ e$.

Notation aside: We shall assume that $\vec a$ is shorthand for $a_1,\ldots,a_n$,
for $n \geq 0$, and $a_i$ will refer to the $i$th component, or indicate
an iteration $i \in 1\ldots n$, depending on context.

The free variables for $L_0$ are defined in the usual way:
\begin{eqnarray*}
   S_0 &=& \power V
\\ \fv,\fv_0 &:& L_0 \fun S_0
\\ \fv(v) &\defs& \setof v
\\ \fv(f~e) &\defs& \fv(f) \union \fv(e)
\\ \fv(\lambda v @ e) &\defs& \fv(e) \setminus \setof v
\end{eqnarray*}

So is substitution:
\begin{eqnarray*}
   \_[\_:=\_] &:& L_0 \fun V \times L_0 \fun L_0
\\ v[x:=e] &\defs&  e \cond{x=v} v
\\ (f~e)[x:=e] &\defs& (f[x:=e])~(e[x:=e])
\\ (\lambda v @ f)[x:=e]
   &\defs&
   \left\{
     \begin{array}{ll}
       \lambda v @ f, & x=v \\
       \lambda v @ f[x:=e], & x\neq v \land v \notin \fv(e)  \\
       \lambda w @ f[v:=w][x:=e], & x\neq v \land v \in \fv(e) \land \mbox{new}~w
     \end{array}
   \right.
\end{eqnarray*}
This extends to simultaneous substitutions as follows
(here $\vec x$ has no duplicates):
\begin{eqnarray*}
   \_[\_:=\_] &:& L_0 \fun (V^n \times (L_0)^n) \fun L_0
\\ e[\nil:=\nil] &\defs& e, \mbox{ the empty case}
\\ v[\vec x:=\vec e] &\defs&  e_i \cond{v=x_i} v, \mbox{ for some }i
\\ (f~e)[\vec x:=\vec e] &\defs& (f[\vec x:=\vec e])~(e[\vec x:=\vec e])
\\ (\lambda v @ f)[\vec x:=\vec e]
   &\defs&
   \left\{
     \begin{array}{ll}
       \lambda w @ f[v:=w][\vec x\setminus x_i:=\vec e\setminus e_i], & v=x_i\\
       \lambda w @ f[v:=w][\vec x:=\vec e], & v\notin\vec x\\
     \end{array}
   \right. \mbox{new}~w
\end{eqnarray*}
This then leads to key theorems regarding free variables and substitution:
\begin{eqnarray}
   \fv(e[x:=f])
   &=&
   \fv(e)\setminus\setof x \cup (\fv(f) \cond{x \in \fv(e)} \emptyset)
\\ \fv(e[\vec x:=\vec e])
   &=&
   \fv(e)\setminus\vec x
   \cup
   \bigcup_i\setof{\fv(e_i) \cond{x_i \in \fv(e)} \emptyset}
\end{eqnarray}
The first is provable, with care, by induction on $e$
(even the base-case is non-trivial !).

We then extend our language to include explicit expression meta-variables ($E \in M$)
and explicit simultaneous substitutions:
\begin{eqnarray*}
   E \in M && \mbox{Given.}
\\ e,f \in L_1 &::\!\!+&  E | e[\vec f/\vec x]
\end{eqnarray*}
Here it is understood that vectors $\vec f$ and $\vec x$ are of the same length
and the latter has no duplicates.

We now extend the notion of free variables, but note that we cannot ``expand'' the
application of $\fv$ to $E$.
The effect of this is that we no longer can return a set of variables, but must instead
return an expression ($s \in S_1$) denoting such a set as a function of its explicit meta-variables:
\begin{eqnarray*}
  s \in S_1 &::=& \setof{\vec v} | E | s \setminus s | \bigcup \setof{\vec s} | v \in s \sthen s
\end{eqnarray*}
The last construction is a conditional: for $v \in s_0 \sthen s_1$,
if the condition holds, then it denotes $s_1$, otherwise it denotes the empty set.

There are obvious injections $L^{01}$ and $S^{01}$ that embed $e_0 : L_0$  and $s_0 : S_0$
into $L_1$ and $S_1$ respectively.

Given an environment ($\rho_M$) mapping $E$ to values from $L_0$,
we can then map both $L_1$ to $L_0$ and $S_1$ to $S_0$ as follows
(where we drop ${}_M$ in most cases as it is obvious from context):
\begin{eqnarray*}
   \rho_M &:& M \fun L_0
\\ L^{10} &:& (M \fun L_0) \fun  L_1 \fun L_0
\\ L^{10}_\rho(v) &\defs& v
\\ L^{10}_\rho(f~e) &\defs& (L^{10}_\rho(f))~(L^{10}_\rho(e))
\\ L^{10}_\rho(\lambda v @ e) &\defs& \lambda v @ (L^{10}_\rho(e))
\\ L^{10}_\rho(E) &\defs& \rho(E)
\\ L^{10}_\rho(e[\vec f/\vec x]) &\defs& (L^{10}_\rho(e))[\vec x:=\vec f]
\\
\\ S^{10} &:& (M \fun L_0) \fun S_1 \fun S_0
\\ S^{10}_\rho(\setof{\vec v}) &\defs& \setof{\vec v}
\\ S^{10}_\rho(E) &\defs& \fv_0(\rho(E))
\\ S^{10}_\rho(s_1 \setminus s_2) &\defs& S^{10}_\rho(s_1) \setminus S^{10}_\rho(s_2)
\\ S^{10}_\rho(\bigcup\setof{\vec s}) &\defs&  \bigcup\setof{\vec{S^{10}_\rho(s)}}
\\ S^{10}_\rho(v \in s_0 \sthen s_1)
    &\defs&
     S^{10}_\rho(s_1)
    \cond{v \in S^{10}_\rho(s_0)}
     \emptyset
\end{eqnarray*}

We can extend $\fv$ to $L_1$ as follows:
\begin{eqnarray*}
   \fv,\fv_1 &:& L_1 \fun S_1
\\ \fv(x) &\defs& \setof x
\\ \fv(e_1~e_2) &\defs& \bigcup( \fv e_1,\fv e_2 )
\\ \fv(\lambda x @ e) &\defs& (\fv e) \setminus \setof x
\\ \fv(E) &\defs& E
\\ \fv (e[\vec f/\vec x])
   &\defs&
   \fv(e)\setminus\setof{\vec x}
   \cup
   \bigcup_i \setof{ x_i \in \fv(e) \sthen \fv(f_i)}
\end{eqnarray*}
For a substitution, we see that the presence of the free variables
of a replacement expression ($e_i$) is contingent
on the presence of the corresponding \emph{target} variable ($x_i$)
in the free variables of the \emph{base} expression ($e$).
If the base expression is a meta-variable, then we get the following
instantiation of the last law:
\begin{eqnarray*}
   \fv(E)[e_1,\ldots,e_n/x_1,\ldots,x_n]
   &=& E\setminus\setof{x_1,\ldots,x_n}
       \union
       \bigcup \setof{  x_i \in  E @ \sthen e_i }
\end{eqnarray*}
We cannot either perform the set difference operation,
nor evaluate any of the conditionals.
In effect, in order to give an accurate description of the free variables
of this language we need to return a (variable-set valued) expression
that describes how the resulting set of variables is contingent
upon the (yet to be determined) free variables of the meta-variables.
This is the motivation for the $\sthen$ construct in $S_1$.



We now find that we can construct the following diagram relating the $L_i$ and $S_i$:
\begin{center}
 \begin{tikzcd}[row sep=4em,column sep=4em]
     L_1 \rar{\fv_1} \dar[bend left]{L^{10}_\rho}
 &   S_1 \dar[bend left]{S^{10}_\rho}
 \\
     L_0 \rar[swap]{\fv_0} \uar[hook,bend left]{L^{01}}
  &  S_0 \uar[hook,bend left]{S^{01}}
 \\
 \end{tikzcd}
\end{center}
From this we can immediately suggest a few lemmas/theorems:
\begin{eqnarray}
   L^{10}_\rho(L^{01}(e_0)) &=& e_0, \mbox{ for all }\rho
\\ S^{10}_\rho(S^{01}(s_0)) &=& s_0, \mbox{ for all }\rho
\\ \fv_1(L^{01}(e_0)) &=& S^{01}(\fv_0(e_0))
\\ S^{10}_\rho(\fv_1(e_1)) &=& \fv_0(L^{10}_\rho(e_1)), \mbox{ for all }\rho
\end{eqnarray}
Proofs of these are by induction over the leftmost $e_i$ in each case,
and are best done in the order given above.
The following is an easy consequence of the above:
\begin{eqnarray*}
   && S^{10}_\rho(\fv_1(L^{01}(e_0)))
\\ &=& S^{10}_\rho(S^{01}(\fv_0(e_0)))
\\ &=& \fv_0(e_0)
\end{eqnarray*}

At this point we need to extend the language further to have explicit
quantifier meta-variables that stand for lists of ordinary variables
or corresponding lists of expressions:
\begin{eqnarray*}
  \lst q,\lst r \in Q && \mbox{Given.}
\end{eqnarray*}
We then extend our language again (where $\lstvec q$, like $\vec x$, has no duplicates):
\begin{eqnarray*}
   e,f \in L_2 &::\!\!+& \lambda \vec v,\lstvec q @ e | e[\vec f,\lstvec r/\vec x,\lstvec q]
\end{eqnarray*}
As before, there is an obvious injection $L^{12} : L_1 \fun L_2$,
as well as $L^{02} : L_0 \fun L_2$,
and the syntactic sugar $\lambda \vec x @ e$  is now a proper part of $L_2$.
In effect the $L_2$ extensions subsume the lambda and substitution
constructs of $L_1$.

The $\lst q$ and $\lst r$ are intended to denote lists (possibly empty)
of ordinary variables ($\vec v$) and expressions ($\vec f$) respectively.
To this end we introduce two new environments:
\begin{eqnarray*}
   \rho_V &:& Q \fun V^*
\\ \rho_E &:& Q \fun L_1^*
\end{eqnarray*}
We can use these to define the conversion
$$L^{21}_{(\rho_V,\rho_E)} : L_2 \fun L_1,$$
and coupled with $\rho_M$, we can then get
$$L^{20}_{(\rho_V,\rho_E,\rho_M)} : L_2 \fun L_0.$$
In the sequel we shall often use $\rho$ to denote one of the above
when it is  obvious from context,
or to denote the entire tuple-parameter of a conversion: i.e. $L^{20}_\rho$.

We can now define the conversion:
\begin{eqnarray*}
   L^{21} &:& (Q \fun V^*) \times(Q \fun L_1^*) \fun  L_2 \fun L_1
\\ L^{21}_\rho(v) &\defs& v
\\ L^{21}_\rho(f~e) &\defs& (L^{21}_\rho(f))~(L^{21}_\rho(e))
\\ L^{21}_\rho(E) &\defs& E
\\ L^{21}_\rho(\lambda \vec v,\lstvec q @ e)
   &\defs&
   \lambda \vec v\cat\vec{\rho_V(\lst q)} @ (L^{21}_\rho(e))
\\ L^{21}_\rho(e[\vec f,\lstvec r/\vec x,\lstvec q])
   &\defs&
   (L^{21}_\rho(e))
     [ \vec{L^{21}(f)}\cat\vec{\rho_E(\lst r)}
       /
       \vec x \cat \vec{\rho_V(\lst q)}
     ]
\end{eqnarray*}
Inbuilt here, are assumptions about the lengths of various lists matching
up in $[\vec f,\lstvec r/\vec x,\lstvec q]$:
\begin{eqnarray*}
   \len\vec f &=& \len\vec x
\\ \len\lstvec r &=& \len\lstvec q
\\ \len(\rho_E(\lst r_i)) &=& \len(\rho_V(\lst q_i))
\end{eqnarray*}

We now need to extend the variable set language to cope with the $Q$ extensions
(again we have $S_2$ subsuming one $S_1$ component, and adding a new one:
\begin{eqnarray*}
   s \in S_2 &::\!\!+& \setof{\vec v,\lstvec q} | \lst q \in s \ssthen \lst r
\end{eqnarray*}
Here we introduce the $\ssthen$ symbol,
similar to $\sthen$,
but note that the contents
and meaning are different: $\lst q$ will denote a list of variables ($\vec v$),
and $\lst r$ will denote a list of sets ($\vec s$), of the same length.
The overall value will be merging all $s_i$ where $v_i$ is in $s$.
Again,the embedding $S^{12} : S_1 \fun S_2$ should be obvious.

Now, the opposite conversion:
\begin{eqnarray*}
   S^{21} &:& (Q \fun V^*) \times(Q \fun L_1^*) \fun S_2 \fun S_1
\\ S^{21}_\rho(\setof{\vec v,\lstvec q}) &\defs& \setof{\vec v\cat\vec{\rho_V(\lst q)}}
\\ S^{21}_\rho(E) &\defs& E
\\ S^{21}_\rho(s_1 \setminus s_2) &\defs& S^{21}_\rho(s_1) \setminus S^{21}_\rho(s_2)
\\ S^{21}_\rho(\bigcup\setof{\vec s}) &\defs&  \bigcup\setof{\vec{S^{21}_\rho(s)}}
\\ S^{21}_\rho(v \in s_0 \sthen s_1)
   &\defs&
   v \in S^{21}_\rho(s_0) \sthen S^{21}_\rho(s_1)
\\ S^{21}_\rho(\lst q \in s \ssthen \lst r)
    &\defs&
    \bigcup_i
    \setof{  \rho_V(\lst q)_i \in S^{21}_\rho(s) \sthen \fv_1(\rho_E(\lst r)_i) }
\end{eqnarray*}

We are now in a position to define free variables over $L_2$:
\begin{eqnarray*}
   \fv,\fv_2 &:& L_2 \fun S_2
\\ \fv(x) &\defs& \setof x
\\ \fv(e_1~e_2) &\defs& \bigcup( \fv e_1,\fv e_2 )
\\ \fv(\lambda \vec x,\lstvec q @ e) &\defs& (\fv e) \setminus \setof{\vec x,\lstvec q}
\\ \fv(E) &\defs& E
\\ \fv (e[\vec f,\lstvec r/\vec x,\lstvec q])
   &\defs&
   \fv(e)\setminus \setof{\vec x,\lstvec q}
\\ && {} \cup
   \bigcup_i
   \setof{
     x_i \in \fv(e) \sthen \fv(f_i)
   }
\\ && {} \cup
   \bigcup_j
   \setof{
     \lst q_j \in \fv(e) \ssthen \lst r_j
   }
\end{eqnarray*}

Once more, we have a commuting square:
\begin{center}
 \begin{tikzcd}[row sep=4em,column sep=4em]
     L_2 \rar{\fv_2} \dar[bend left]{L^{21}_\rho}
 &   S_2 \dar[bend left]{S^{21}_\rho}
 \\
     L_1 \rar[swap]{\fv_1} \uar[hook,bend left]{L^{12}}
  &  S_1 \uar[hook,bend left]{S^{12}}
 \\
 \end{tikzcd}
\end{center}
and corresponding theorems:
\begin{eqnarray}
   L^{21}_\rho(L^{12}(e_1)) &=& e_1, \mbox{ for all }\rho
\\ S^{21}_\rho(S^{12}(s_1)) &=& s_1, \mbox{ for all }\rho
\\ \fv_2(L^{12}(e_1)) &=& S^{12}(\fv_1(e_1))
\\ S^{21}_\rho(\fv_2(e_2)) &=& \fv_1(L^{21}_\rho(e_2)), \mbox{ for all }\rho
\end{eqnarray}

At this point we consider normal forms for free-variable
set expressions, which require three further
language extensions:
we extend membership to include conjunction,
and introduce a shorthand for the empty conjunction (true).
We also allow explicit subtractions on the rhs of $\ssthen$:
$$\begin{array}{rcll}
  m \in Member  & ::= & v \in s | \lst q \in s | \lst q \in \lst r & \mbox{Element Membership}
\\              &  |  & \bigwedge (m_1,\ldots,m_n) & \mbox{Conjunction}
\\ \top &\defs& \bigwedge()
\\ s \in S_3 &::\!\!+& m \ssthen \lst r \setminus \setof{\vec v,\lstvec q}
\end{array}$$
Note that we also allow a membership predicate of the form $\lst q \in s$
to appear in $\sthen$, with a different semantics to the same thing in $\ssthen$:
\begin{eqnarray*}
   S^{21}_\rho(\lst q \in s_1 \sthen s_2)
    &\defs&
    \bigcup_i
    \setof{  \rho_V(\lst q)_i \in S^{21}_\rho(s_1) \sthen s_2 }
\end{eqnarray*}
In addition we allow $\lst q \in \lst r$ to appear in $\ssthen$,
where we break the need for corresponding indices of $\lst q$ and $\lst r_1$ to match,
but retain it between $\lst q_i$ and $(\lst r_2)_i$:
\begin{eqnarray*}
   S^{21}_\rho(\lst q \in \lst r_1 \ssthen \lst r_2)
    &\defs&
    \bigcup_i
    \setof{  \rho_V(\lst q)_i \in\fv_1(\rho_E(\lst r_1)) \sthen \fv_1(\rho_E(\lst r_2)_i) }
\end{eqnarray*}

We also introduce
a form of conjunction for membership predicates in $\sthen$
and the notion of  a naked $Q$ variable $\lst r$ denoting a list of expressions:
\begin{eqnarray*}
   s \in S_3 &::\!\!+& m \sthen s |  \lst r
\end{eqnarray*}
Embedding $S^{23} : S_2 \fun S_3$ is obvious,
whilst the conversion in the opposite direction is less so:
\begin{eqnarray*}
   S^{32} &:& S_3 \fun S_2
\\ S^{32}(\bigwedge () \sthen s)
   &\defs&
   s
\\ S^{32}(\bigwedge (m_1,\ldots,m_n) \sthen s)
   &\defs&
   m_1 \sthen (m_2 \sthen \ldots (m_n \sthen s)\ldots)
\\ S^{32}(\lst r) &\defs& \lst q \in \setof \lst q \ssthen \lst r
\end{eqnarray*}

So, we can use $\fv_2$ to get the desired free-variable set for any instance of $L_2$,
as a member of $S_2$,
and then embed into $S_3$ for normalisation and reasoning.

\newpage
At this point it is worth writing out $S_3$ explicitly (as $S$),
and fine-tuning the syntax of membership:
\begin{eqnarray*}
   s \in S &::=& \setof{\vec v,\lstvec q}
\\ &|& E
\\ &|& s \setminus s
\\ &|&  \bigcup \setof{\vec s}
\\ &|&  m \sthen s
\\ &|&  \lst q \in s \ssthen \lst r \setminus \setof{\vec v,\lstvec q}
\\ &|&  \lst r
\\ m \in Mmbr  & ::= & v \in s | \lst q \in s  | \bigwedge (m_1,\ldots,m_n)
\end{eqnarray*}
It is worth stressing how the constructs $\lst r$, $\sthen$ and $\ssthen$ should be interpreted:
\begin{itemize}
  \item As a standalone,
        $\lst r$ denotes the union of the free-variables of the list of expressions given by $\rho_E$.
        $$\sem{\lst r} = \bigcup_i\setof{\fv(\rho_E(\lst r)_i)}$$
  \item
     The construct $(v|\lst q) \in s \sthen t$ equals $t$ if all of $v$ or $\lst q$ is contained in $s$.
     If either $s$ or $t$ is an instance of $\lst r$ then it is viewed as per the previous bullet-point.
     \begin{eqnarray*}
        \sem{v \in s \sthen t} &=& \sem{t} \cond{v \in \sem{s}} \emptyset
     \\\sem{\lst q \in s \sthen t} &=& \sem{t} \cond{\lst q \subseteq \sem{s}} \emptyset
     \end{eqnarray*}
  \item The construct $\lst q \in s \ssthen \lst r \setminus \setof{\vec v,\lstvec q}$ is well-formed only if $\len\rho_V(\lst q)=\len\rho_E(\lst r)$,
     and basically determines its result on a component-wise examination of the elements
     of $\lst q$.
     \begin{eqnarray*}
        \sem{\lst q \in s \ssthen \lst r\setminus \setof{\vec v,\lstvec q}}
        &=&
        \bigcup_{i=1}^{\len~\lst q}\setof{ \fv(\rho_E(\lst r)_i) \cond{\rho_V(\lst q)_i \in \sem{s}} \emptyset} \setminus \setof{\vec v,\lstvec q}
     \\ &=&
        \bigcup_{i=1}^{\len~\lst q}\setof{ \rho_V(\lst q)_i \in s \sthen \fv(\rho_E(\lst r)_i)} \setminus \setof{\vec v,\lstvec q}
     \end{eqnarray*}
     If $s$ is an instance of $\lst r$ then we use the interpretation in the first bullet point.
     The case $v \in s \ssthen \lst r$ is only valid if $\len\rho_E(\lst r)=1$.
\end{itemize}
In effect $\sthen$ is a global conditional, while $\ssthen$ works pointwise
on corresponding members of $\lst q$ and $\lst r$.

Note that if $\lst q$ and $\lst r$ have length one, then $\lst q \in s \sthen \lst r$ and $\lst q \in s \ssthen \lst r$
are the same.

\newpage
A conditional expression $m \sthen s$ is \emph{upfront}
if $s$ does not itself contain any conditionals,
noting that $m \ssthen \lst r$ are always up-front by construction.
A set-expression is \emph{atomic} if it has one of the
following forms ($a$ is $v$ or $\lst q$):
$$
  \setof{ a_1, \ldots, a_n}
  \qquad\qquad
  E \setminus \setof{ a_1 , \ldots, a_n }, ~  n \geq 0
  \qquad\qquad \lst r\setminus \setof{ a_1 , \ldots, a_n }, ~  n \geq 0
$$
We define our normal form to be a union of upfront conditional set-expressions
whose set components are atomic.
We convert to normal-form by repeatedly applying the following equivalences
left-to-right,
designed to bring union and conditional to the top
\begin{eqnarray*}
   m_1 \sthen (m_2 \sthen s) &=& \bigwedge(m_1,m_2) \sthen s
\\ (m \sthen s) \setminus \setof{a_i} &=& m \sthen (s \setminus \setof{a_i})
\\ m \sthen \bigcup( s_i ) &=& \bigcup( m \sthen s_i )
\\ m \sthen \setof{} &=& \setof{}
\\ \lst q \in (v \in s_0 \sthen s_1) \ssthen \lst r
   &=&
   v \in s_0 \sthen ( \lst q \in s_1 \ssthen \lst r)
\\ \lst q_1 \in (\lst q_2 \in s_2 \ssthen \lst r_2) \ssthen \lst r_1
   &=&
   \bigwedge(\lst q_1 \in \lst r_2,\lst q_2 \in s_2) \ssthen \lst r_1
\\ \bigwedge( m_i, a \in (m \sthen s_1) )\sthen s_2
   &=& \bigwedge( m_i, a \in s_1, m ) \sthen s_2
\\ \bigwedge( m_i, a \in \setof{a_i}, a \in \setof{a_j} )\sthen s_3
   &=& \bigwedge( m_i, a \in (\setof{a_i}\cap\setof{a_j}), m ) \sthen s_3
\\ \bigwedge( m_i, a \in \setof{} ) \sthen s &=& \setof{}
\\ a \in N \setminus \setof{a_i}
   &=& a \in N, \qquad a \notin \setof{a_i}
\\ &=& a \in \setof {}, \qquad a \in \setof{a_i}
\\ \bigwedge( m_i, a \in \bigcup( s_i ) )\sthen s
   &=& \bigcup( \bigwedge( m_i, a \in s_i ) \sthen s)
\\ \bigcup( s_i ) \setminus \setof{ a_i }
   &=&
   \bigcup( s_i \setminus \setof{ a_i} )
\\ (s \setminus \setof{ a_i} ) \setminus \setof{ a_j }
   &=&
   s \setminus \setof{ a_i, a_j }
\\ s &=& \bigcup( \top \sthen s )
\end{eqnarray*}
We assume that nested unions and conjunctions are flattened on-the-fly.

