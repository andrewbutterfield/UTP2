\section{\LaTeX\ Formatting}

In this section we describe the \LaTeX\ notation
used to input theory information into \Saoithin.
We base our description around the components of the \texttt{Theory}
(a.k.a. \texttt{ProofContext}) datatype.

\subsection{Theory-Related \LaTeX\ Environments}

Each major component of the ProofContext datatype
has its components described within
a specific \LaTeX\ environment,
which we list below.

\begin{description}
  \item[\texttt{ctxtName,ctxtSeqNo :: String,Int}]~\\
    We use a single macro within a \texttt{theoryid} environment
    to encode these two values:
    \begin{verbatim}
    \begin{theoryid}
    \theoryNameVersion{LaTeX-Test}{99}
    \end{theoryid}
    \end{verbatim}
  \item[\texttt{typedefs :: Trie Type}]~\\
    Type-definition maps a (type-)name to a type(-definition).
    \\\verb"\begin{typedef}"
    \\ \phantom{mm}\textit{\textsf{name1}} \verb" & \defs & "\textit{\textsf{type1}}
    \\ \verb"\\" \textit{\textsf{  name2}} \verb" & \defs & "\textit{\textsf{type2}}
    \\ \ldots
    \\\verb"\end{typedef}"
    \\ An issue that arises is how the parser distinguishes between
    a newline (double-backslash) acting as a definition separator
    from one acting to split a long righthand-side over multiples lines (e.g.):
    \\ \phantom{mm}\textit{\textsf{name1}} \verb" & \defs & "\textit{\textsf{a-long}}
    \\ \verb"\\        &       & "\textit{\textsf{type-expression}}
    \\ \verb"\\" \textit{\textsf{  name2}} \verb" & \defs & "\textit{\textsf{type2}}
    \\ Perhaps we need an explixit \LaTeX\ macro (\verb"\next" ?)
    to denote definition separation?
  \item[\texttt{consts :: Trie Expr}]~\\
    Constant-definition maps a (variable-)name to a ground expr(-definition).
    \\\verb"\begin{consts}"
    \\ \phantom{mm}\textit{\textsf{name}} \verb" & \defs & "\textit{\textsf{expr}}
    \\\verb"\end{consts}"
        \\(From here on, we only list a single element).
  \item[\texttt{exprs :: Trie Expr}]~\\
    Expression-definition maps a (expr-)name to a expr(-definition).
    \\\verb"\begin{exprs}"
    \\ \phantom{mm}\textit{\textsf{name}} \verb" & \defs & "\textit{\textsf{expr}}
    \\\verb"\end{exprs}"
  \item[\texttt{preds :: Trie Pred}]~\\
    Predicate-definition maps a (expr-)name to a predicate(-definition).
    \\\verb"\begin{preds}"
    \\ \phantom{mm}\textit{\textsf{name}} \verb" & \defs & "\textit{\textsf{pred}}
    \\\verb"\end{preds}"
  \item[\texttt{obs :: Trie Type}]~\\
    Observation table maps a (variable-)name to a type.
    \\\verb"\begin{obs}"
    \\ \phantom{mm}\textit{\textsf{name}} \verb" & : & "\textit{\textsf{type}}
    \\\verb"\end{obs}"
  \item[\texttt{precs :: Trie Precs}]~\\
    Precedences table maps a (operator-)name to precedence data.
    \\\verb"\begin{precs}"
    \\ \phantom{mm}\textit{\textsf{name}} \verb" & \defs & "\textit{\textsf{prec}}
    \\\verb"\end{precs}"
  \item[\texttt{types :: Trie Type}]~\\
    Type table maps a (variable-)name to its type.
    \\\verb"\begin{types}"
    \\ \phantom{mm}\textit{\textsf{name}} \verb" & : & "\textit{\textsf{type}}
    \\\verb"\end{types}"
  \item[\texttt{alphas :: Trie Alphabet}]~\\
    Alphabet table maps a (predicate-)name to its alphabet.
    \\\verb"\begin{alphas}"
    \\ \phantom{mm}\textit{\textsf{name}} \verb" & \hasalpha??? & "\textit{\textsf{alpha}}
    \\\verb"\end{alphas}"
  \item[\texttt{laws :: LawTable}]~\\
    Law table maps a (law-)name to laws
    \\\verb"\begin{laws}"
    \\ \phantom{mn}\textit{\textsf{name}} \verb" & \defs & "\textit{\textsf{law}}  (axiom)
    \\ \verb"\\"   \textit{\textsf{name}} \verb" &   =   & "\textit{\textsf{law}}  (theorem)
    \\\verb"\end{laws}"
  \item[\texttt{conjectures :: Trie Law}]~\\
    Conjecture table maps a (conjecture-)name to laws
    \\\verb"\begin{conjectures}"
    \\ \phantom{mm}\textit{\textsf{name}} \verb" & \defs & "\textit{\textsf{law}}
    \\\verb"\end{conjectures}"
  \item[\texttt{theorems :: Trie Proof}]~\\
    Conjecture table maps a (proof-)name to proof
    \\\verb"\begin{theorems}"
    \\ \phantom{mm}\textit{\textsf{name}} \verb" & \hasproof??? & "\textit{\textsf{proof}}
    \\\verb"\end{theorems}"
\end{description}

\newpage
\subsection{The \LaTeX\ Syntax of Types}

\begin{verbatim}
B                                \bool
Z                                \num
Tprod [Type]                     ... \cross ...
Tfun Type Type                   ... \fun  ...
Tpfun Type Type                  ... \pfun ...
Tmap Type Type                   ... \ffun ...
Tset Type                        \power ...
Tseq Type                        \seq ...
Tseqp Type                       \seq_1 ...
Tfree String [(String,[Type])]    ...  ::=  (... \ldata ... \rdata ) * |
Tvar String                      ...
Tenv                             \Env
Terror String                    ...
Tarb                             \tarb
\end{verbatim}


\newpage
\subsection{The \LaTeX\ Syntax of Expressions}

\begin{verbatim}
T
F
Num Int
Var String
Prod [Expr]
App String Expr
Bin String Int Expr Expr
Equal Expr Expr
Set [Expr]
Setc QVars Pred Expr
Seq [Expr]
Seqc QVars Pred Expr
Map [(Expr,Expr)]
Cond Pred Expr Expr
Build String [Expr]
The QVars Pred Pred -- definite description
Evar String -- meta-variable denoting an expression
Eqvar String  -- meta-variable denoting a quantified variable
Eabs QVars Expr
Esub Expr ESubst
Eerror String
Efocus Expr [(Expr,Int)] -- focus, parent-list
\end{verbatim}


\newpage
\subsection{The \LaTeX\ Syntax of Predicates}

\begin{verbatim}
TRUE
FALSE
Obs Type Expr -- type is "underlying type" (See Types.lhs).
Defd Expr
TypeOf Expr Type
Not Pred
And [Pred]
Or [Pred]
Imp Pred Pred
Eqv Pred Pred
Forall QVars Pred Pred
Exists QVars Pred Pred
Exists1 QVars Pred Pred
Univ Pred
Sub Pred ESubst
Pvar String -- predicate meta-variable
>  -- UTP fundamentals
If Pred Pred Pred
Comp Pred Pred
NDC Pred Pred
RfdBy Pred Pred
LPredPred String Int Pred Pred -- e.g  Pre |- Post
LVarExpr String Int String Expr --  e.g.  x := e
LExprPred String Int Expr Pred  --  e.g.  c * P,  a --> P
LPredExprPred String Int Pred Expr Pred -- e.g   P [| S |] Q
Peabs String Pred  -- abstracting over an expression variable
Ppabs String Pred  -- abstracting over a predicate variable
Papp Pred Pred
Pfocus FContext Pred [(Pred,Int,FContext)] -- focus, ancestors
\end{verbatim}


\newpage
\subsection{The \LaTeX\ Syntax of Side-Conditions}

\begin{verbatim}
SCtrue
SCisCond String -- no dashed variables
SCnotFreeInP [String]  String -- Qvar, Pvar
SCnotFreeInE [String] String -- Qvar, Eevar
SCareTheFreeOfP [String] String -- Qvar, Pvar
SCcoverTheFreeOfP [String] String -- Qvar, Pvar
SCAnd [SideCond]
\end{verbatim}

