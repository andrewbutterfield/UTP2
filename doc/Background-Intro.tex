\subsection{The Goal}

We want a 2nd-order (at least) logic,
that is 2-valued (?),
that handles partial expressions, explicit substitutions,
and that gives
the results we want for UTP proofs.
The additional trick is in finding an axiomatization that supports
the equational reasoning we prefer, which means that Liebniz' rule
needs a light touch, and we hope to keep the Deduction Theorem,
and the one-point rules.

In this section we explore various proposed logics%
\cite{Prawitz56,Farmer90,Owe93,JM94,Feferman95,MB98,CMR99},
mostly partial, and typically either first- or higher-order.



The key test cases for us are proofs of the idempotence
of \RR2 and the commutativity of \RR1 and \RR2,
which are shown here for ease of reference.

\newpage
\subsubsection{Proof: \RR2 is idempotent}

The proof of \RR2's idempotence requires us to
use Liebniz inside (explicit) substitution terms,
without any side-conditions regarding the definedness
of the term $tr'-tr$ (defined only if $tr \le tr'$).
\begin{eqnarray*}
  && \RR2(\RR2(P))
\EQV{defn. \RR2, twice}
\\&& \exists s @
      (\exists t @ P[t,t\cat(tr'-tr)/tr,tr'])
      [s,s\cat(tr'-tr)/tr,tr']
\EQV{defn. substitution, flatten quantifiers}
\\&& \exists s,t @
      P[t,t\cat(tr'-tr)/tr,tr'][s,s\cat(tr'-tr)/tr,tr']
\EQV{substitution composition}
\\&& \exists s,t @
      P[t,t\cat((s\cat(tr'-tr))-s)/tr,tr']
\EQV{sequence property, \textbf{ignoring definedness issues for $tr'-tr$}}
\\&& \exists s,t @ P[t,t\cat(tr'-tr)/tr,tr']
\EQV{$s$ no longer mentioned}
\\&& \exists t @ P[t,t\cat(tr'-tr)/tr,tr']
\EQV{defn. \RR2}
\\&& \RR2(P)
\end{eqnarray*}

\subsubsection{Proof: \RR1 and \RR2 commute}

By contrast, the proof of commutativity of \RR1 and \RR2
depends on the definedness side-condition for term $tr'-tr$
being invoked:
\begin{eqnarray*}
  && \RR2(\RR1(P))
\EQV{defns. \RR1, \RR2}
\\&& \exists s @ (P \land tr \le tr')[s,s\cat(tr'-tr)/tr,tr']
\EQV{defn. substitution}
\\&& \exists s @ P[s,s\cat(tr'-tr)/tr,tr'] \land s \le s\cat(tr'-tr)
\EQV{$\sigma \le \sigma\cat\tau \equiv True$, \textbf{with definedness condition for }$tr'-tr$}
\\&& \exists s @ P[s,s\cat(tr'-tr)/tr,tr'] \land True \land tr \le tr'
\EQV{shrink scope, prop. calc.}
\\&& (\exists s @ P[s,s\cat(tr'-tr)/tr,tr']) \land tr \le tr'
\EQV{defns. \RR2, \RR1}
\\&& \RR1(\RR2(P))
\end{eqnarray*}
The key issue of importance here is why we could ignore the definedness
of $tr'-tr$ in the first proof, but had to invoke it in the second.
The underlying intuition is that in the first case,
the potential need to consider definedness was present
in both the before- and after-steps, namely that the reasoning
step kept the troublesome term $tr'-tr$ around.
In the second case, however, the whole term containing $tr'-tr$
is replaced by $True$, and so the implied need to consider the definedness
of this term has vanished, and therefore we are forced to deploy it explicitly.
The challenge is to formalise these two distinct steps and show their
soundness.

Another example,
often cited is the following recursive definition
of partial natural number subtraction:
\begin{eqnarray*}
   subp(i,j) &\defs& \IF~i=j
\\ && \THEN~0
\\&& \ELSE subp(i,j+1)+1
\end{eqnarray*}
and a proof of the following property:
\begin{eqnarray*}
   \forall i,j : \num @ i \ge j \implies subp(i,j) = i-j
\end{eqnarray*}
To this we add the following properties,
analagous to those regarding sequences
that arise in the \RR1/\RR2 proofs above:
\begin{eqnarray*}
  && \forall i,j :\num @ i \ge j \implies subp(i+subp(i,j),i) = subp(i,j)
\\ && \forall i,j : \num @ i \ge j \implies \num @ i \le i + subp(i,j)
\end{eqnarray*}
The question is if we can modify the first
to not require the precondition, \emph{i.e.}:
\begin{eqnarray*}
  ? && \forall i,j :\num @ subp(i+subp(i,j),i) = subp(i,j)
\end{eqnarray*}

\subsection{The Approach}

In what follows, we (hope to )present all the logics in a
standardised fashion, using uniform naming schemes for axioms,
inference rules, derived laws, and meta-theorems, whilst also
giving the mapping to the notations, names and terms used in
the original source material.

After giving the uniform listing,
we kick off with a total/classical 2nd-order logic
and then treat each partial logic
in chronological order of published sources.
