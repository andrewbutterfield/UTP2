Are variables precisely the free ones of a predicates ($V = P$)~?
\\If we only have free variables this is straightforward,
or if there are no such, and only a single predicate meta-variable.
Other cases are complicated and the relevant condition cannot be
expressed in our current side-condition notation.
If we cannot answer in the negative,
then we insist that all meta-variables have the same free variables,
equal to those required.
\begin{eqnarray*}
  V = \setof{v_1,\ldots,v_n} &\equiv& \mbox{the obvious!}
\\ V = ( P \setminus H ) &\impliedby & V = P \land P \DISJ H
\\ V = ( ms \sthen c ) && \textrm{Cannot say without } \rho.
\\ V = ( m \ssthen r ) && \textrm{Cannot say without } \rho.
\end{eqnarray*} 