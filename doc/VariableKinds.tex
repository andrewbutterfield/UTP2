Conceptually,
variables have a root and decoration,
and if list-variables, may also have a list of `subtracted' roots
\begin{eqnarray*}
  (r,d,rs),v \in Var &\defs& Root \times Decor \times Root^*
\end{eqnarray*}
The root is a simple name:
\begin{eqnarray*}
  r \in Root &\defs& Name
\end{eqnarray*}
A decoration is either a pre-marking, a post-marking, or a subscript:
\begin{eqnarray*}
  d \in Decor &\defs&  Pre | Post | PrePost | Subscript~Name
\end{eqnarray*}
We use the notation $(r,d)$ when the subtracted-list is empty or irrelevant.

What has just been presented is the \emph{abstract} form of a variable.

Current concrete rendering:

\begin{tabular}{|c|c|}
  \hline
  Abstract & Concrete 
  \\ \hline
  $(r,Pre)$ & \verb"r"
  \\ \hline
  $(r,Post)$ & \verb"r'"
  \\ \hline
  $(r,PrePost)$ & \verb"r?"
  \\ \hline
  $(r,Subscript~s)$ & \verb"r_s"
  \\ \hline
\end{tabular}

