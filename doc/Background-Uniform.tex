\subsection{Partial Logics --- Uniform Notation}

Here we build up a collection of uniform material
on naming schemes for
axioms, inference rules, derived laws, and meta-theorems,
and valuations.
The presentation of all the background material is then given
using this notation throughout,
with a brief discussion in each section of any differences.

In general we follow the notation and style of the logics
used to present UTP and Z.

\subsubsection{Syntax}

Terms and expressions are built over variables and constants
using function and operator terms, all of which are typed:
\RLEQNS{
   a,b,c,\ldots &\in& Constant      & \elabel{Constant:decl}
\\ \ldots v,w,x,y,z &\in& Variable  & \elabel{Variable:decl}
\\ \ldots X,Y,Z \in VarSet &\defs& \power Variable
   & \elabel{VarSet:def}
\\ f,g,h,\ldots &\in& Function      & \elabel{Function:decl}
\\ +,-,\oplus,\ominus,\ldots &\in& Operator & \elabel{Operator:decl}
}
We do not annotate constants, functions and operators
with types but assume these can be looked up.
Variable types are often inferred:
\RLEQNS{
  e,t \in Expr
  &::=& c | x | f(e_1,\ldots,e_n) | e_1 \oplus e_2
  & \elabel{Expr:def}
}
The usual rules about bracketing and precedence apply.
We assume that the constants include the two boolean values,
and that three versions of equality
(existential,weak and strong) are available:
\RLEQNS{
   \setof{True,False} &\subseteq& Constant & \elabel{Constant:bool}
\\ \setof{=_e,=_w,=_s} &\subseteq& Operator & \elabel{Operator:equal}
}

Atomic formulas are simply applications of boolean-result
functions or operators to non-boolean arguments%
\footnote{rather, at least one argument should be non-boolean,
and the boolean arguments must be constants or variables}%
.
\RLEQNS{
 A,B,C, \ldots &\in& AtomicPred \subseteq Expr
 & \elabel{Atomic:decl}
}

We then define a standard predicate syntax,
except that quantifier variables have type
annotations:
\RLEQNS{
   P,Q,R \in Predicate
   &::=&
   A | \lnot P | P \land Q | P \lor Q
\\&& {} | P \implies Q | P \equiv Q
\\&& {} | \forall \vec x: \vec T @ P
        | \exists \vec x: \vec T @ P
   & \elabel{Predicate:def}
}
Here $\vec x$ denotes a non-empty vector of variables
The type annotations may be omitted if determinable
from context.

We say very little about types but expect them at least
to have (the usual) basic types, as well as function (total \& partial)
and tuple types:
\RLEQNS{
   \mathbb{T} \in BasicType
    &::=&
    \Bool | \num | \ldots
    & \elabel{Basic:decl}
\\ T \in Type
    &::=&
    \mathbb{T} | T_1 \times T_2 | T_1 \fun T_2| T_1 \pfun T_2
    & \elabel{Type:def}
}

\subsubsection{Syntax Functions}

We shall write certain key syntactical functions
using sans-serif fonts, i.e.
\fv, \bv\ (free and bound variables).
We shall also overload notation, so that, for instance,
\fv\ can be applied to $Expr$, $AtomicPred$ and $Predicates$,
and we shall use the type $WFE$ (Well-Formed Expression)
to refer to any of these three.

Free and Bound variables are standard and we do not give
any further details.
\RLEQNS{
   \fv &:& WFE \fun \power Variable & \elabel{fv:sig}
\\ \fv~P &\defs& \mbox{the usual}   & \elabel{fv:def}
\\ \bv &:& WFE \fun \power Variable & \elabel{bv:sig}
\\ \bv~P &\defs& \mbox{the usual}   & \elabel{bv:def}
}
In the event that an operator or function is partial,
we provide a syntactical function
that builds a predicate that characterizes the requisite pre-condition:
\RLEQNS{
   \fpre &:& Function \times Variable \fun Predicate
&  \elabel{ppred:sig}
\\ \fpre(f,v) &\defs& \mbox{appropriate } P \mbox{ where } \fv~P \subseteq \setof v
&  \elabel{ppred:def}
\\ \opre &:& Variable \times Operator \times Variable \fun Predicate
&  \elabel{ppred:sig}
\\ \opre(u,\oplus,v) &\defs& \mbox{appropriate } P \mbox{ where } \fv~P \subseteq \setof{u,v}
&  \elabel{opred:def}
}
We define the notion of substitution:
\RLEQNS{
   Substn &\defs& Variable \pfun Expr &\elabel{Subs:sig}
\\ \_[\_] &:& Expr \times Substn \fun Expr & \elabel{Subs:expr:sig}
\\ e[\theta] &\defs& \mbox{usual notion of substitution} & \elabel{Subs:expr:def}
}
We also define a predicate that checks if a substitution under a set of binders
avoids free-variable capture:
\RLEQNS{
   \nocapture &:& Substn \times \power Variable \fun Bool & \elabel{nocapture:sig}
\\ \nocapture~\theta~X &\defs& \fv(\bigcup(rng~\theta)) \cap X = \emptyset
}

\subsubsection{Basic Semantic Domains}

We assume boolean, three-valued and number domains:
\RLEQNS{
   Bool &\defs& \setof{T,F} & \elabel{Bool:def}
\\ Three &\defs& Bool \union\setof{U} & \elabel{Three:def}
\\ Nat &\defs& \setof{0,1,2,\ldots} & \elabel{Nat:def}
\\ Int &\defs& \setof{\ldots,-2,-1,0,1,2,\ldots} & \elabel{Int:def}
}
Here $Bool$ is a classical 2-value logic domain
with classical operations defined upon it.
$Three$ is for three-valued logic,
and we will standardise the names for all
the different extensions there are from 2- to 3-valued
logic operators.

\subsubsection{General Semantic Functions}

Key semantic functions are given using roman fonts,
e.g. \ctype\ (Constant Type)

We associate type information with every ``constant'',
i.e.  value from $Constant$, $Function$ and $Operator$
\RLEQNS{
   \ctype &:& Constant \fun Type & \elabel{ctype:sig}
\\ \ctype~c &\defs& \mbox{as appropriate for a theory}
   &\elabel{ctype:def}
}
We also have meta-predicates on types that
asserts true if they denote a non-functional type
or a partial function (at the top level):
\RLEQNS{
   \isbasic &:& Type \fun Bool & \elabel{isbasic:sig}
\\ \isbasic(T_1 \fun T_2) &\defs& F & \elabel{isbasic:fun}
\\ \isbasic(T_1 \pfun T_2) &\defs& F & \elabel{isbasic:pfun}
\\ \isbasic(\_) &\defs& T & \elabel{isbasic:other}
\\ \ispartial &:& Type \fun Bool & \elabel{ispartial:sig}
\\ \ispartial~\mathbb{T} &\defs& F & \elabel{ispartial:basic}
\\ \ispartial(T_1 \times T_2) &\defs& F & \elabel{ispartial:prod}
\\ \ispartial(T_1 \fun T_2) &\defs& \ispartial~T_2 & \elabel{ispartial:fun}
\\ \ispartial(T_1 \pfun T_2) &\defs& \isbasic~T_2 & \elabel{ispartial:pfun}
}
We have a relationship that must hold between \ctype, \ispartial\
and \fpre\ and/or \opre: the function/operator type
being partial is equivalent to its pre-condition being non-trivial,
(i.e. not equal to $True$)
\RLEQNS{
  \ispartial(\ctype f)
  &\equiv&
  \fpre(f,\_) \neq True
  &\elabel{partial:f:consistency}
\\ \ispartial(\ctype \oplus)
  &\equiv&
  \opre(\_,\oplus,\_) \neq True
  &\elabel{partial:o:consistency}
}


\subsubsection{Propositional Connectives}

We first present here the 2-valued propositional connectives.

\begin{itemize}
%
\item Negation $\lnot : Bool \fun Bool$
\elabel{Bool:not:sig}
$$\begin{array}{|c|c|}
  \hline
  P & \lnot P \\\hline
  F & T       \\\hline
  T & F       \\\hline
\end{array}
\qquad \elabel{Bool:not:def}
$$
%
\item Conjunction $\_\land\_ : Bool\times Bool \fun Bool$
\elabel{Bool:and:sig}
$$\begin{array}{|c||c|c|}
  \hline
  \land & F & T \\\hline\hline
  F     & F & F \\\hline
  T     & F & T \\\hline
\end{array}
\qquad \elabel{Bool:and:def}
$$
%
\item Disjunction $\_\lor\_ : Bool\times Bool \fun Bool$
\elabel{Bool:or:sig}
$$\begin{array}{|c||c|c|}
  \hline
  \lor  & F & T \\\hline\hline
  F     & F & T \\\hline
  T     & T & T \\\hline
\end{array}
\qquad \elabel{Bool:or:def}
$$
%
\item Implication $\_\implies\_ : Bool\times Bool \fun Bool$
\elabel{Bool:implies:sig}
$$\begin{array}{|c||c|c|}
  \hline
  \implies & F & T \\\hline\hline
  F        & T & T \\\hline
  T        & F & T \\\hline
\end{array}
\qquad \elabel{Bool:implies:def}
$$
%
\item Equivalence $\_\equiv\_ : Bool\times Bool \fun Bool$
\elabel{Bool:equiv:sig}
$$\begin{array}{|c||c|c|}
  \hline
  \land & F & T \\\hline\hline
  F     & T & F \\\hline
  T     & F & T \\\hline
\end{array}
\qquad \elabel{Bool:equiv:def}
$$
\end{itemize}

We now describe the  variety of ways in which a 2-valued
propositional connective can be expanded to the 3-valued world.
We only show the new 3-valued entries, as the 2-valued core is unchanged.

\begin{itemize}
\item whatever $\_?\_ : Three\times Three \fun Three$
\elabel{Three:??:sig}
$$\begin{array}{|c||c|c|c|}
  \hline
  ?     & \bot & F    & T    \\\hline\hline
  \bot  & X    & X    & X    \\\hline
  F     & X    & -    & -    \\\hline
  T     & X    & -    & -    \\\hline
\end{array}
\qquad \elabel{Three:??:def}
$$
\end{itemize}

\subsubsection{Equality and Definedness}
We present the different definitions of equality,
whose inputs may be total or partial,
and whose result may be boolean or three-valued.
In what follows, $a$, $b$ and $\bot$ all denote distinct
elements of some domain type $T$, with $\bot$, if present,
denoting undefinedness.
We also present undefinedness predicates here,
as they and equality are closely linked.

\begin{itemize}
\item Equals $\_=\_ : T\times T \fun Bool$
\elabel{equals:sig}
$$\begin{array}{|c||c|c|}
  \hline
  \land & a & b \\\hline\hline
  a     & T & F \\\hline
  b     & F & T \\\hline
\end{array}
\qquad \elabel{equals:def}
$$
%
\item Existential Equality\cite{CMR99} $\_=_e\_ : T_\bot\times T_\bot \fun Bool$
\elabel{eequals:sig}
$$\begin{array}{|c||c|c|c|}
  \hline
  =_e   & \bot & a & b \\\hline\hline
  \bot  & F & F & F \\\hline
  a     & F & - & - \\\hline
  b     & F & - & - \\\hline
\end{array}
\qquad \elabel{eequals:def}
$$
%
\item Strong Equality\cite{CMR99} $\_=_s\_ : T_\bot\times T_\bot \fun Bool$
\elabel{sequals:sig}
$$\begin{array}{|c||c|c|c|}
  \hline
  =_s   & \bot & a & b \\\hline\hline
  \bot  & T & F & F \\\hline
  a     & F & - & - \\\hline
  b     & F & - & - \\\hline
\end{array}
\qquad \elabel{sequals:def}
$$
Strong equality is just standard equality on the underlying
semantic domain, viewed as flat.
%
\item Weak Equality\cite{CMR99} $\_=_w\_ : T_\bot\times T_\bot \fun Bool$
\elabel{wequals:sig}
$$\begin{array}{|c||c|c|c|}
  \hline
  =_w   & \bot & a & b \\\hline\hline
  \bot  & T & T & T \\\hline
  a     & T & - & - \\\hline
  b     & T & - & - \\\hline
\end{array}
\qquad \elabel{wequals:def}
$$
%
\item Definedness $\_\defined : T_\bot \fun Bool$
\elabel{defined:sig}
\RLEQNS{
  c \defined &\defs& \lnot( c =_s \bot ) & \elabel{defined:def}
}
\end{itemize}

\subsubsection{Axioms}

Here we present a list of all axioms encountered,
with unique names.
This is simply a collection and does not form a coherent
or consistent whole.
Note also that definitions of certain formulas and terms
as abbreviations for others are considered here as axioms.

Abbreviations:
\RLEQNS{
   e\defined &\defs& e =_e e & \elabel{abrv:defd:eeq}
\\ e =_w e' &\defs& (e\defined \land e'\defined) \implies e =_e e'
   & \elabel{abrv:weq:eeq}
\\ e =_s e' &\defs& (e\defined \lor e'\defined) \implies e =_e e'
   & \elabel{abrv:seq:eeq}
\\ e =_w e' &\defs& (e\defined \land e'\defined) \implies e =_s e'
   & \elabel{abrv:weq:seq}
\\ e =_e e' &\defs& e\defined \implies e =_s e'
   & \elabel{abrv:eeq:seq}
\\ e =_e e' &\defs& e\defined \land e'\defined \land e =_w e'
   & \elabel{abrv:eeq:weq:defd}
\\ e =_s e' &\defs& (e\defined \lor e'\defined)
                    \implies
                    e\defined \land e'\defined \land e =_w e'
   & \elabel{abrv:seq:weq:defd}
\\ \lnot P &\defs& P \implies False & \elabel{abrv:not:implies:F}
\\ P \lor Q &\defs& \lnot(\lnot P \land \lnot Q) & \elabel{abrv:or:deMorgan}
\\ P \equiv Q &\defs& (P \implies Q) \land (Q \implies P)
  & \elabel{abrv:equiv:bi-implies}
\\ \exists x @ P &\defs& \lnot(\forall x @ \lnot P) & \elabel{abrv:exists:deMorgan}
}
Derivation Axioms:

\subsubsection{Inference Rules}

Here we present a list of all inference rules encountered,
with unique names.
This is simply a collection and does not form a coherent
or consistent whole.

\subsubsection{Derived Rules}

Here we present a list of all derived rules encountered,
with unique names.
This is simply a collection and does not form a coherent
or consistent whole.


\subsubsection{Valuations/Satisfaction}

Here we present a list of all valuations
and satisfaction relationships encountered,
with unique names.
This is simply a collection and does not form a coherent
or consistent whole.


\subsubsection{Meta-theorems}

Here we present a list of all meta-theorems encountered,
with unique names.
This is simply a collection and does not form a coherent
or consistent whole.
