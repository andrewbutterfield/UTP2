Type matching can only return type-bindings,
so it is simpler than the others.

For the formal presentation we assume the following mathematical
type syntax:
\begin{eqnarray*}
  \rho,\tau \in Type
    & ::= &  B \mbox{ --- Boolean}
  \\ &|& Z \mbox{ --- Integer}
  \\ &|& t \mbox{ --- Type variable}
  \\ &|& ? \mbox{ --- Arbitrary Type}
  \\ &|& P~\tau \mbox{ --- Set Type Constructor}
  \\ &|& \tau^* \mbox{ --- List Type Constructor}
  \\ &|& \tau \times \tau \mbox{ --- Product Type Constructor}
  \\ &|& \tau \fun \tau \mbox{ --- Function Type Constructor}
  \\ &|& !s \mbox{ --- Error Type}
\end{eqnarray*}

We can define inference rules for matching:
\input{doc/formal/Matching-Type-Rules}
The following rules are controversial
(a form of reverse matching):
$$\begin{array}{r@{\qquad}l}
   \MRTVarRN & \MRTVarR
\\ \MRTArbRN& \MRTArbR
\end{array}$$
The rules are intended to check law matches
for type-compatibility.